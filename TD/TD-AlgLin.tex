%-------------------------------------------------------------------------------
\subsection{Matrices et déterminants}%-------------------------------------------------------------------------------

%-------------------------------------------------------------------------------
\begin{exercise} \label{exo:AlgLin-Det33}
  Calculer les déterminants des matrices suivantes 
  \begin{align*}
%     A_1 & = \left[\begin{array}{rrr}
%       2 & -1 & 3 \\ 2 & -1 & 6 \\ -2 & 1 & 0
%       \end{array}\right], &
%     %
%     A_2 & = \left[\begin{array}{rrr}
%       2 & -1 & 3 \\ 2 & -1 & 6 \\ 1 & 0  & 2
%       \end{array}\right], \\
%     %
    A_3 & = \left[\begin{array}{rrr}
      1 & 1 & 0 \\ -5 & -2 & 5 \\ -1 & 0 & 2
      \end{array}\right], &
    A_4 & = \left[\begin{array}{rrr}
      -1 & 3 & 1 \\ 0 & 2 & 1 \\ 2 & 1 & 2
      \end{array}\right].  
  \end{align*}
\end{exercise}
\solution{
  \begin{align*}
%     |A_1| & = 0 & & \text{colonne } 1 = -2 \text{ colonne } 2 \\
%     %
%     |A_2| & = 1 \times (-3) + 2 \times 0 = -3 & & \text{développement / dernière ligne} \\
    %
    |A_3| & = 1 \times (-4) - 1 \times (-5) = 1 & & \text{développement / première ligne} \\
    %
    |A_4| & = -1 \times 3 + 2 \times 1 = -1 & &  \text{développement / première colonne}
  \end{align*}
}

%-------------------------------------------------------------------------------
\begin{exercise} \label{exo:AlgLin-PolCar33}
  Déterminer le polynôme caratéristique des matrices suivantes et en déduire si elles sont diagonalisables
  \begin{align*}
    A_1 & = \left[\begin{array}{rrr}
      2 & -1 & 3 \\ 2 & -1 & 6 \\ 1 & 0  & 2
      \end{array}\right], &
    A_2 & = \left[\begin{array}{rrr}
      1 & 1 & 0 \\ -5 & -2 & 5 \\ -1 & 0 & 2
      \end{array}\right], \\ 
  \end{align*}
\end{exercise}

\solution{
  \begin{description}
    \item[$A_1$ :] on a
    \begin{align*}
        P_A(\lambda) 
        = \left| \begin{array}{rrr}
          2 - \lambda & -1 & 3 \\ 2 & -1 - \lambda & 6 \\ 1 & 0  & 2 - \lambda 
          \end{array}\right|
        = -\lambda^3 + 3 \lambda^2 + \lambda - 3
        = -(\lambda - 3) (\lambda - 1) (\lambda + 1).
    \end{align*}
    Les valeurs propres sont donc $3$, $1$ et $-1$ qui sont toutes réelles et distinctes, donc $A_1$ est diagonalisable.
    \item[$A_1$ :] on a
    \begin{align*}
        P_A(\lambda) 
        = \left| \begin{array}{rrr}
            1 - \lambda & 1 & 0 \\ -5 & -2 - \lambda & 5 \\ -1 & 0 & 2 - \lambda
          \end{array}\right|
        = - \lambda^3 + \lambda^2 - \lambda + 1
        = - (\lambda-1) (\lambda^2 + 1).
    \end{align*}
    Les valeurs propres sont donc $1$, $i$ et $-i$ qui ne sont pas toutes réelles, donc $A_2$ n'est pas diagonalisable.
  \end{description}
}

%-------------------------------------------------------------------------------
\begin{exercise} \label{exo:AlgLin-Trace}
  On définit la trace $\tr(A)$ d'une matrice carrée $A \in \Mcal_n$ comme la somme de ses termes diagonaux : $\tr(A) = \sum_{i=1}^n a_{ii}$.
  \begin{enumerate}
   \item Montrer que le coefficient d'ordre $n-1$ du polynôme caratéristique de $A$ (noté $P_A$) vaut
  $$
  [\lambda^{n-1}] P_A(\lambda) = (-1)^{n-1} \tr(A).
  $$
  On pourra procéder par récurrence.
  \item En déduire que la trace est égale à la somme des raçines $\{\lambda_1, \dots \lambda_n\}$ de $P_A(\lambda)$ :
  $$
  \tr(A) = \sum_{i=1} \lambda_i
  $$
  On pourra utiliser la factorisation du polynôme caractéristique.
  \end{enumerate}
\end{exercise}

\solution{
  \begin{enumerate}
    \item On vérifie facilement que la propriété est vraie pour $n=2$. En la supposant vraie au rang $n-1$, on peut calculer le polynôme caractéristique de $A \in \Mcal_n$ en développant par la dernière ligne : 
    $$
    P_A(\lambda) = 
    \left| A - \lambda I\right|
    = \sum_{j=1}^{n-1} (-1)^{n+j} a_{nj} \left| (A - \lambda I)^{(n, j)} \right|
    + (a_{nn} -\lambda) \left| (A - \lambda I)^{(n, n)} \right|
    $$
    en notant $B^{(i,j)}$ la matrice $B$ privée de sa $i$ème ligne et $j$ème colonne. On peut alors remarquer que les termes de la première somme sont tous de degré au plus $n-2$ et que $\left|(A - \lambda I)^{(n, n)} \right| = P_{A^{(n, n)}}(\lambda)$. On a donc, en notant $Q_m$ un polynôme quelconque de degré inférieur ou égal à $m$,
    \begin{align*}
      P_A(\lambda) 
      & = (a_{nn} -\lambda) P_{A^{(n, n)}}(\lambda) + Q_{n-2}(\lambda) \\
      & = (a_{nn} -\lambda) \left( (-1)^{n-1} \lambda^{n-1} + (-1)^{n-2} \tr(A^{(n, n)} \lambda^{n-2} + Q_{n-3}(\lambda) \right) 
      & & (\text{par hypothèse}) \\
      & = (-1)^n \lambda^n + (-1)^{n-1} \underset{\tr(A)}{\underbrace{(a_{nn} + \tr(A^{(n, n)})}} \lambda^{n-1} + Q_{n-2}(\lambda)
    \end{align*}
    donc
    $$
    [\lambda^{n-1}] P_A(\lambda) = (-1)^{n-1} \tr(A).
    $$
    \item On utilise cette fois la version factorisée de $P_A$, soit 
    $$
    P_A(\lambda) 
    = \prod_{i=1}^n (\lambda_i - \lambda)
    $$
    où les $\lambda_i$ sont les $n$ valeurs propres (pas nécessairement distinctes ni réelles) de $A$. Lors du développement de $P_A(\lambda)$, les termes en $\lambda^{n-1}$ apparaissent en multipliant un terme $\lambda_i$ par $n-1$ termes $(-\lambda)$, c'est-à-dire
    $$
    \left( \sum_{i=1}^n \lambda_i \right) (-\lambda)^{n-1}
    = (-1)^{n-1} \left( \sum_{i=1}^n \lambda_i \right) \lambda^{n-1}.
    $$
    On a donc
    $$
    [\lambda^{n-1}] P_A(\lambda) 
    = (-1)^{n-1} \left( \sum_{i=1}^n \lambda_i \right) 
    \qquad \Leftrightarrow \qquad
    \tr(A) = \sum_{i=1}^n \lambda_i.
    $$
  \end{enumerate}
}
 
%-------------------------------------------------------------------------------
\subsection{Calcul du déterminant par la méthode des cofacteurs}%-------------------------------------------------------------------------------

\begin{exercise}
  Montrer que, pour toute matrice $A \in \Mcal_n$ et pour tout $i_0, j_0 \in \{1, \dots, n\}$, on a 
  \begin{align*}
    |A| 
    & = \sum_{j=1}^n a_{i_0j} (-1)^{i_0+j} |A^{(i_0j)}| & (\text{développement par rapport à la ligne $i_0$}) \\
    & = \sum_{i=1}^n a_{ij_0} (-1)^{i+j_0} |A^{(ij_0)}| & (\text{développement par rapport à la colonne $j_0$})
  \end{align*}
\end{exercise}

\todo{Ajouter une question intermédiaire avec $i_0 = 1$ ?} 

\solution{
  On considère le développement par rapport à la ligne $i_0$. Par multilinéarité, on a
  $$
  |A| = \sum_{j=1}^n a_{i_0j} 
    \left|\begin{array}{ccccc}
      a_{11} & \cdots & a_{1j} & \cdots & a_{1n} \\
      \vdots & & \vdots & & \vdots \\
      0 & \cdots & 1 & \cdots & 0 \\
      \vdots & & \vdots & & \vdots \\
      a_{n1} & \cdots & a_{nj} & \cdots & a_{nn} \\
    \end{array}\right|.
  $$
  On effectue ensuite les $j-1$ interversions de colonnes adjacentes amenant la colonne $j$ en colonne 1, {\em en préservant l'ordre des autres colonnes entres elles}. Du fait du caractère alterné du déterminant, chaque interversion engendre un changement de signe : 
  $$
  |A| = \sum_{j=1}^n (-1)^{j-1} a_{i_0j} 
    \left|\begin{array}{ccccccc}
      a_{1j} & a_{11} & \cdots & a_{1,j-1} & a_{1,j+1} & \cdots & a_{1n} \\
      \vdots & \vdots & & \vdots & \vdots & & \vdots \\
      1 & 0 & \cdots & 0 & 0 & \cdots & 0 \\
      \vdots & \vdots & & \vdots & \vdots & & \vdots\\
      a_{nj} & a_{n1} & \cdots & a_{n,j-1} & a_{n,j+1} & \cdots & a_{nn} \\
    \end{array}\right|.
  $$
  On effectue la même opération pour amener la ligne $i_0$ en premier
  $$
  |A| = \sum_{j=1}^n (-1)^{(j-1) + (i_0-1)} a_{i_0j} 
    \left|\begin{array}{ccccccc}
      1 & 0 & \cdots & 0 & 0 & \cdots & 0 \\
      a_{1j} & a_{11} & \cdots & a_{1,j-1} & a_{1,j+1} & \cdots & a_{1n} \\
      \vdots & \vdots & & \vdots & \vdots & & \vdots \\
      a_{i_0-1, j} & a_{i_0-1, 1} & \cdots & a_{i_0-1,j-1} & a_{i_0-1, j+1} & \cdots & a_{i_0-1, n} \\
      a_{i_0+1, j} & a_{i_0+1, 1} & \cdots & a_{i_0+1, j-1} & a_{i_0+1, j+1} & \cdots & a_{i_0+1, n} \\
      \vdots & \vdots & & \vdots & \vdots & & \vdots\\
      a_{nj} & a_{n1} & \cdots & a_{n,j-1} & a_{n,j+1} & \cdots & a_{nn} \\
    \end{array}\right|.
  $$
  On utilise alors la formule du déterminant par bloc
  $$
  |A| = \sum_{j=1}^n (-1)^{i_0 + j} a_{i_0j} |[1]| \times
    \left|\begin{array}{cccccc}
      a_{11} & \cdots & a_{1,j-1} & a_{1,j+1} & \cdots & a_{1n} \\
      \vdots & & \vdots & \vdots & & \vdots \\
      a_{i_0-1, 1} & \cdots & a_{i_0-1,j-1} & a_{i_0-1, j+1} & \cdots & a_{i_0-1, n} \\
      a_{i_0+1, 1} & \cdots & a_{i_0+1, j-1} & a_{i_0+1, j+1} & \cdots & a_{i_0+1, n} \\
      \vdots & & \vdots & \vdots & & \vdots\\
      a_{n1} & \cdots & a_{n,j-1} & a_{n,j+1} & \cdots & a_{nn} \\
    \end{array}\right|
  $$
  où on reconnaît les mineurs $A^{(i_0, j}$ et les cofacteurs $(-1)^{i_0 + j} |A^{(i_0, j}|$. \\
  La démonstration pour le développement par rapport à une colonne est symétrique.
}
 
%-------------------------------------------------------------------------------
\subsection{Inverse d'un matrice orthonormale}%-------------------------------------------------------------------------------

\begin{exercise} \label{exo:AlgLin-Trace}
  Soit $P \in \Mcal_n$ orthonormale. Montrer que $P^{-1} = P^\top$.
\end{exercise}

\solution{
  En notant $P = [v_{ij}]$, $v_j$ le $j$ème vecteur colonne de $P$ et $B = [b_{ij}] = P^\top P$, on a
  $$
  P^\top = [v_{ji}] 
  \quad \Rightarrow \quad
  b_{ik} 
  = \sum_{k=1}^n [P^\top]_{ik} [P]_{kj} 
  = \sum_{k=1}^n v_{ki} v_{kj}
  = < v_i, v_j >
  = \left\{\begin{array}{rl} 1 & \text{si } i = j \\ 0 & \text{sinon} \end{array}\right.
  $$
  (puique les vecteurs $v_j$ sont orthonormés), donc $P^\top P = B = I$. La démonstration de $P P^\top = I$ est symétrique.
}
%-------------------------------------------------------------------------------
\subsection{Dynamique d'une population structurée}%-------------------------------------------------------------------------------
 \todo{Voir notes}
 
%-------------------------------------------------------------------------------
\subsection{Analyse en composante principales}%-------------------------------------------------------------------------------
 \todo{Voir notes + $X_i = U + V_i \Rightarrow \Sigma = \sigma^2 J + \gamma^2 I \Rightarrow$ 1 seule vp non nulle $\lambda = \gamma^2 + n \sigma^2$.}
 
% %-------------------------------------------------------------------------------
% \subsection{Matrice de rotation}
% %-------------------------------------------------------------------------------
% \todo{}
