%-------------------------------------------------------------------------------
\subsubsection{Equivalence des normes}
%-------------------------------------------------------------------------------

Donner des constantes $c_1$ et $c_2$ permettant de comparer les trois normes $\|\cdot\|_1$, $\|\cdot\|_2$ et $\|\cdot\|_\infty$ dans $\Rbb^n$.
\solution{On considère les trois paires de normes.
\begin{description}
  \item[$\|x\|_2$ et $\|x\|_\infty$:] On a vu en cours que
  $$
  \|x\|_\infty \leq \|x\|_2 \leq \sqrt{n} \|x\|_\infty.
%   \qquad \Rightarrow \qquad
%   \frac1{\sqrt{n}} \|x\|_2 \leq \|x\|_\infty \leq \|x\|_2. 
  $$
  \item[$\|x\|_1$ et $\|x\|_\infty$:]  On voit facilement que 
  $$
  \|x\|_\infty \leq \|x\|_1 \leq n \|x\|_\infty.
%   \qquad \Rightarrow \qquad
%   \frac1n \|x\|_1 \leq \|x\|_\infty \leq \|x\|_1.
  $$
  \item[$\|x\|_1$ et $\|x\|_2$:]  On peut déduire une comparaison de $\|x\|_1$ et $\|x\|_2$ en combinant ces résultats
  \begin{align*}
  & \|x\|_1 \leq n \|x\|_\infty \leq n \|x\|_2 \leq n^{3/2} \|x\|_\infty \leq n^{3/2} \|x\|_1 \\
  \Rightarrow \qquad
  & \frac1n \|x\|_1 \leq \|x\|_2 \leq \sqrt{n} \|x\|_1.
%   \Rightarrow \qquad
%   & \frac1{\sqrt{n}} \|x\|_2 \leq \|x\|_1 \leq n \|x\|_2.
  \end{align*}
  On peut cependant affiner les deux constantes :
  \begin{description}
    \item[$\|x\|_2 \leq c_1 \|x\|_1$ :] en définissant, pur $1 \leq i \leq n$, les vecteurs $u_i = x_i e_i$ (où les $e_i$ sont les vecteurs de la base canonique), l'inégalité triangulaire pour la norme $\|\cdot\|_2$ implique que
    $$
    \|x\|_2 \leq \sum_i \|u_i\|_2 = \sum_i |x_i| = \|x\|_1 ;
    $$
    \item[$\|x\|_1 \leq c_2 \|x\|_2$ :] en utilisant le fait que $2ab \leq a^2 + b²$, il vient
%     la boule unité de la norme $\|\cdot\|_2$ est incluse dans la boule de rayon $\sqrt{n}$ de la norme $\|\cdot\|_1$, il vient que $\|x\|_1 \leq \sqrt{n} \|x\|_2$.
    \begin{align*}
      \|x\|_1^2
      & = \left(\sum_i |x_i|\right)^2
      =  \sum_i |x_i|^2 + \sum_{i < j} 2 |x_i||x_j| \\ 
      & \leq \sum_i |x_i|^2 + \sum_{i < j} (|x_i|^2 + |x_j|^2) 
      = \sum_i |x_i|^2 + \sum_{i \neq j} |x_i|^2 \\
      & = n \sum_i |x_i|^2 = n \|x\|_2^2,
    \end{align*}
    c'est-à-dire : $\|x\|_1 \leq \sqrt{n} \|x\|_2$ ;
  \end{description}
  soit au total
  $$
  \frac1{\sqrt{n}} \|x\|_1 \leq \|x\|_2 \leq \|x\|_1.
  $$
\end{description}
On peut vérifier que les constantes sont optimales en donnant des exemples pour lesquels chacune des égalités est obtenue. Considérons le vecteur $x$ dont toutes les coordonnées sont égales et le vecteur $y$ dont toutes les coordonnées sont nulles, sauf une (par exemple, la première) :
$$
x = a 1_n, \qquad y = b e_1.
$$
On a :
\begin{align*}
  & \text{pour } \|\cdot\|_2 \text{ et } \|\cdot\|_\infty : & 
  \|x\|_2 & = \sqrt{n} \|x\|_\infty, & 
  \|y\|_2 & = \|y\|_\infty; \\
  & \text{pour } \|\cdot\|_1 \text{ et } \|\cdot\|_\infty : & 
  \|x\|_1 & = n \|x\|_\infty, & 
  \|y\|_1 & = \|y\|_\infty; \\
  & \text{pour } \|\cdot\|_1 \text{ et } \|\cdot\|_2 : & 
  \|y\|_2 & = \|y\|_1, &
  \|x\|_1 & = \sqrt{n} \|x\|_2.
\end{align*}
}

