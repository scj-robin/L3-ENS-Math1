%-------------------------------------------------------------------------------
\subsubsection{Système dynamique en $y^5$ \todo{}}
%-------------------------------------------------------------------------------

On souhaite déterminer les points d'équilibre (et leur nature) du système
$$
\dot y = F(y) = \mu y + 2 y^3 - y^5.
$$
\solution{
On a 
$$
F(y) = y(\mu + y^2 - y^4)
$$
qui s'annule pour $y = 0$ et pour les solutions de $(\mu + 2 y^2 - y^4)$. En posant, $z = y^2$, $\mu + 2 z - z^2 = 0$ admet des solutions si $\Delta =  4(1 + \mu) \geq 0$, soit $\mu \geq -1$. Ces solutions sont alors $z^* = -1 \pm \sqrt{1+\mu}$. La seule solution possiblement positive est $z^* = -1 + \sqrt{1+\mu}$ et elle l'est ssi $\mu > 1$. \\
Le système admet donc un unique point fixe $x^*=0$ si $\mu < 1$ et un second point fixe $x^* = -1+\sqrt{1+\mu}$ si $\mu > 1$. \\
On a de plus
$$
F'(x) = \mu + 3 y^2 - 5y^4
$$
dont le signe est celui de $\mu$ pour $x=0$ et \todo{nature de $x^* = -1+\sqrt{1+\mu}$ si $\mu > 1$.}
}
