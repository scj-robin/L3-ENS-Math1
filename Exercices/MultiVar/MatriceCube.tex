%-------------------------------------------------------------------------------
\subsubsection{Cube d'une matrice}
%-------------------------------------------------------------------------------

Soit la fonction
$$
\begin{array}{rrcl}
  f :  & \Mcal_n(\Rbb) & \mapsto & \Mcal_n(\Rbb) \\
  & X & \rightarrow & X^3.
\end{array}
$$

\begin{enumerate}
%   \item Démontrer que $f$ partout différentiable.
%     \solution{
%     }
  %
  \item Déterminer l'application linéaire tangente à $f$, $D_X f$, en une matrice $X$ de $\Mcal_n(\Rbb)$
    \solution{
      On a
      $$
      f(X+H) = X^3 + X^2H + XHX + HX^2 + H^2X + HXH + XH^2 + H^3,
      $$
      où
      \begin{align*}
        H^2X & = \left[\sum_{k=1}^n \sum_{\ell=1}^n h_{ik} h_{k\ell} x_{\ell j}\right]_{1 \leq i, j \leq n}, &
        HXH & = \left[\sum_{k=1}^n \sum_{\ell=1}^n h_{ik} x_{k\ell} h_{\ell j}\right]_{1 \leq i, j \leq n}, \\ 
        XH^2 & = \left[\sum_{k=1}^n \sum_{\ell=1}^n x_{ik} h_{k\ell} h_{\ell j}\right]_{1 \leq i, j \leq n}, &
        H^3 & = \left[\sum_{k=1}^n \sum_{\ell=1}^n h_{ik} h_{k\ell} h_{\ell j}\right]_{1 \leq i, j \leq n}.\\
      \end{align*}
      Comme pour l'étude de la fonction $f(X) = X^2$, on observe que, pour tout $1 \leq i, j \leq n$ :
      $$
      \left|\sum_{k=1}^n \sum_{\ell=1}^n h_{ik} h_{k\ell} x_{\ell j}\right| \leq n^2 \|X\|_\infty \|H\|^2_\infty =  o(\|H\|_\infty)
      $$
      et de même pour les termes génériques respectifs de $HXH$ et de $XH^2$, et que
      $$
      \left|\sum_{k=1}^n \sum_{\ell=1}^n h_{ik} h_{k\ell} h_{\ell j}\right| \leq n^2 \|H\|^3_\infty =  o(\|H\|^2_\infty).
      $$
      On a donc 
      $$
      f(X+H) - f(X) = X^2H + XHX + HX^2 + o(\|H\|)
      $$
      où $X^2H + XHX + HX^2$ est linéaire en $H$. L'application linéaire tangente est donc
      $$
      \begin{array}{rrcl}
        L_X : & \Mcal_n(\Rbb) & \mapsto & \Mcal_n(\Rbb) \\
        & H & \rightarrow & X^2H + XHX + HX^2.
      \end{array}
      $$
    }
\end{enumerate}

\solution{
}
