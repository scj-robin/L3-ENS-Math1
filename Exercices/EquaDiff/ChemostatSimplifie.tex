% %-------------------------------------------------------------------------------
\subsubsection{Chémostat (simplifié)}
%-------------------------------------------------------------------------------

% Voir \url{https://umr5558-shiny.univ-lyon1.fr/web/}

On considère le modèle du chemostat qui décrit l'évolution d'une population bactérienne au cours du temps dans un milieu alimenté par un flux constant de nutriments. 
Plus précisément, la population bactérienne se développe dans une enceinte close alimentée d'un côté par un flux constant de nutriments et dont les nutriments peuvent également s'évacuer à proportion de leur quantité.

On note
\begin{itemize}
  \item $x(t)$ la taille de la population bactérienne au temps $t$ et
  \item $y(t)$ la quantité de nutriments présente dans le milieu au temps $t$.
\end{itemize}
On considère ici une version simplifiée du système différentiel contrôlant l'évolution de ces deux variables au cours du temps et qui s'écrit
$$
\left\{
  \dot x = \displaystyle{a x y - x}, 
  \qquad
  \dot y = \displaystyle{- x y - y + b}
\right\}
$$
où on suppose que les paramètres $a$ et $b$ sont positifs ou nuls : $a, b \geq 0$.

\bigskip
\begin{enumerate}
  \item Interpréter les différents termes de ce modèle ainsi que ses paramètres.
  \solution{
    \begin{itemize}
     \item[$\dot x$ :] 
     Le terme $a x y$ induite une croissance proportionnelle à la fois à la taille de la population $x$ et aux nutriments disponibles $y$ : $a$ est le paramètre d'efficacité de la consommation des nutriments par les bactéries. \\
     Le terme $-x$ décrit la décroissance (exponentielle) de la population en l'absence de nutriments.
     \item[$\dot y$ :] 
     Le terme $- x y$ décrit la consommation des nutriments par les bactéries. \\
     Le terme $- y$ décrit leur évacuation (à vitesse proportionnelle à leur quantité). \\
     $b$ est le seul terme positif, qui décrit l'apport constant de nutriments dans le système.
    \end{itemize}
  }
  %
  \item Montrer que les points stationnaires du système sont.
  $$
  A = (0, b) \qquad \text{et} \qquad B = \left(ab - 1, 1/a\right).
  $$
  \solution{
  On annule simultanément $F_1(x, y) = axy - x$ et $F_2(x, y) = -xy -y +b$. 
  \begin{itemize}
   \item $F_1(x, y)$ est nul pour $x = 0$ et $y = 1 / a$. 
   \item $F_2(x, y)$ est nul ssi $xy = (b-y)$. Si $x = 0$ alors cela impose $y = b$, ce qui donne le point $A$. Si $x > 0$, alors $y = 1 / a$ (puisque $F_1(x, y)$ doit être nulle) et donc $F_2(x, 1/a)$ est nulle si
   $$
   x / a = b - 1/a
   \qquad \Leftrightarrow \qquad
   x = a b -1
   $$
   ce qui donne le point $B$.
  \end{itemize}
  }
  %
  \item \label{q:chemostatDistincts} Donner les conditions sur $a$ et $b$ qui assurent que les deux équilibres correspondent à des états distincts et biologiquement pertinents (c'est-à-dire avec $x$ et $y$ positifs ou nuls). 
  \solution{
    $a$ et $b$ sont positifs ou nuls par hypothèse.
    Les coordonnées de $A$ et $B$ sont donc positives si $ab \geq 1$ et les point sont distincts dès que $ab \neq 1$. Il suffit donc que $ab > 1$.
    }    
  %
  \item Sous ces conditions, pour quelles valeurs de $(x, y)$ la taille de la population est-elle en croissance ($\dot x > 0$) ? Même question pour la quantité de nutriments ($\dot y > 0$).
  \solution{
    Si la taille de la population est nulle ($x = 0$), sa croissance l'est également. Si $x > 0$ alors on a $\dot x > 0$ pour $a y > 1$, soit $y > 1 / a$. \\
    La quantité de nutriment croît si $y(-x-1) + b > 0$, c'est à dire pour $y < f(x) = b / (x + 1)$, où $f$ est une fonction décroissante ($f'(y) < 0$) et convexe $(f''(y) > 0$) qui passe nécessairement par les points stationnaires $A$ et $B$.
  }
  \item Tracer le diagramme indiquant l'orientation du vecteur gradient $(\dot x, \dot y)$ dans le quadrant $\{x \geq 0, y \geq 0\}$.
  \solution{
  $$
  \includegraphics[width=.5\textwidth, trim=0 10 30 50, clip=]{ChemostatSimplifie-gradient}
  $$
  }
  %
  \item Déterminer la matrice jacobienne du système.
  \solution{
    On a 
    $$
    J_{x, y} = \left(\begin{array}{cc}
              ay - 1 & a x \\
              -y & -x -1 
        \end{array}\right).
    $$
  }
  %
  \item \'Etudier la stabilité de chacun des points d'équilibre en supposant que les conditions établies à la question \ref{q:chemostatDistincts} sont vérifiées.
  \solution{
  \begin{description}
    \item[$A = (0, b)$ :] La jacobienne vaut alors
    $$
    J_A = \left(\begin{array}{cc}
                ab - 1 & 0 \\
                -b & - 1 
          \end{array}\right)
    $$
    qui est triangulaire avec des termes diagonaux distincts : $J_A$ est donc diagonalisable et ses valeurs propres sont ses termes diagonaux. \\
%     Comme, de plus, 
% %     \begin{align*}
% %       \tr(J_A) & = \frac{ab}{1+b} - 2 = \frac{}{1+b}, \\
% %       \det(J_A) & = 1 - \frac{ab}{1+b} = \frac{1 - b(a-1)}{1+b} \leq 0
% %     \end{align*}
%     $$
%     \det(J_A) = 1 - \frac{ab}{1+b} = \frac{1 - b(a-1)}{1+b} < 0
%     $$
%     du fait des conditions imposées aux paramètres $a$ et $b$, 
    Les deux valeurs propres de $J_A$ sont donc $\lambda_1 = -1$ et $\lambda_2 = ab - 1 > 0$, qui sont de signes opposés : $A$ est donc un équilibre instable.
    \item[$B = (ab - 1, 1/a)$ :] La jacobienne vaut alors
    $$
    J_B = \left(\begin{array}{cc}
                0 & a (ab - 1) \\
                -1/a & -ab 
          \end{array}\right)
    $$
    Son polynôme caractéristique est
    $$
    P_B(\lambda) 
    = \left|\begin{array}{cc}
              -\lambda & a (ab - 1) \\
              -1/a & -ab - \lambda
            \end{array}\right|
    = \lambda^2 + a b \lambda + (ab - 1)
    = (\lambda +1) (\lambda + (ab - 1))
    $$
  Les deux valeurs propres de $J_B$ sont donc $\lambda_1 = - 1$ et $\lambda_2 = - (ab - 1)$, toutes deux négatives : $B$ est donc un équilibre stable.
  \end{description}
  }
  %
  \item Décrire qualitativement la trajectoire du système partant d'une population de petite taille $x(0) > 0$ et en l'absence de nutriment ($y(0) = 0$).
  \solution{
  \begin{itemize}
    \item La population commence par décroître ($\dot x < 0$) sans s'annuler, alors que la quantité de nutriments augmente ($\dot y > 0$). \textcolor{gray}{($x$ ne s'annule pas car, en l'absence de nutriment ($y = 0)$, $x$ est solution de $\dot x  = -x$, dont la solution $x(t) = x(0) e^-t$. L'apparition des nutriments ajoute un terme positif à $\dot x$ : $x(t)$ est donc minoré par $x(0) e^{-t}$, qui  ne s'annule jamais.)}
    \item Quand la quantité de nutriments atteint le seuil $y = 1/(a-1)$, la population commence à augmenter : $\dot x > 0$.  
    \item Quand la population dépasse le seuil $x = f(y)$, elle continue à croître, mais les nutriments se mettent à décroître, jusqu'à ce que le système atteigne le seul équilibre stable, situé en $B$.
  \end{itemize}
  $$
  \includegraphics[width=.5\textwidth, trim=0 10 30 50, clip=]{ChemostatSimplifie-path}
  $$
  }
\end{enumerate}
