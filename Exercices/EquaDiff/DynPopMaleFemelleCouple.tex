%-------------------------------------------------------------------------------
\subsubsection{Dynamique de population à trois classes : mâles, femelles et couples}
%-------------------------------------------------------------------------------

On considère une population sexuée panmictique, au sein de laquelle on désigne
respectivement par $x(t)$, $y(t)$ et $z(t)$ les densités au temps $t$ de femelles flottantes, de mâles flottants, et de couples. On suppose que la dynamique de la population respecte le système dynamique suivant
\begin{equation} \label{eq:Dyn3Pop}
  \left\{\begin{array}{rcl}
          \dot x(t) & = & - \alpha x y + r z, \\
          \dot y(t) & = & - \alpha x y + r z, \\
          \dot z(t) & = & + \alpha x y - c z^2,
          \end{array} \right.
\end{equation}
où les coefficients $\alpha$, $r$ et $c$ sont strictement positifs.
\begin{enumerate}
  \item Interpréter ces équations et la signification de chacun des coefficients $\alpha$, $r$ et $c$.
  \solution{Les mâles et femelles flottant(e)s s'apparient pour former des couples : 
  \begin{itemize}
    \item $\alpha$ est le taux de formation des couples, 
    \item $r$ est le taux de natalités de mâles et des femelles (supposés égaux),
    \item $c$ est le taux de mortalité des couples.
  \end{itemize}
  }
  \item En notant $S = x(0) - y(0)$, montrer que $x(t) - y(t) = S$ pour tout temps $t$. En
  déduire les fonctions $y$ et $z$ satisfont le système 
  \begin{equation} \label{eq:Dyn3Pop2}
    \left\{\begin{array}{rcl}
            \dot y(t) & = & - \alpha (y^2 + Sy) + r z, \\
            \dot z(t) & = & + \alpha (y^2 + Sy) - c z^2.
            \end{array} \right.
  \end{equation}
  Dans la suite on supposera que $S > 0$.
  \solution{On remarque que
  $$
  \dot x(t) - \dot y(t) = 0,
  $$
  ce qui implique que la différence $x(t) - y(t)$ reste constante au cours du temps et égale à $x(0) - y(0) = S$. \\
  On peut donc remplacer $x(t) = y(t) + S$ dans le système \eqref{eq:Dyn3Pop} pour obtenir le système \eqref{eq:Dyn3Pop2}.}
  \item Déterminer les points d'équilibre du système \eqref{eq:Dyn3Pop2}.
  \solution{
  \begin{itemize}
    \item $(y^* = 0, z^* = 0)$ est un équilibre (trivial).
    \item $(y = -S, z^* = 0)$ n'est pas un équilibre intéressant du point de vue du modèle car on s'intéresse aux effectifs positifs ou nuls. 
    \item Si on suppose $({y^*}^2 + Sy^*) \neq 0$, il vient
    $$
    \alpha({y^*}^2 + Sy^*) = rz^* = c{z^*}^2 
    \qquad \Rightarrow \qquad 
    z^* = r / c
    $$
    et $y^*$ doit vérifier
    $$
    {y^*}^2 + Sy^* - \frac{r^2}{\alpha c} = 0,
    $$
    dont le discriminant est 
    $$
    \Delta =  S^2 + \frac{4 r^2}{\alpha c} > S^2,
    $$
    et dont la seule solution positive est
    $$
    y^* = \frac{\sqrt{\Delta} - S}2.
    $$
    Le second équilibre intéressant est donc $(y^* = (\sqrt{\Delta} - S)/2, z^* = r/c)$. \\
    $(y^* = (-\sqrt{\Delta} - S)/2, z^* = r/c)$ est bien un point d'équilibre, mais sans intérêt du point de vue du modèle.
  \end{itemize}
  }
  \item \'Ecrire la matrice jacobienne du système \eqref{eq:Dyn3Pop2} et étudier la nature du ou des équilibres non triviaux.
  \solution{La jacobienne vaut
  $$
  J = \left[\begin{array}{rr}
              -2 \alpha y - \alpha S & r \\ 2 \alpha y + \alpha S & -2 c z
            \end{array}\right]
  $$
  \begin{description}
    \item[En $(0, 0)$ :] on a 
    $$
    J_{(0, 0)} = \left[\begin{array}{rr}
                - \alpha S & r \\ \alpha S & 0
              \end{array}\right]
    \qquad \Rightarrow \qquad
    P(\lambda) = \lambda^2 + \alpha S \lambda - r \alpha S
    $$
    où
    $$
    \Delta_0 = \alpha^2 S^2 (1 - 4 r / (\alpha S)).
    $$
    \begin{itemize}
      \item Si $\Delta_0 \geq 0$, les deux valeurs propres
      $$
      \lambda = \frac{\alpha S}2 \left(-1 \pm \sqrt{1 - 4r/(\alpha S)}\right)
      $$
      sont négatives (car $\sqrt{1 - 4r/(\alpha S)} < 1$) et $(0, 0)$ est un équilibre stable. 
      \item Si $\Delta_0 < 0$, la partie réelle ($-\alpha S/2$) des deux valeurs propres est négative et l'équilibre est également stable.
    \end{itemize}
    \item[En $(y^* = \sqrt{\Delta} - S)/2, z^* = r/c)$ :] on a 
    $$
    J_{(y^*, z^*)} = \left[\begin{array}{rr}
                - \alpha \delta & r \\ \alpha \delta & -2 r
              \end{array}\right]
    \qquad \Rightarrow \qquad
    P(\lambda) = \lambda^2 + (\alpha \delta + 2r) \lambda - r \alpha \delta,
    $$
    en notant $\delta = 2(\sqrt{\Delta} - S) + S = 2\sqrt{\Delta} - S > 0$. On a cette fois
    $$
    \Delta^* 
    = \frac12 \left((\alpha \delta + 2r)^2 - 4 r \alpha \delta\right)
    = \frac12 (\alpha \delta - 2r)^2 \geq 0,
    $$
    soit
    $$
    \lambda = \frac12 \left(-(\alpha \delta + 2r) \pm \sqrt{\Delta^* }\right) \leq 0
    $$
    car, les coefficients $\alpha$, $r$ et $c$ étant tous positifs,
    $$
    \frac12 (\alpha \delta - 2r)^2 < (\alpha \delta + 2r)^2.
    $$
    $(y^* = \sqrt{\Delta} - S)/2, z^* = r/c)$ est donc un équilibre stable.
  \end{description}
  $$
  \includegraphics[width=.5\textwidth]{DynPopMaleFemelleCouple.png}
  $$
  }
\end{enumerate}

