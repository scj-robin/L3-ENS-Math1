%-------------------------------------------------------------------------------
\subsubsection{Matrice paramétrée}
%-------------------------------------------------------------------------------

Soit la matrice
\begin{align*}
  A & = \left[\begin{array}{rrr}
    1 & 4 & 2 \\ 0 & \alpha & 0 \\ 1 & 1 & 0
    \end{array}\right].
\end{align*}
\begin{enumerate}
  \item Déterminer son polynôme caractéristique.
  \solution{En développant par rapport à la 2ème ligne, on obtient
    $$
    P_A(\lambda) 
    = (\alpha - \lambda) 
      \left| \begin{array}{cc} 1 - \lambda & 2 \\1 & - \lambda \end{array} \right|
    = (\alpha - \lambda) (\lambda^2 - \lambda - 2)
    = (\alpha - \lambda) (\lambda - 2 ) ( \lambda + 1).
    $$
  }
  \item En déduire ses valeurs propres et sa valeur propre dominante en fonction de $\alpha$.
  \solution{Les valeurs propres de $A$ sont donc $\{\alpha, 2, -1\}$. La valeur propre dominante est donc $\max(|\alpha|, 2)$ si $|\alpha| \neq 2$. Si $|\alpha| = 2$, la plus grande valeur propre en module n'est pas unique : elle n'est donc pas dominante.}
  \item A quelle condition sur $\alpha$ $A$ est-elle diagonalisable ?
  \solution{
    \begin{description}
    \item[$\alpha \notin \{-1, 2\}$:] les 3 valeurs propres sont réelles et distinctes donc $A$ est diagonalisable.
    \item[$\alpha = 2$:] $2$ est alors raçine double de $P_A(\lambda)$ et ses vecteurs propres associés sont solutions de 
    $$
    \left\{ \begin{array}{rcl}  
            u_1 + 4 u_2 + 2 u_3 & = & 2 u_1 \\
            2 u_2 & = & 2 u_2 \\
            u_1 + u_2 & = & 2 u_3 
            \end{array} \right.
    \quad \Leftrightarrow \quad
    \left\{ \begin{array}{rcl}  
            2 u_1 + 5 u_2 & = & 2 u_1 \\
            u_1 + u_2 & = & 2 u_3 
            \end{array} \right.         
    \quad \Leftrightarrow \quad
    \left\{ \begin{array}{rcl}  
            u_2 & = & 0 \\
            u_1 & = & 2 u_3 
            \end{array} \right.         
    $$
    qui définit un sous-espace propre de dimension 1 : $A$ n'est donc pas diagonalisable.
    \item[$\alpha = -1$:] $-1$ est alors raçine double de $P_A(\lambda)$ et ses vecteurs propres associés sont solutions de 
    $$
    \left\{ \begin{array}{rcl}  
            u_1 + 4 u_2 + 2 u_3 & = & - u_1 \\
            2 u_2 & = & - u_2 \\
            u_1 + u_2 & = & - u_3 
            \end{array} \right.
    \quad \Leftrightarrow \quad
    \left\{ \begin{array}{rcl}  
            u_1 + 2 u_3 & = & - u_1 \\
            u_2 & = & 0 \\
            u_1& = & - u_3 
            \end{array} \right.
    \quad \Leftrightarrow \quad
    \left\{ \begin{array}{rcl}  

    u_2 & = & 0 \\
            u_1 & = & - u_3 
            \end{array} \right.
    $$
    qui définit un sous-espace propre de dimension 1 : $A$ n'est donc pas diagonalisable. 
  \end{description}
  Alternativement, on peut 
  \begin{enumerate}
    \item former les matrices $A - I$ et $A + I$ et déterminer leurs rangs en exhibant des vecteurs colonnes indépendants, 
    \item en déduire la dimension de l'image $\dim(Im(A - \lambda I)$, qui vaut 2 pour $\lambda = 2$ et $\lambda = -1$, 
    \item puis en déduire la dimension du sous-espace propre associé $\dim(\Ecal(\lambda)) = \dim(Ker(A - \lambda I)) = 3 - \dim(Im(A - \lambda I)$, qui vaut 1 dans les deux cas.
  \end{enumerate}
  }
\end{enumerate}

