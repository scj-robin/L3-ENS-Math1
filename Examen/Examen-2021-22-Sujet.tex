%-------------------------------------------------------------------------------
%-------------------------------------------------------------------------------
% \section*{Sujet}
%-------------------------------------------------------------------------------

% %-------------------------------------------------------------------------------
% \subsection{Algèbre linéaire}
% %-------------------------------------------------------------------------------

\begin{exercise}[Algèbre linéaire] ~ \label{exam:AlgLin}
  \begin{enumerate}
% [Ex 1.3.3]
  \item On considère la matrice 
  $$
  A = \left[\begin{array}{cc} a & b \\ c & d\end{array}\right]
  $$
  Déterminer son polynôme caractéristique et donner les conditions sur $p = \tr(A)$ et $q = \det(A)$ pour que $A$ admette deux valeurs propres réelles.
  %
  \item Montrer le polynôme caractéristique de la matrice 
  \begin{align*}
    B & = \left[\begin{array}{rrr}
      -1 & 3 & 1 \\ 0 & 2 & 1 \\ \alpha & \beta & 2
      \end{array}\right].  
  \end{align*}
  est
  $$
  P_B(\lambda) = -(1+\lambda)(\lambda^2 - 4 \lambda + 4 - \alpha - \beta)
  $$
  \item En déduire une condition sur $\alpha$ et $\beta$ pour que $B$ soit diagonalisable.
  \end{enumerate}
\end{exercise}

% %-------------------------------------------------------------------------------
% \subsection{Fonction de plusieurs variables}
% %-------------------------------------------------------------------------------

% %-------------------------------------------------------------------------------
% \subsection{\'Equations différentielles}
% %-------------------------------------------------------------------------------

\begin{exercise}[Dynamique de population] ~ \label{exam:DynPop}
  On considère une population sexuée panmictique, au sein de laquelle on désigne
  respectivement par $x(t)$, $y(t)$ et $z(t)$ les densités au temps $t$ de femelles flottantes, de mâles flottants, et de couples. On suppose que la dynamique de la population respecte le système dynamique suivant
  \begin{equation} \label{eq:DynPop}
    \left\{\begin{array}{rcl}
            \dot x(t) & = & - \alpha x y + r z, \\
            \dot y(t) & = & - \alpha x y + r z, \\
            \dot z(t) & = & + \alpha x y - c z^2,
           \end{array} \right.
  \end{equation}
  où les coefficients $\alpha$, $r$ et $c$ sont strictement positifs.
  \begin{enumerate}
   \item Interpréter ces équations et la signification de chacun des coefficients $\alpha$, $r$ et $c$.
   \item En notant $S = x(0) - y(0)$, montrer que $x(t) - y(t) = S$ pour tout temps $t$. En déduire les fonctions $y$ et $z$ satisfont le système 
  \begin{equation} \label{eq:DynPop2}
    \left\{\begin{array}{rcl}
            \dot y(t) & = & - \alpha (y^2 + Sy) + r z, \\
            \dot z(t) & = & + \alpha (y^2 + Sy) - c z^2.
           \end{array} \right.
  \end{equation}
  Dans la suite on supposera que $S > 0$.
  \item Déterminer les points d'équilibre du système \eqref{eq:DynPop2}.
  \item \'Ecrire la matrice jacobienne du système \eqref{eq:DynPop2} et étudier la nature du ou des équilibres non triviaux.
  \end{enumerate}
\end{exercise}

% %-------------------------------------------------------------------------------
% \subsection{Probabilités}
% %-------------------------------------------------------------------------------

\begin{exercise}[Probabilités] ~ \label{exam:Proba}
  Les organismes d’une espèce asexuée se reproduisent suivant un processus de Bienaymé–Galton–Watson, où chaque individu laisse à la génération suivante un nombre aléatoire $X$ d’individus. \\
  On note $p_k = \Pr\{X = k\}$ et on suppose que 
  $$
  p_0 = b
  \qquad \text{et} \qquad 
  p_k = (1 - b) (1 - a) a^{k-1} \text{ pour } k \geq 1.
  $$
  \begin{enumerate}
    \item Montrer que la fonction génératrice de $X$ vaut
    $$
    f(s) := \Esp(s^X) = b + (1-b)(1-a) \frac{s}{1 - as}.
    $$
    \item Déterminer la valeur de la probabilité, notée $q$, qu'une population de cette espèce issue d’un seul individu fondateur s'éteigne.
  \end{enumerate}
  On introduit un nombre aléatoire $N$ d'individus de cette espèce dans une île. On note $g$ la fonction génératrice de $N$. On suppose que les descendances des $N$ individus fondateurs évoluent indépendamment les unes des autres. On s'intéresse à la probabilité $p$ de colonisation, c'est-à-dire à la probabilité pour que cette population ne s'éteigne pas.
  \begin{enumerate}
    \setcounter{enumi}{2}
    \item A quelle condition $p$ est-elle non nulle ?
    \item Exprimer $p$ en fonction de $q$. 
    \item Déterminer $p$ si $N = 1$ et si $N$ est distribué comme $X$. Commenter.
  \end{enumerate}
\end{exercise}
