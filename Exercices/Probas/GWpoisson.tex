%-------------------------------------------------------------------------------
\subsubsection{Processus de Galton–Watson poissonnien} 
%-------------------------------------------------------------------------------
  On considère un processus de Galton-Watson partant d'une population de taille 1 et dans lequel le nombre de descendants $X$ par individu suit une loi de Poisson $\Pcal(\lambda)$ :
  $$
  \Pr\{X = k\} = e^{-\lambda} \frac{\lambda^k}{k!}.
  $$
  \begin{enumerate}
    \item Donner le nombre moyen $m$ de descendants par individu.
    \solution{$m = \Esp(X) = \lambda$.}
    \item A quelle condition sur $\lambda$ la probabilité d'extinction $q$ est-elle strictement inférieure à 1 ?
    \solution{Si $m = \lambda > 1$.}
    \item Déterminer la fonction génératrice $\phi$ du nombre de descendants par individu.
    \solution{Si $X \sim \Pcal(\lambda)$, 
    $$
    \phi(s) 
    = \Esp(s^X) 
    = e^{-\lambda} \sum_{k \geq 0} \frac{(\lambda s)^k}{k!} 
    = e^{\lambda(s-1)}.
    $$}
    \item Pour quelles valeurs de $\lambda$ la probabilité d'extinction $q$ vaut-elle $1/2$,  $1/10$ et $1/100$ ?
    \solution{On sait que $q$ vérifie $q = \phi(q)$, soit, 
    $q = \exp{\lambda(q - 1)}$, c'est-à-dire $\lambda = \log q / (q - 1)$. 
    On obtient ainsi $\lambda \simeq 1.4$ pour $q = 1/2$, $\lambda \simeq 2.6$ pour $q = 1/10$ et $\lambda\simeq4.7$ pour $q = 1/100$. \\
    Avec presque 5 descendants (en moyenne) par individus, la population a encore 1\% de chance de s'éteindre. ($q \simeq .007$ pour $\lambda = 5$).
    }
  \end{enumerate}
