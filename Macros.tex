% Environnement
\newtheorem{theorem}{Théorème}
\newtheorem{definition}{Définition}
\newtheorem{lemma}{Lemme}
\newtheorem{proposition}{Proposition}
\newtheorem{corollary}{Corollaire}
\newtheorem*{corollary*}{Corollaire}
\newtheorem*{theorem*}{Théorème}
\newtheorem*{definition*}{Définition}
\newtheorem*{proposition*}{Proposition}
\newtheorem*{assumption*}{Hypothèse}
\newtheorem*{algorithm*}{Algorithme}
\newtheorem*{lemma*}{Lemme}
\newtheorem*{remark*}{Remarque}
\newtheorem*{exercise*}{Exercice}
\newtheorem{exercise}{Exercice}
\newcommand{\parbfsl}[1]{
  \medskip
  \noindent {\bf \textsl{#1}} \xspace}
\newcommand{\parsl}[1]{
  \medskip
  \noindent {\sl #1} \xspace}
\newcommand{\remark}{\parbfsl{Remarque.}}
\newcommand{\remarks}{\parbfsl{Remarques.}}
\renewcommand{\proof}{\parbfsl{Démonstration.}}
\newcommand{\eproof}{
$\blacksquare$
\bigskip}

% Effets, couleurs
\definecolor{darkgreen}{cmyk}{0.5, 0, 0.5, 0.4}
\newcommand{\emphase}[1]{\textcolor{red}{#1}}
\newcommand{\demoProp}[1]{\noindent{\textbf{\textsl{Démonstration de la proposition \ref{#1} :}}}}
\newcommand{\itemdot}{\textbullet}

% Moments
\DeclareMathOperator{\Esp}{\mathbb{E}}
\DeclareMathOperator{\diag}{diag}
\DeclareMathOperator{\Cov}{\mathbb{C}ov}
\DeclareMathOperator{\tr}{tr}
\DeclareMathOperator{\Var}{\mathbb{V}}
\let\Pr\relax\DeclareMathOperator{\Pr}{\mathbb{P}}
\renewcommand{\d}{\text{d}}

% R, N, ...
\newcommand{\cst}{\text{cst}}
\newcommand{\Cbb}{\mathbb{C}}
\newcommand{\Ibb}{\mathbb{I}}
\newcommand{\Nbb}{\mathbb{N}}
\newcommand{\Rbb}{\mathbb{R}}
\newcommand{\Zbb}{\mathbb{Z}}

% Indicateurs

% Lois et ensembles
\newcommand{\Acal}{\mathcal{A}}
\newcommand{\Bcal}{\mathcal{B}}
\newcommand{\Ccal}{\mathcal{C}}
\newcommand{\Ecal}{\mathcal{E}}
\newcommand{\Gcal}{\mathcal{G}}
\newcommand{\Ical}{\mathcal{I}}
\newcommand{\Lcal}{\mathcal{L}}
\newcommand{\Mcal}{\mathcal{M}}
\newcommand{\Ncal}{\mathcal{N}}
\newcommand{\Pcal}{\mathcal{P}}
\newcommand{\Rcal}{\mathcal{R}}
\newcommand{\Scal}{\mathcal{S}}
\newcommand{\Ucal}{\mathcal{U}}
\newcommand{\Xcal}{\mathcal{X}}
\newcommand{\Ycal}{\mathcal{Y}}

% Comments
\newcommand{\SR}[2]{\textcolor{gray}{#1}\textcolor{red}{#2}}
\newcommand{\todo}[1]{\textcolor{red}{\`A faire~: {\sl #1}}}
\newcommand{\dessin}[1]{
\bigskip
\begin{breakbox} \noindent \textcolor{purple}{{\sl #1}} \end{breakbox}
}
\newcommand{\progres}[1]{
\newpage
\begin{breakbox} \noindent \textcolor{blue}{{\sl #1}} \end{breakbox}
}
\newcommand{\solution}[1]{
\bigskip
\begin{breakbox} \parsl{Solution.}
  \noindent #1
\end{breakbox}
\medskip
}
\newcommand{\exemple}[1]{
\bigskip
\begin{breakbox} \parbfsl{Exemple.}
  \noindent #1
\end{breakbox}
\medskip
}
\newcommand{\warning}[1]{(\textcolor{red}{\faWarning\ #1})}
\newcommand{\SRcorrect}[2]{\textcolor{gray}{#1}\textcolor{blue}{#2}}
\newcommand{\SRcomment}[1]{\textcolor{blue}{[{\sl SR: #1}]}}

% TikZ
\newcommand{\nodesize}{2em}
\newcommand{\edgeunit}{2.5*\nodesize}
\tikzstyle{square}=[rectangle, draw]
\tikzstyle{param}=[draw, rectangle, fill=gray!50, minimum width=\nodesize, minimum height=\nodesize, inner sep=0]
\tikzstyle{hidden}=[draw, circle, fill=gray!50, minimum width=\nodesize, inner sep=0]
\tikzstyle{hiddenred}=[draw, circle, color=red, fill=gray!50, minimum width=\nodesize, inner sep=0]
\tikzstyle{observed}=[draw, circle, minimum width=\nodesize, inner sep=0]
\tikzstyle{observedred}=[draw, circle, minimum width=\nodesize, color=red, inner sep=0]
\tikzstyle{eliminated}=[draw, circle, minimum width=\nodesize, color=gray!50, inner sep=0]
\tikzstyle{empty}=[draw, circle, minimum width=\nodesize, color=white, inner sep=0]
\tikzstyle{blank}=[color=white]
\tikzstyle{edge}=[-, line width=1pt]
\tikzstyle{edgebendleft}=[-, >=latex, line width=1pt, bend left]
\tikzstyle{lightedge}=[-, line width=1pt, color=gray!50]
\tikzstyle{arrow}=[->, >=latex, line width=1pt]
\tikzstyle{arrowbendleft}=[->, >=latex, line width=1pt, bend left]
\tikzstyle{arrowred}=[->, >=latex, line width=1pt, color=red]
\tikzstyle{arrowbendleftred}=[->, >=latex, line width=1pt, bend left, color=red]
\tikzstyle{arrowblue}=[->, >=latex, line width=1pt, color=blue]
\tikzstyle{dashedarrow}=[->, >=latex, dashed, line width=1pt]
\tikzstyle{dashededge}=[-, >=latex, dashed, line width=1pt]
\tikzstyle{dashededgebendleft}=[-, >=latex, dashed, line width=\edgewidth, bend left]
\tikzstyle{lightarrow}=[->, >=latex, line width=1pt, color=gray!50]
