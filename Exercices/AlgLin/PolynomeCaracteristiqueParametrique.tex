%-------------------------------------------------------------------------------
\subsubsection{Polynôme caractéristique d'une matrice paramétrée}
%-------------------------------------------------------------------------------

On considère la matrice
\begin{align*}
  B & = \left[\begin{array}{rrr}
    -1 & 3 & 1 \\ 0 & 2 & 1 \\ a & b & 2
    \end{array}\right].  
\end{align*}

\begin{enumerate}
  \item Montrer son polynôme caractéristique est
  $$
  P_B(\lambda) = -(1+\lambda)(\lambda^2 - 4 \lambda + 4 - a - b).
  $$
  \solution{
  On développe par rapport à la première colonne et on développe : 
  \begin{align*}
    P_B(\lambda) 
    & = |B - \lambda I_3|
    = (-1 -\lambda) \left|\begin{array}{cc} 2 - \lambda & 1 \\ b & 2 - \lambda \end{array}\right| 
    - a \left|\begin{array}{cc} 3 & 1 \\ 2 - \lambda & 1 \end{array}\right| \\
    & = (-1 - \lambda) ((2 - \lambda)^2 - b) - a (3 - (2 - \lambda)) \\
    & = - (1 + \lambda) (\lambda^2 - 4 \lambda + 4 - b) - a (1 + \lambda) \\
    & = - (1 + \lambda) (\lambda^2 - 4 \lambda + 4 - (b + a)).
  \end{align*}
  }
  \item A quelle(s) condition(s) sur $a$ et $b$, $B$ possède-t-elle trois valeurs propres réelles distinctes ?
  \solution{La première valeur propre est donc $\lambda_1 = -1$ et le discriminant du polynôme 
  $$
  Q(\lambda) = \lambda^2 - 4 \lambda + 4 - (a+b)
  $$
  vaut $\Delta = 16 - 4(4 - (a+b)) = 4(a + b)$. Les deux valeurs propres restantes $\lambda_2$ et $\lambda_3$ sont donc réelles et distinctes ssi $a + b > 0$ et valent alors
  $$
  \lambda_2 = 2 + \sqrt{a + b} > 0 
  \qquad \text{et} \qquad
  \lambda_3 = 2 - \sqrt{a + b}.
  $$
  $\lambda_3$ est de plus distincte de $\lambda_1 = -1$ ssi $a + b \neq 9$. \\
  Dans ce cas ($a + b > 0$ et $a + b \neq 9$), les 3 valeurs propres sont réelles et distinctes (et $B$ est donc diagonalisable). 
  }
  %
  \item A quelle(s) condition(s) sur $a$ et $b$, $B$ est-elle diagonalisable ?
  \solution{
  \begin{description}
    \item[$a+b < 0$:] $B$ n'admet qu'une seule valeur propre réelle $(\lambda = -1$) et son ordre de multiplicité est 1, donc $B$ n'est pas diagonalisable.
    \item[$a + b = 0$:] On a alors $\lambda_2 = \lambda_3 = 2$, et il faut résoudre le système
    $$
    \left\{\begin{array}{rcl}
            -x + 3y + z & = & 2x \\
            2y + z & = & 2y \\
            ax -ay + 2z & = & 2z
           \end{array}\right.
    \qquad \Leftrightarrow \qquad
    \left\{\begin{array}{rcl}
            x & = & y \\
            z & = & 0
           \end{array}\right. .
    $$
    L'espace propre associé à $\lambda = 2$ est une droite, donc $B$ n'est pas diagonalisable.
    \item[$a+b > 0$ et $a + b = 9$:] On a alors $\lambda_1 = \lambda_3 = -1$, et il faut résoudre le système
    $$
    \left\{\begin{array}{rcl}
            -x + 3y + z & = & -x \\
            2y + z & = & -y \\
            ax +(9-a)y + 2z & = & 2z
           \end{array}\right.
    \qquad \Leftrightarrow \qquad
    \left\{\begin{array}{rcl}
            3y & = & -z \\
            ax & = & ay
           \end{array}\right. .
    $$
    \begin{itemize}
      \item Si $a \neq 0$, l'espace propre associé à $\lambda = -1$ est la droite d'équation $\{3y = z, ax = ay\}$ et $B$ n'est pas diagonalisable.
      \item Si $a = 0$, l'espace propre associé à $\lambda = -1$ est le plan d'équation $\{3y = z\}$ et $B$ est diagonalisable.
    \end{itemize}
    \item[$a+b > 0$ et $a+b \neq 9$:] les trois valeurs sont réelles et distinctes, donc $B$ est diagonalisable.
  \end{description}
  $B$ est donc diagonalisable si $a+b > 0$ et $a+b \neq 9$, ou si $a = 0$ et $b = 9$.
  }
\end{enumerate}
