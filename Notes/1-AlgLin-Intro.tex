\bigskip
\progres{Début Cours 1. 
\begin{itemize}
 \item Plan du cours
\end{itemize}
}

%-------------------------------------------------------------------------------
%-------------------------------------------------------------------------------
\section{Introduction} \label{sec:LinAlg-Intro}
%-------------------------------------------------------------------------------
%-------------------------------------------------------------------------------

\paragraph{Notations maticielles :} ~
\begin{itemize}
  \item $x, y$  réels ou vecteurs de $\Rbb^n$
  \item $X, A$  matrices ou variables aléatoires
  \item $\Mcal_{n, p}(\Rbb) = \Mcal_{n, p}$, 
  \item $\Mcal_n(\Rbb) = \Mcal_n$ (matrices carrées), 
  \item $A = [a_{ij}]$, élément générique
  \item $\diag(\lambda_1, \dots \lambda_n)$, 
  \item $I_n$.
\end{itemize}

%-------------------------------------------------------------------------------
%-------------------------------------------------------------------------------
\subsection{Matrice de covariance} \label{sec:MatCov}
%-------------------------------------------------------------------------------

\begin{definition*}[Matrice de covariance]
  $A$ = matrice de covariance des variables aléatoires $X_1, \dots X_n$ :
  $$
  a_{ij} = \Cov(X_i, X_j) = \Esp[(X_i - \Esp(X_i)) (X_j - \Esp(X_j))].
  $$
\end{definition*}

\begin{proposition*}[Propriétés d'une matrice de covariance] ~
  \begin{itemize}
   \item $\Cov(X_i, X_j) = \Esp(X_i X_j) - \Esp(X_i) \Esp(X_j)$
   \item $a_{ij} = a_{ji}$ ($A$ symétrique) et $a_{ii} \geq 0$
  \end{itemize}
\end{proposition*}

\proof Directe. \eproof

\remark section \ref{sec:LinAlg-Pos} : définie positivité.

%-------------------------------------------------------------------------------
%-------------------------------------------------------------------------------
\subsection{Matrice de Leslie}  \label{sec:MatLeslie}
%-------------------------------------------------------------------------------

$x(t) = (x_1(t), \dots, x_i(t), \dots, x_k(t))$ répartition d'une population au temps $t$ en catégories $1, \dots, i, \dots, k$ :
$$
x_i(t+1) = \sum_{i=1}^k a_{ij} x_j(t) 
$$
$A \in \Mcal_k$, $a_{ij} \geq 0 : $ fraction de la catégorie $j$ de la population passant dans la catégorie $i$ au temps suivant. \textcolor{red}{Les colonnes contribuent aux lignes : orientation différentes de chaînes de Markov}

C'est-à-dire
$$
x(t+1) = A x(t) 
\qquad \Rightarrow \qquad 
x(t) = A^t x(0).
$$

\remark section \ref{sec:LinAlg-Pos} : comportement contrôlé par sa valeur propre dominante.

%-------------------------------------------------------------------------------
\paragraph{Cas particulier = 'Matrice de Leslie'.}

Catégories $(1, \dots k)$ = classes d'âges : $a_{ij} = 0$ sauf 
\begin{align*}
  a_{i, i-1} : & \text{ taux de survie de la classe d'âge $i$}, \\
  a_{1, i} : & \text{ taux de fécondité de la classe d'âge $i$}, 
\end{align*}
soit (\textcolor{red}{les colonnes contribuent aux lignes : orientation inverses des chaînes de Markov})
$$
A = \left[\begin{array}{cccccc}
            a_{11} & a_{12} & \dots  & \dots & a_{1k} \\
            a_{21} & 0 & 0  & \dots & 0 \\
            0 & a_{32} & \ddots & & \vdots \\
            \vdots & \vdots & \ddots & \ddots & \vdots \\
            0 & \dots & 0 & a_{k, k-1} & 0 \\
          \end{array}\right].
$$

%-------------------------------------------------------------------------------
%-------------------------------------------------------------------------------
\subsection{Matrice jacobienne}  \label{sec:MatJacob}
%-------------------------------------------------------------------------------

\begin{definition*}[Matrice jacobienne]
  Soit la fonction
  $$
  \begin{array}{r|rcl}
    f : & \Rbb^n & \mapsto & \Rbb^k \\
    & x = (x_1, \dots x_n) & \rightarrow & f(x)
  \end{array}
  $$
  où
  $$
  f(x) = \left[\begin{array}{c}
                f_1(x_1, \dots, x_n) \\
                \vdots \\
                f_k(x_1, \dots, x_n) 
               \end{array} \right].
  $$
  avec $f_i: \Rbb^n \mapsto \Rbb$ pour $i = 1 \dots n$.
  La matrice jacobienne de $f$ au point $x$, noté $J_x(f)$ est la matrice de $\Mcal_{k, n}$ de terme générique
  $$
  \frac{\partial f_i}{\partial x_j}.
  $$
\end{definition*}

\begin{exercise*}
  Calculer la matrice jacobienne $J_x(f)$ pour
  $$
  \left\{\begin{array}{rl}
          f_1(x_1, x_2) & = a x_1 - c x_1 x_2, \\
          f_2(x_1, x_2) & = -b x_2 + d x_1 x_2.
         \end{array} \right.
  $$
pour un $x$ quelconque, puis pour $x = (0, 0)$ et pour $x=(1, 1)$.
\end{exercise*}

\remark section \ref{sec:Multivar-Diff} pour l'étude de $J_x(f)$ et section \ref{sec:EquaDiff-nonLineaire2D} pour l'étude de systèmes dynamique de la forme
$$
\dot{x} = f(x).
$$

%-------------------------------------------------------------------------------
%-------------------------------------------------------------------------------
\subsection{Matrice stochastique}  \label{sec:MatStoch}
%-------------------------------------------------------------------------------

\begin{definition*}[Matrice stochastique]
  $A \in \Mcal_{n, p}$ est stochastique si $a_{ij} \geq 0$ et, pour tout $1 \leq i \leq n$:
  $$
  \sum_{j=1}^p a_{ij} = 1.
  $$
\end{definition*}

\begin{proposition*}[Produit de deux matrices stochastiques]
  Si $A \in \Mcal_{\ell, m}$ et $B \in \Mcal_{m, n}$ sont stochastiques, alors $C = A B$ est stochastique.
\end{proposition*}

\proof Directe. \eproof

\paragraph{Cas particulier.} Si $A \in \Mcal_n$, $A$ est dite \emph{matrice de transition} et $A^n$ est également stochastique.

\remark section \ref{sec:LinAlg-Pos} pour l'étude des valeurs propres de $A$ et section \ref{sec:Proba-Markov} pour son utilisation dans l'étude des chaînes de Markov.
