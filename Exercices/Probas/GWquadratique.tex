%-------------------------------------------------------------------------------
\subsubsection{Processus de Galton–Watson quadratique \todo{}} 
%-------------------------------------------------------------------------------
  On considère un processus de Bienaymé-Galton-Watson partant d'une population de taille 1 et dans laquelle le nombre de descendants par individu suit la loi suivante :
  $$=
  \Pr\{X = k\} = \left\{
    \begin{array}{ll}
      1 - a - b & \text{si $k = 0$} \\
      a & \text{si $k = 1$} \\
      b & \text{si $k = 2$}
    \end{array}\right..
  $$
  \begin{enumerate}
    \item Montrer que la fonction génératrice de $X$ est
    $$
    f(s) = 1 - a - b + as + bs^2.
    $$
    \solution{\todo{}}
    \item En déduire le nombre de descendants moyen par individu $m$
    $$
    m = a + 2b.
    $$
    \solution{On a $f'_X(s) = a + 2bs$, soit $m = f'_X(1) = a + 2b$.}
    \item A quelle condition, la probabilité d'extinction de la population est elle
    $$
    q = b/(1 - a - b) ?
    $$
    \solution{La probabilité d'extinction vérifie $q = f(q)$, soit
    $(1 - a -b) + (a-1)q + bq^2 = 0$
    dont le discriminant est
    $$
    \Delta = (a-1)^2 - 4 b (1 - a -b) = ... = (a + 2b - 1)^2 = (m-1)^2
    $$
    d'où
    $$
    q 
    = \frac{(1-a) \pm (m-1)}{2(1 - a - b)}
    = \left\{\begin{array}{ll}
      1 & (+) \\
      b/(1- a - b) & (-).
    \end{array}\right..
    $$
    $b/(1- a - b)$ est la plus petite solution (donc la probabilité d'extinction) ssi 
    $b/(1- a - b) < 1$, soit $a + 2b > 1$, c'est-à-dire $m > 1$.
    \\
    \todo{à verifier}
    }
  \end{enumerate}
