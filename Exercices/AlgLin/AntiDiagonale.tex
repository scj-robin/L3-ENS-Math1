%-------------------------------------------------------------------------------
\subsubsection{Inverse de matrice $3 \times 3$}
%-------------------------------------------------------------------------------

% Source http://exo7.emath.fr/ficpdf/fic00054.pdf (Del05-TD-ULCB-fic00054.pdf)

Soit la matrice
$$
A = \left[\begin{array}{rrr}
      0 & 1 & 1 \\ 1 & 0 & 1 \\ 1 & 1 & 0
    \end{array}\right].  
$$

\begin{enumerate}
  \item Montrer que $A^2 = A + 2 I$.
  \solution{
    On calcule directement
    $$
    A^2 = \left[\begin{array}{rrr}
            1 & 1 & 2 \\ 1 & 2 & 1 \\ 1 & 1 & 2
          \end{array}\right]
        = A + 2 I.
    $$
  }
  %
  \item En déduire que $A$ est inversible et calculer son inverse.
  \solution{
    On a 
    $$
    A^2 - A = 2 I
    \qquad \Leftrightarrow \qquad
    A \left[\frac12 (A - I)\right] = I,
    $$
    donc $A$ est inversible et son inverse est $(A - I)/2$.
  }
  \item Déterminer le polynôme caractéristique de $A$ et en déduire ses valeurs propres.
  \solution{
    \begin{align*}
      P_A(\lambda) 
      & = \left|\begin{array}{rrr}
                  -\lambda & 1 & 1 \\ 1 & -\lambda & 1 \\ 1 & 1 & -\lambda
                \end{array}\right|
       = - \lambda \left|\begin{array}{rr}
                            -\lambda & 1 \\ 1 & -\lambda
                            \end{array}\right|
         - 1 \left|\begin{array}{rr}
                            1 & 1 \\ 1 & -\lambda
                            \end{array}\right|
         + 1 \left|\begin{array}{rr}
                            1 & -\lambda \\ 1 & 1 
                            \end{array}\right|      \\                      
       & = - \lambda \left|\begin{array}{rr}
                            -\lambda & 1 \\ 1 & -\lambda
                            \end{array}\right|
         - 2 \left|\begin{array}{rr}
                            1 & 1 \\ 1 & -\lambda
                            \end{array}\right|
        \quad (\text{car le déterminant est une forme alternée}) \\
        & = - \lambda (\lambda^2 - 1) + 2(\lambda +1)
        = -\lambda (\lambda - 1) (\lambda + 1)  + 2 (\lambda + 1)
        = (\lambda + 1) (-\lambda^2 + \lambda + 2) \\
        & = (\lambda + 1) (\lambda + 1) (2 - \lambda).
    \end{align*}
    les valeurs propres sont donc $\lambda_1 = 2$ et $\lambda_2 = -1$, qui est double.
  }
  \item Quelle transformation cette matrice opère-t-elle ?
  \solution{
    Le vecteur propre $v_1 = [x \; y \; z]$ associé à $\lambda_1 = 2$ vérifie
    $$
    \left\{\begin{array}{rcl}
            y + z & = & 2 x \\
            x + z & = & 2 y \\
            x + y & = & 2 z
          \end{array}\right.
    \quad \Leftrightarrow \quad x = y = z,
    $$    
    soit, par exemple $v_1 = [1 \; 1 \; 1]$. \\
    De plus, $A$ est symétrique, donc diagonalisable, donc $\Ecal(\lambda_2)$ est un plan, et ses vecteurs propres sont orthogonaux, donc $\Ecal(\lambda_2)$ est le plan des vecteurs orthogonaux à $v_1$ : 
    $$
    \Ecal(\lambda_2) = \{w \in \Rbb^3 : v_1^\top w = 0\}.
    $$
    $A$ opère donc une homothétie (de rapport $\lambda_1 = 2$) le long de la première bissectrice et une symétrie ($\lambda_1 = -1$) dans le plan orthogonal. Pour $u = a v_1 + b w$ (avec $w \in \Ecal(\lambda_2)$), on a $A u = 2 a v_1 - b w$.
  }
\end{enumerate}

