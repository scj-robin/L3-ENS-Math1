%-------------------------------------------------------------------------------
\subsubsection{Matrices diagonalisables ?}
%-------------------------------------------------------------------------------

Déterminer le polynôme caratéristique des matrices suivantes et en déduire si elles sont diagonalisables :
\begin{align*}
  A & = \left[\begin{array}{rrr}
    2 & -1 & 3 \\ 2 & -1 & 6 \\ 1 & 0  & 2
    \end{array}\right], &
  B & = \left[\begin{array}{rrr}
    1 & 1 & 0 \\ -5 & -2 & 5 \\ -1 & 0 & 2
    \end{array}\right].
\end{align*}
  
\solution{
  \begin{description}
    \item[$A$ :] on a
    \begin{align*}
        P_A(\lambda) 
        = \left| \begin{array}{rrr}
          2 - \lambda & -1 & 3 \\ 2 & -1 - \lambda & 6 \\ 1 & 0  & 2 - \lambda 
          \end{array}\right|
        = -\lambda^3 + 3 \lambda^2 + \lambda - 3
        = -(\lambda - 3) (\lambda - 1) (\lambda + 1).
    \end{align*}
    Les valeurs propres sont donc $3$, $1$ et $-1$ qui sont toutes réelles et distinctes, donc $A$ est diagonalisable.
    \item[$B$ :] on a
    \begin{align*}
        P_B(\lambda) 
        = \left| \begin{array}{rrr}
            1 - \lambda & 1 & 0 \\ -5 & -2 - \lambda & 5 \\ -1 & 0 & 2 - \lambda
          \end{array}\right|
        = - \lambda^3 + \lambda^2 - \lambda + 1
        = - (\lambda-1) (\lambda^2 + 1).
    \end{align*}
    Les valeurs propres sont donc $1$, $i$ et $-i$ qui ne sont pas toutes réelles, donc $B$ n'est pas diagonalisable.
  \end{description}
}


