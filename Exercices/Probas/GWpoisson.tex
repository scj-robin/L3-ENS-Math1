%-------------------------------------------------------------------------------
\subsubsection{Processus de Galton–Watson poissonnien} %-------------------------------------------------------------------------------
  On considère un processus de Bienaymé-Galton-Watson partant d'une population de taille 1 et dans laquelle le nombre de descendants par individu suit une loi géométrique $\Pcal(\lambda)$ :
  $$
  \Pr\{X = k\} = e^{-\lambda} \frac{\lambda^k}{k!}.
  $$
  \begin{enumerate}
    \item Rappeler le nombre moyen $m$ de descendants par individu.
    \solution{$m = \lambda$.}
    \item Déterminer la fonction génératrice $\phi$ du nombre de descendants par individu.
    \solution{Si $X \sim \Pcal(\lambda)$, 
    $$
    \phi(s) 
    = \Esp(s^X) 
    = e^{-\lambda} \sum_{k \geq 0} \frac{(\lambda s)^k}{k!} 
    = e^{\lambda(s-1)}.
    $$}
    \item A quelle condition sur $\lambda$ la probabilité d'extinction $q$ est-elle strictement inférieure à 1 ?
    \solution{Si $m = \lambda > 1$.}
    \item Pour quelle valeur de $\lambda$ a-t-on $q = 1/2$ ?
    \solution{On sait que $q$ vérifie $q = \phi(q)$, soit, 
    $$
    \lambda = \frac{\log q}{q - 1},
    $$
    soit, pour $q = 1/2$, $\lambda = 2 \log 2 \simeq 1.386$.}
  \end{enumerate}
