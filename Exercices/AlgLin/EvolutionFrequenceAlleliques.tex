%-------------------------------------------------------------------------------
\subsubsection{\'Evolution de fréquences alléliques}
%-------------------------------------------------------------------------------

\todo{Bof : voir {\tt DynPop-MailAL-30oct2022} pour une preuve alternative de la fin.}

Dans une population diploïde panmictique, on s’intéresse à un gène existant sous la forme de
$m$ allèles. On désigne par $p_i$ la proportion du pool gamétique portant l’allèle $i$ ($\sum_i p_i = 1$) et par $p'_i$ cette proportion à la génération qui suit. La dynamique des fréquences alléliques est donnée par
\begin{equation} \label{eq:dynFreqModele}
  V(p) \; p'_i = p_i \sum_{j=1}^m a_{ij} p_j 
  \qquad \text{où} \qquad
  V(p) = \sum_i \sum_j a_{ij} p_i p_j
\end{equation}
(de sorte que $\sum_i p'_i = 1$).
On définit de plus la matrice, supposée symétrique, $A = [a_{ij}]$ où les $a_{ij}$ sont positifs ou nuls, mais non tous nuls.

\begin{enumerate}
  \item Interpréter l'équation \eqref{eq:dynFreqModele}. Pourquoi $A$ est-elle supposée symétrique ?
  \solution{$a_{ij}$ = avantage relatif du génotype $(A_i, A_j)$, symétrique par nature. \\
  Deux remarques : 
  \begin{itemize}
  \item $V(p) = p^\top A p$ est l'avantage moyen d'un descendant, du fait de la reproduction panmictique ; 
  \item ce modèle est paramétré à une constante près, c'est-à-dire qu'on aboutit à la même dynamique en remplaçant $A$ par $B = k A$ pour n'importe quel $k > 0$.
  \end{itemize}
  }
\end{enumerate}

\bigskip
\noindent On suppose maintenant qu'il existe un équilibre $p^* = [p^*_1, \dots p^*_m]^\top$ non trivial ($\forall i, p^*_i \neq 0$). On définit la viabilité à l'équilibre par $V^* = V(p^*) > 0$.

\begin{enumerate}
  \setcounter{enumi}{1}
  \item Montrer que, pour tout $i$, $V^* = \sum_j a_{ij} p^*_j$.
  \solution{L'équilibre $p^*$ est un point stationnaire de \eqref{eq:dynFreqModele}, donc
  $$
  \forall i: \quad V^* p^*_i = p^*_i \sum_{j=1}^m a_{ij} p^*_j,
  $$
  soit $V^* = \sum_{j=1}^m a_{ij} p^*_j, \forall i$. }
  %
  \item En écrivant tout vecteur de fréquences alléliques $p$ sous la forme $p = p^* + x$, montrer que $V^*$ est maximale ssi
  \begin{equation} \label{eq:dynFreqCondition}
    \forall x: \sum_i x_i = 0, \qquad x^\top A x \leq 0.
  \end{equation}
  \solution{  Si $p = p^* + x$, on a 
  $$
  V(p) 
  = p^\top A p
  = {p^*}^\top A p^* + 2 x^\top A p^* + x^\top A x
  = V^* + 2 x^\top A p^* + x^\top A x.
  $$
  De plus, $p$ et $p^*$ étant des vecteurs de fréquences ($1_m^\top p = 1_m^\top p^* = 1$), si $p = p^* +x$, alors $1_m^\top x = \sum_i x_i = 0$. Or on a vu que $\sum_j a_{ij} p^*_j$ est indépendant de $i$ (et égal à $V^*$), donc $A p^* = V^* 1_m$, donc
  $$
  x^\top A p^* = x^\top V^* 1_m = V^* x^\top 1_m = 0
  $$
  donc
  $$
  V(p) 
  = V^* + x^\top A x.
  $$
  Enfin, $V^*$ est maximale, ssi $\forall p, V(p) \leq V^*$, c'est-à-dire ssi
  \eqref{eq:dynFreqCondition}.
  }
  %
  \item Montrer qu'on peut écrire $A$ sous la forme $A = R \Lambda R^\top$ où
  $$
  \Lambda = \text{diag}(\lambda_1, \dots, \lambda_k, 
    \lambda_{k+1}, \dots, \lambda_{k+\ell}, 
    0, \dots, 0)
  $$
  où $k \geq 0$, $\ell \geq 0$, $k+\ell \leq m$, $\lambda_1, \dots \lambda_k > 0$ et $\lambda_{k+1}, \dots \lambda_{k+\ell} < 0$.
  \solution{
  $A$ étant symétrique, elle est diagonalisable et ses vecteurs propres sont orthogonaux. On peut donc l'écrire sous la forme $A = R \Lambda R^\top$ en mettant en premier les valeurs propres strictement positives, puis strictement négatives, puis, éventuellement, nulles. 
  }
  %
  \item Montrer que $k \geq 1$.
  \solution{
  Si $k = 0$, alors tous les $\lambda_i$ sont négatifs (ou nuls), donc
  $$
  \forall x \in \Rbb^m: \qquad 
  x^\top A x 
  = x^\top R \Lambda R^\top x 
  = y^\top \Lambda y  
  = \sum_i \lambda_i y_i^2 \leq 0 
  $$
  en posant $y = R^\top x$, or $V^* = {p^*}^\top A p^* > 0$ (car $A \geq 0$ et $p^*_i > 0$, $\forall i$), donc on a nécessairement $k \geq 1$.
  }
  %
%   \item Montrer que \eqref{eq:dynFreqCondition} peut s'écrire 
%   $$
%   \forall y: \sum_i \lambda_i q_i y_i = 0, \qquad \sum_i \lambda_i y_i^2 \geq 0, 
%   $$
%   où $q$ est un vecteur à préciser.
%   \solution{
%   D'après la question précédente, toujours en posant $y = R^\top x \Leftrightarrow x = R y$, on a 
%   \begin{align*}
%     1_n^\top x & = 0 &
%     \Leftrightarrow \qquad 1_n^\top R y & = 0 & 
%     \Leftrightarrow \qquad V^* 1_n^\top R y & = 0 & 
%   \end{align*}
%   car $V^* > 0$. En utilisant la question 1, on a $V^* 1_n = A p^*$, donc
%   $$
%   1_n^\top x = 0 
%   \quad \Leftrightarrow \quad {p^*}^\top A R y = 0 
%   \quad \Leftrightarrow \quad {p^*}^\top R \Lambda y = 0 
%   \quad \Leftrightarrow \quad q^\top \Lambda y = \sum_i \lambda_i q_i y_i = 0
%   $$
%   en prenant $q = R^\top p^*$.
%   }
  %
  \item En déduire que si $V^*$ maximale, alors $k=1$ et $\ell \geq 1$.
  \solution{
%   Supposons $p \geq 2$ et prenons 
%   $$
%   y = \left[\lambda_2 q_2 \quad -\lambda_1 q_1 \quad 0 \quad \dots \quad 0\right]\top.
%   $$
%   $y$ vérifie bien
%   $$
%   \sum_i \lambda_i q_i y_i = \lambda_1 q_1 \lambda_2 q_2 - \lambda_2 q_2 \lambda_1 q_1 = 0
%   $$
%   et on a
%   $$
%   \sum_i \lambda_i y_i^2 = \lambda_1 \lambda_2 (\lambda_1 q_1^2 + \lambda_2 q_2^2).
%   $$
  On a vu que, pour que $V^*$ soit maximale, il faut que $x^\top A x \leq 0$ pour tout $x$ vérifiant $\sum_i x_i = 0$, qui définit une sous-espace de dimension $m-1$. $k$ ne peut donc pas être supérieur ou égal à 2 (car alors il existerait alors au moins deux dimensions dans lesquelles $x^\top A x > 0$). \\
  La condition $\ell \geq 1$ assure seulement qu'il existe des vecteurs de fréquences $p$ donnant une viabilité $V(p) < V^*$. 
  }
%   $$
%   \text{\textcolor{red}{Voir mail AL 30 oct 2022}}
%   $$
\end{enumerate}
