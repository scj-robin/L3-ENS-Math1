%-------------------------------------------------------------------------------
\subsubsection{Système dynamique linéaire.} 
%-------------------------------------------------------------------------------

On considère le système dynamique suivant
$$
\left\{\begin{array}{rcl}
        \dot x & = & -a_1 x + b_1 y + c_1 \\ 
        \dot y & = & -a_2 x + b_2 y + c_2
        \end{array}\right.
$$
où tous les coefficients constants sont strictement positifs.
\begin{enumerate}
  \item À quelle condition y a-t-il un unique équilibre ? Lorsque c’est le cas, à quelle condition est-il stable ?
  \solution{\todo{}}
  \item Lorsqu’il n’existe pas d’équilibre unique, représenter les deux isoclines (c'est à dire les ensemble de points ou s'annule $\dot x$ d'une part et $\dot y$ d'autre part). Y a-t-il une infinité d’équilibres ou aucun équilibre ?
  \solution{\todo{}}
  \item Représenter les orbites dans le plan de phase.
  \solution{\todo{}}
\end{enumerate}
