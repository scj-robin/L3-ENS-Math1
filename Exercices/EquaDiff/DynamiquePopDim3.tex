%-------------------------------------------------------------------------------
\subsubsection{Dynamique de population à trois classes}
%-------------------------------------------------------------------------------

On considère une population sexuée panmictique, au sein de laquelle on désigne
respectivement par $x(t)$, $y(t)$ et $z(t)$ les densités au temps $t$ de femelles flottantes, de mâles flottants, et de couples. On suppose que la dynamique de la population respecte le système dynamique suivant
\begin{equation} \label{eq:DynPop}
  \left\{\begin{array}{rcl}
          \dot x(t) & = & - \alpha x y + r z, \\
          \dot y(t) & = & - \alpha x y + r z, \\
          \dot z(t) & = & + \alpha x y - c z^2,
          \end{array} \right.
\end{equation}
où les coefficients $\alpha$, $r$ et $c$ sont strictement positifs.
\begin{enumerate}
  \item Interpréter ces équations et la signification de chacun des coefficients $\alpha$, $r$ et $c$.
  \solution{\todo{}}
  \item En notant $S = x(0) - y(0)$, montrer que $x(t) - y(t) = S$ pour tout temps $t$. En
  déduire les fonctions $y$ et $z$ satisfont le système 
  \begin{equation} \label{eq:DynPop2}
    \left\{\begin{array}{rcl}
            \dot y(t) & = & - \alpha (y^2 + Sy) + r z, \\
            \dot z(t) & = & + \alpha (y^2 + Sy) - c z^2.
            \end{array} \right.
  \end{equation}
  Dans la suite on supposera que $S > 0$.
  \solution{\todo{}}
  \item Déterminer les points d'équilibre du système \eqref{eq:DynPop2}.
  \solution{\todo{}}
  \item \'Ecrire la matrice jacobienne du système \eqref{eq:DynPop2} et étudier la nature du ou des équilibres non triviaux.
  \solution{\todo{}}
\end{enumerate}

