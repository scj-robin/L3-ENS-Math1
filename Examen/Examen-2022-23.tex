\documentclass[french, 12pt]{article}

%-------------------------------------------------------------------------------
\usepackage[a4paper,top=2cm,bottom=2cm,left=2cm,right=2cm,marginparwidth=1.75cm]{geometry}
\usepackage{amsmath,amsfonts,amssymb,amsthm}
\usepackage[french]{babel}
\usepackage[utf8]{inputenc}
\usepackage[T1]{fontenc}
\usepackage{enumerate}
\usepackage{natbib}
\usepackage{graphicx}
\usepackage{xspace}
\usepackage{color,xcolor}
\usepackage{tikz}
\usepackage{remreset}
\usepackage{url}
% \usepackage{extsizes} % Permet \documentclass[french, 14pt]{extreport}
% \usepackage[a4paper,top=1cm,bottom=2cm,left=1cm,right=1cm,marginparwidth=.75cm]{geometry}
% \usepackage{minitoc}

\graphicspath{{../Figures/}}
% Environnement
\newtheorem{theorem}{Théorème}
\newtheorem{definition}{Définition}
\newtheorem{lemma}{Lemme}
\newtheorem{proposition}{Proposition}
\newtheorem*{theorem*}{Théorème}
\newtheorem*{definition*}{Définition}
\newtheorem*{proposition*}{Proposition}
\newtheorem*{corollary*}{Corollaire}
\newtheorem*{assumption*}{Hypothèse}
\newtheorem*{algorithm*}{Algorithme}
\newtheorem*{lemma*}{Lemme}
\newtheorem*{remark*}{Remarque}
\newtheorem*{exercise*}{Exercice}
\newtheorem{exercise}{Exercice}
\newcommand{\remark}{\bigskip\noindent\textbf{\textsl{Remarque.}}\xspace}
\newcommand{\remarks}{\bigskip\noindent\textbf{\textsl{Remarques.}}\xspace}
\newcommand{\parSR}[1]{\paragraph*{\textsl{#1}}\xspace}
\renewcommand{\proof}{\bigskip\noindent\underline{\textsl{Démonstration}.}\xspace}
\newcommand{\eproof}{$\blacksquare$}

% Effets, couleurs
\newcommand{\emphase}[1]{\textcolor{red}{#1}}
\newcommand{\demoProp}[1]{\noindent{\textbf{\textsl{Démonstration de la proposition \ref{#1} :}}}}
\newcommand{\itemdot}{\textbullet}

% Moments
\DeclareMathOperator{\Esp}{\mathbb{E}}
\DeclareMathOperator{\diag}{diag}
\DeclareMathOperator{\Cov}{\mathbb{C}ov}
\DeclareMathOperator{\tr}{tr}
\DeclareMathOperator{\Var}{\mathbb{V}}
\let\Pr\relax\DeclareMathOperator{\Pr}{\mathbb{P}}
\renewcommand{\d}{\text{d}}

% R, N, ...
\newcommand{\cst}{\text{cst}}
\newcommand{\Cbb}{\mathbb{C}}
\newcommand{\Ibb}{\mathbb{I}}
\newcommand{\Nbb}{\mathbb{N}}
\newcommand{\Rbb}{\mathbb{R}}
\newcommand{\Zbb}{\mathbb{Z}}

% Indicateurs

% Lois et ensembles
\newcommand{\Acal}{\mathcal{A}}
\newcommand{\Bcal}{\mathcal{B}}
\newcommand{\Ccal}{\mathcal{C}}
\newcommand{\Ecal}{\mathcal{E}}
\newcommand{\Gcal}{\mathcal{G}}
\newcommand{\Ical}{\mathcal{I}}
\newcommand{\Lcal}{\mathcal{L}}
\newcommand{\Mcal}{\mathcal{M}}
\newcommand{\Ncal}{\mathcal{N}}
\newcommand{\Pcal}{\mathcal{P}}
\newcommand{\Rcal}{\mathcal{R}}
\newcommand{\Scal}{\mathcal{S}}
\newcommand{\Ucal}{\mathcal{U}}
\newcommand{\Xcal}{\mathcal{X}}
\newcommand{\Ycal}{\mathcal{Y}}

% Comments
\newcommand{\SR}[2]{\textcolor{gray}{#1}\textcolor{red}{#2}}
\newcommand{\todo}[1]{\textcolor{red}{\`A faire~: {\sl #1}}}
\newcommand{\dessin}[1]{
\begin{center}\framebox{\begin{minipage}{\textwidth}
  \textcolor{purple}{#1}
\end{minipage}}\end{center}
\bigskip
}
\newcommand{\progres}[1]{
\begin{center}\framebox{\begin{minipage}{\textwidth}
  \textcolor{blue}{{\sl #1}}
\end{minipage}}\end{center}
\bigskip
}
\newcommand{\solution}[1]{
\begin{center}\framebox{\begin{minipage}{\textwidth}
  \noindent{\sl Solution :}
  #1
\end{minipage}}\end{center}
\bigskip
}
% \newcommand{\exemple}[1]{
% \begin{center}\framebox{\begin{minipage}{\textwidth}
%   \parSR{Exemple.}
%   #1
% \end{minipage}}\end{center}
% \bigskip
% }
\newcommand{\exemple}[1]{
\begin{breakbox}
  \parSR{Exemple.}
  #1
\end{breakbox}
\bigskip
}

\newcommand{\SRcorrect}[2]{\textcolor{gray}{#1}\textcolor{blue}{#2}}
\newcommand{\SRcomment}[1]{\textcolor{blue}{[{\sl SR: #1}]}}



% Section numbering
\usepackage{chngcntr}
\renewcommand{\thepart}{\Roman{part}}
% \counterwithout{section}{part}
\setcounter{secnumdepth}{3}
\setcounter{tocdepth}{1}

% Proposition numbering
% \numberwithin{proposition}{section}
% \numberwithin{exercise}{section}
% \numberwithin{equation}{section}

% %-------------------------------------------------------------------------------
% %-------------------------------------------------------------------------------
% \title{\Huge{Ce qu'un biologiste doit savoir en mathématiques}}
% \author{SR d'après \cite{Lam20} : Annexes}
% \date{\today}


\title{}

%-------------------------------------------------------------------------------
%-------------------------------------------------------------------------------
\begin{document}
%-------------------------------------------------------------------------------
%-------------------------------------------------------------------------------

\begin{centering}
  \footnotesize{\sc École Normale Supérieure de Paris} 
  
  \bigskip
  \footnotesize{\sc Licence de Biologie L3}
  
  \bigskip
  \footnotesize{\sc Année 2022–23}
  
  \bigskip
  {\bf Mathématiques I : ce qu’un biologiste ne doit pas ignorer} 
  
  \bigskip
  {\sc Examen}
  
\end{centering}

\bigskip
L'épreuve dure 2 heures. 
Les notes de cours individuelles sont autorisées.
L’usage de tout appareil électronique est interdit à l’exception d’une calculatrice.

%-------------------------------------------------------------------------------
\exemple{
  On souhaite déterminer les points d'équilibre (et leur nature) du système
  $$
  \dot y = F(y) = \mu y + 2 y^3 - y^5.
  $$
  On a 
  $$
  F(y) = y(\mu + y^2 - y^4)
  $$
  qui s'annule pour $y = 0$ et pour les solutions de $(\mu + 2 y^2 - y^4)$. En posant, $z = y^2$, $\mu + 2 z - z^2 = 0$ admet des solutions si $\Delta =  4(1 + \mu) \geq 0$, soit $\mu \geq -1$. Ces solutions sont alors $z^* = -1 \pm \sqrt{1+\mu}$. La seule solution possiblement positive est $z^* = -1 + \sqrt{1+\mu}$ et elle l'est ssi $\mu > 1$. \\
  Le système admet donc un unique point fixe $x^*=0$ si $\mu < 1$ et un second point fixe $x^* = -1+\sqrt{1+\mu}$ si $\mu > 1$. \\
  On a de plus
  $$
  F'(x) = \mu + 3 y^2 - 5y^4
  $$
  dont le signe est celui de $\mu$ pour $x=0$ et \todo{nature de $x^* = -1+\sqrt{1+\mu}$ si $\mu > 1$.}
}

%-------------------------------------------------------------------------------
\todo{Voir L3 Bio SU : TD1, exercice 2}

%-------------------------------------------------------------------------------
%-------------------------------------------------------------------------------
\end{document}
%-------------------------------------------------------------------------------
%-------------------------------------------------------------------------------


