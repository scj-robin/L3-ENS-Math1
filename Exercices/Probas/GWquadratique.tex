%-------------------------------------------------------------------------------
\subsubsection{Processus de Galton-Watson quadratique} 
%-------------------------------------------------------------------------------
  On consid\`ere un processus de Bienaym\'e-Galton-Watson partant d'une population de taille 1 et dans laquelle le nombre de descendants par individu suit la loi suivante :
  $$
  \Pr\{X = k\} = \left\{
    \begin{array}{ll}
      1 - a - b & \text{si $k = 0$} \\
      a & \text{si $k = 1$} \\
      b & \text{si $k = 2$}
    \end{array}\right..
  $$
  \begin{enumerate}
    \item Montrer que la fonction g\'en\'eratrice de $X$ est
    $$
    f_X(s) = 1 - a - b + as + bs^2.
    $$
    \solution{
    Par d\'efinition, on a
    \begin{align*}
        f_X(s) = \Esp(s^X) = (1 - a - b) s^0 + a s + b s^2 = (1 - a - b) + as + b s^2.
    \end{align*}
    }
    \item En d\'eduire que le nombre de descendants moyen par individu est
    $$
    m = a + 2b.
    $$
    \solution{
    On sait que $m = f'_X(1)$, or on a $f'_X(s) = a + 2bs$, soit $m = f'_X(1) = a + 2b$.
    }
    \item Donner la condition pour que la probabilit\'e d'extinction $q$ soit strictement inf\'erieure \`a 1 et, dans ce cas, donner sa valeur.
    \solution{
    La probabilit\'e d'extinction v\'erifie $q = f(q)$, soit
    $(1 - a -b) + (a-1)q + bq^2 = 0$
    dont le discriminant est
    $$
    \Delta = (a-1)^2 - 4 b (1 - a -b) = ... = (a + 2b - 1)^2 = (m-1)^2
    $$
    d'o\`u
    $$
    q 
    = \frac{(1-a) \pm (m-1)}{2(1 - a - b)}
    = \left\{\begin{array}{ll}
      1 & (+) \\
      b/(1- a - b) & (-)
    \end{array}\right..
    $$
    $b/(1- a - b)$ est la plus petite solution (donc la probabilit\'e d'extinction) ssi 
    $b/(1- a - b) < 1$, soit $a + 2b > 1$, c'est-à-dire $m > 1$.
    }
  \end{enumerate}
