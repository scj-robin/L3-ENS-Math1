%-------------------------------------------------------------------------------
\subsection{Application linéaire tangente}
%-------------------------------------------------------------------------------

%-------------------------------------------------------------------------------
\begin{exercise}[Application linéaire tangente] \label{MultiVar-Tangente}
  Soit la fonction
  $$
  \begin{array}{rrcl}
    f :  & \Mcal_n(\Rbb) & \mapsto & \Mcal_n(\Rbb) \\
    & A & \rightarrow & A^3.
  \end{array}
  $$
  Déterminer son application linéaire tangente $D_A f$ en toute matrice $A$ de $\Mcal_n(\Rbb)$.
\end{exercise}

\solution{\todo{}}

%-------------------------------------------------------------------------------
\subsection{Matrice jacobienne}
%-------------------------------------------------------------------------------

%-------------------------------------------------------------------------------
\begin{exercise}[Jacobienne] \label{MultiVar-Jacobienne}
  On considère une fonction $F$ de $[0, 1] \times [0, 1]$ dans $[0, 1] \times [0, 1]$ de classe $C^1$ qu'on écrit
  $$
  F(x, y) = (f_0(x, y), f_1(x, y)).
  $$
  \begin{enumerate}
    \item \'Ecrire la matrice jacobienne $J(x, y)$ de la fonction $F$ en tout $(x, y) \in [0, 1] \times [0, 1]$.
    \item \'Ecrire l'application linéaire tangente de $F \circ F$ en tout point $(x, y)$.
    \item On suppose que $F(1, 1) = (1, 1)$. Donner la matrice jacobienne $J_n(1, 1)$ de la fonction $F$ itérée $n$ fois avec elle-même, en $(x, y) = (1, 1)$.
  \end{enumerate}
\end{exercise}

\solution{\todo{}}
