%-------------------------------------------------------------------------------
\subsubsection{Evolution de plusieurs populations de Galton–Watson}
%-------------------------------------------------------------------------------

On considère une espèce dont les générations sont régies par un processus de Bienaymé-Galton-Watson. On note $f$ la fonction génératrice du nombre de descendants et $q$ la probabilité d'extinction d'une population issue d'un unique individu.

On introduit un nombre aléatoire $N$ d'individus de cette espèce dans une île. On note $g$ la fonction génératrice de $N$. On suppose que les descendances des $N$ individus fondateurs évoluent indépendamment les unes des autres. On s'intéresse à la probabilité $p$ de colonisation, c'est-à-dire à la probabilité pour que cette population ne s'éteigne pas.
\begin{enumerate}
  \item A quelle condition $p$ est-elle non nulle ?
  \solution{La colonisation réussit si la descendance d'au moins un individu fondateur ne s'éteint pas, ce qui se produit avec probabilité $1 - q^N$. $p$ est donc nulle pour tout $N$ si $q = 1$ et non nulle ssi $q < 1$, c'est à dire si $b < a$.}
  \item Exprimer $p$ en fonction de $q$. 
  \solution{$1 - q^N$ est la probabilité de colonisation conditionnelle à l'effectif initial $N$. La probabilité de colonisation vaut donc
  $$
  p = \Esp\left(1 - q^N\right) = 1 - \Esp(q^N) = 1 - g(q).
  $$}
  \item Déterminer $p$, d'une part, si $N = 1$ et, d'autre part, si $N$ est distribué comme $X$. \\
  Commenter.
  \solution{
  \begin{description}
    \item[$N = 1$ :] on a immédiatement $p = 1 - q$. 
    \item[$N \overset{\Delta}{=} X$:] alors $g = f$ et donc
    $$
    p = 1 - g(q) = 1 - f(q) = 1 - q
    $$
    puisque $q$ est un point fixe de $f$ ($f(q) = q$). 
  \end{description}
  L'égalité des résultats n'est pas surprenante puisque le second cas correspond au premier décalé d'une génération.}
\end{enumerate}

