%-------------------------------------------------------------------------------
\subsection{Chaînes de Markov}
%-------------------------------------------------------------------------------

%-------------------------------------------------------------------------------
\begin{exercise}[Chaîne de Markov] \label{exo:ChaineMarkov}
  On considère une chaîne de Markov $(X_n)_{n \geq 0}$ prenant ses valeurs dans $\{1, 2, 3, 4\}$, et on note $p(i, j)$ la probabilité de transition de $i$ à $j$. On suppose que $p(1, 1) = p(4, 4) = 1$, que $p(2, 3) = 1 - p(2, 1) =: p$ et que $p(3, 2) = 1 - p(3, 4) =: q$.
  \begin{enumerate}
   \item Donner la matrice de transition de cette chaîne de Markov.
   \item Quelles sont les classes de communication de cette chaîne ? Donner leur nature. Quels sont les comportements asymptotiques possibles de cette chaîne ?
   \item On appelle $x_i$ la probabilité que la chaîne soit absorbée en 1 sachant que $X_0 = i$. Donner $x_1$ et $x_4$, ainsi que deux équations reliant $x_2$ et $x_3$, puis les calculer.
  \end{enumerate}
\end{exercise}

\solution{
  \begin{enumerate}
    \item La matrice de transiiton est
    $$
    P = \left( \begin{array}{cccc}
                1 & 0 & 0 & 0 \\
                1-p & 0 & p & 0 \\
                0 & q & 0 & 1-q \\
                0 & 0 & 0 & 1
              \end{array} \right)
    $$
    \item Les classes de communications sont 
    $$
    C_1 = \{1\}, \qquad C_2 = \{2, 3\}, \qquad C_3 = \{4\}.
    $$
    $C_1$ et $C_3$ sont absorbantes, $C_2$ est transiente. \\
    La chaîne finit nécessairement par être absorbé en 1 ou 4.
    \item D'après la question précédente, on a 
    $$
    x_1 = 1, \qquad x_4 = 0.
    $$
    Pour être absorbé en 1 partant de 3, la chaîne doit d'abord passer de 3 en 2 (avec probabilité $q$), puis être absorbée en 1 partant de 2, donc
    $$
    x_3 = q x_2.
    $$
    Partant de 2 la chaîne peut être immédiatement absorbée en 1 (avec probabilité $1-p$) ou passer en 3 (avec probabilité $p$), puis être absorbée en 1 depuis 3, donc
    $$
    x_2 = (1 - p) + p x_3.
    $$
    $x_2$ vérifie donc
    $$
    x_2 = (1 - p) + p q x_2 
    \qquad \Leftrightarrow \qquad x_2 = \frac{1-p}{1 - pq}
    \qquad \Rightarrow \qquad x_3 = \frac{q(1-p)}{1 - pq}.
    $$
    On peut aussi calculer directement $x_2$ en définissant le temps d'atteinte de l'état $1$ depuis l'état $2$:
    $$
    T_{21} = \inf\{t: X_t = 1 \mid X_0 = 2\}.
    $$
    On remarque alors que, pour $k \geq 0$,  
    $$
    \Pr\{T_{21} = 2k\} = 0, \qquad \Pr\{T_{21} = 2k + 1\} = (1-p) \; (pq)^k
    $$
    et on en déduit
    $$
    x_2 = \sum_{n \geq 0} \Pr\{T_{21} = n\} = \sum_{k \geq 0} \Pr\{T_{21} = 2k+1\} 
    = (1-p) \sum_{k \geq 0} (pq)^k = \frac{1-p}{1 - pq}.
    $$
    De même, on peut déterminer $x_3$ directement à partir de la loi du temps d'atteinte $T_{31}$.
  \end{enumerate}
}

%-------------------------------------------------------------------------------
\begin{exercise}[Processus de branchement] \label{exo:Proba-BGWgeometrique}
  On considère un processus de Bienaymé-Galton-Watson partant d'une population de taille 1 et dans laquelle le nombre de descendants par individu suit une loi géométrique $\Gcal(a)$ (pour mémoire, $X \sim \Gcal(a)$ ssi $\Pr\{X = k\} = (1-a) a^k$).
  \begin{enumerate}
    \item Calculer le nombre moyen $m$ de descendants par individus en fonction de $a$.
    \item Pour quelles valeurs de $a$ le processus est-il critique, sous-critique ou sur-critique ?
    \item Montrer que, pour $m > 1$, la probabilité d'extinction $q$ de cette population vaut $1/m$.
  \end{enumerate}
\end{exercise}

\solution{
  \begin{enumerate}
    \item On a
    $$
    m = \Esp(X) 
    = \sum_{k \geq 0} (1-a) \; a^k \; k 
    = a(1-a) \sum_{k \geq 0} k \; a^{k-1}
    $$
    où on reconnaît la dérivée de la fonction $f(a) = \sum_{k \geq 0} a^{k} = (1-a)^{-1}$, soit $f'(a) = (1-a)^{-2}$. On a donc
    $$
    m
    = a(1-a) (1-a)^{-2}
    = a / (1-a)
    $$
    (qui est une fonction monotone croissante de $a$).
    \item On a $m \leq 1 \; \Leftrightarrow \; a \leq 1-a \; \Leftrightarrow \; a \leq 1/2$.  Le processus est donc critique pour $a = 1/2$, sous critique pour $a < 1/2$ et sur-critique pour $a > 1/2$.
    \item On sait que la probabilité d'extinction $q$ partant d'une population de taille 1 est le plus petit point fixe de l'équation $s = f_X(s)$ où $f_X$ est la fonction génératrice des probabilités de la loi géométrique : 
    $$
    f_X(s) = (1 - a) / (1 - as).
    $$
    On cherche donc à résoudre $1 - a = (1 - as) s \; \Leftrightarrow \; as^2 - s + (1-a) = 0$ dont on sait que $s = 1$ est solution. On peut donc factoriser
    $$
    as^2 - s + (1-a) = a (s-1) \left(s  - \frac{1-a}a\right)
    $$
    où $(1-a)/a < 1$ dès que $a > 1/2$. La probabilité d'extinction vaut donc
    $$
    q = \frac{1-a}a = \frac1m.
    $$
  \end{enumerate}
}
  
%-------------------------------------------------------------------------------
\begin{exercise}[Fonction génératrice] \label{exo:Proba-Generatrice}
  Soit $(X, Y)$ un couple de variables aléatoires entières et $\phi: [0, 1] \times [0, 1] \mapsto [0, 1]$ sa fonction génératrice de $(X, Y)$ définie par 
  $$
  \phi(s, t) = \Esp(s^X t^Y) = \sum_{x \geq 0} \sum_{y geq 0} \Pr\{X=x, Y=y\} \; s^x \; t^y.
  $$
  \begin{enumerate}
   \item Que valent $\phi(1, 1)$, $\phi(1, 0)$, $\phi(0, 1)$ et $\phi(0, 0)$ ?
   \item Exprimer les fonctions génératrices de $X$, de $Y$ et de $X+Y$ en fonction de $\phi$.
   \item Montrer que, si $X$ et $Y$ sont indépendantes, alors $\phi(s, t) = \phi(s, 1) \; \phi(1, t)$.
   \item Montrer que si les couples $(X_1, Y_1)$ et $(X_2, Y_2)$ ont pour fonctions génératrices respectives $\phi_1$ et $\phi_2$, alors le couple $(X_1+X_2, Y_1+Y_2)$ a pour fonction génératrice $\psi = \phi_1 \phi_2$. 
   \item En déduire que si $N$ est une variable aléatoire entière de fonction génératrice $f$ et que $((X_i, Y_i))_{i \geq 1}$ sont des couples i.i.d. de même fonction génératrice $\phi$, alors le couple $(U, V)$ tel que
   $$
   U = \sum_{i=1}^N X_i, \qquad V = \sum_{i=1}^N Y_i
   $$
   a pour fonction génératrice $f \circ \phi$.
  \end{enumerate}
\end{exercise}

\solution{\todo{}}
