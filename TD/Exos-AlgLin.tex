%-------------------------------------------------------------------------------
\subsection{Exemples de matrices et déterminants}%-------------------------------------------------------------------------------

%-------------------------------------------------------------------------------
\paragraph{Calculs de déterminants.}
Calculer les déterminants des matrices suivantes 
\begin{align*}
%     A_1 & = \left[\begin{array}{rrr}
%       2 & -1 & 3 \\ 2 & -1 & 6 \\ -2 & 1 & 0
%       \end{array}\right], &
%     %
%     A_2 & = \left[\begin{array}{rrr}
%       2 & -1 & 3 \\ 2 & -1 & 6 \\ 1 & 0  & 2
%       \end{array}\right], \\
%     %
  A_3 & = \left[\begin{array}{rrr}
    1 & 1 & 0 \\ -5 & -2 & 5 \\ -1 & 0 & 2
    \end{array}\right], &
  A_4 & = \left[\begin{array}{rrr}
    -1 & 3 & 1 \\ 0 & 2 & 1 \\ 2 & 1 & 2
    \end{array}\right].  
\end{align*}

\solution{
  \begin{align*}
%     |A_1| & = 0 & & \text{colonne } 1 = -2 \text{ colonne } 2 \\
%     %
%     |A_2| & = 1 \times (-3) + 2 \times 0 = -3 & & \text{développement / dernière ligne} \\
    %
    |A_3| & = 1 \times (-4) - 1 \times (-5) = 1 & & \text{développement / première ligne} \\
    %
    |A_4| & = -1 \times 3 + 2 \times 1 = -1 & &  \text{développement / première colonne}
  \end{align*}
}

%-------------------------------------------------------------------------------
\paragraph{Matrices diagonalisables ?}
Déterminer le polynôme caratéristique des matrices suivantes et en déduire si elles sont diagonalisables
\begin{align*}
  A_1 & = \left[\begin{array}{rrr}
    2 & -1 & 3 \\ 2 & -1 & 6 \\ 1 & 0  & 2
    \end{array}\right], &
  A_2 & = \left[\begin{array}{rrr}
    1 & 1 & 0 \\ -5 & -2 & 5 \\ -1 & 0 & 2
    \end{array}\right].
\end{align*}
  
\solution{
  \begin{description}
    \item[$A_1$ :] on a
    \begin{align*}
        P_A(\lambda) 
        = \left| \begin{array}{rrr}
          2 - \lambda & -1 & 3 \\ 2 & -1 - \lambda & 6 \\ 1 & 0  & 2 - \lambda 
          \end{array}\right|
        = -\lambda^3 + 3 \lambda^2 + \lambda - 3
        = -(\lambda - 3) (\lambda - 1) (\lambda + 1).
    \end{align*}
    Les valeurs propres sont donc $3$, $1$ et $-1$ qui sont toutes réelles et distinctes, donc $A_1$ est diagonalisable.
    \item[$A_1$ :] on a
    \begin{align*}
        P_A(\lambda) 
        = \left| \begin{array}{rrr}
            1 - \lambda & 1 & 0 \\ -5 & -2 - \lambda & 5 \\ -1 & 0 & 2 - \lambda
          \end{array}\right|
        = - \lambda^3 + \lambda^2 - \lambda + 1
        = - (\lambda-1) (\lambda^2 + 1).
    \end{align*}
    Les valeurs propres sont donc $1$, $i$ et $-i$ qui ne sont pas toutes réelles, donc $A_2$ n'est pas diagonalisable.
  \end{description}
}

%-------------------------------------------------------------------------------
\paragraph{Matrice paramétrée}
Soit la matrice
\begin{align*}
  A & = \left[\begin{array}{rrr}
    1 & 4 & 2 \\ 0 & \alpha & 0 \\ 1 & 1 & 0
    \end{array}\right]
\end{align*}
\begin{enumerate}
  \item Déterminer son polynôme caractéristique.
  \solution{En développant par rapport à la 2ème ligne, on obtient
    $$
    P_A(\lambda) 
    = (\alpha - \lambda) 
      \left| \begin{array}{cc} 1 - \lambda & 2 \\1 & - \lambda \end{array} \right|
    = (\alpha - \lambda) (\lambda^2 - \lambda - 2)
    = (\alpha - \lambda) (\lambda - 2 ) ( \lambda + 1).
    $$
  }
  \item En déduire ses valeurs propres et sa valeur propres dominante en fonction de $\alpha$.
  \solution{Les valeurs propres de $A$ sont donc $\{\alpha, 2, -1\}$. La valeur propre dominante    est donc $\max(\alpha, 2)$.}
  \item A quelle condition sur $\alpha$ $A$ est-elle diagonalisable ?
  \solution{
    \begin{description}
    \item[$\alpha \notin \{-1, 2\}$:] les 3 valeurs propres sont réelles et distinctes donc $A$ est diagonalisable.
    \item[$\alpha = 2$:] $2$ est alors raçine double de $P_A(\lambda)$. Ses vecteurs propres associés sont solutions de 
    $$
    \left\{ \begin{array}{rcl}  
            u_1 + 4 u_2 + 2 u_3 & = & 2 u_1 \\
            2 u_2 & = & 2 u_2 \\
            u_1 + u_2 & = & 2 u_3 
            \end{array} \right.
    \quad \Leftrightarrow \quad
    \left\{ \begin{array}{rcl}  
            2 u_1 + 5 u_2 & = & 2 u_1 \\
            u_1 + u_2 & = & 2 u_3 
            \end{array} \right.         
    \quad \Leftrightarrow \quad
    \left\{ \begin{array}{rcl}  
            u_2 & = & 0 \\
            u_1 & = & 2 u_3 
            \end{array} \right.         
    $$
    qui définit un sous-espace propre de dimension 1 : $A$ n'est donc pas diagonalisable.
    \item[$\alpha = -1$:] $-1$ est alors raçine double de $P_A(\lambda)$. Ses vecteurs propres associés sont solutions de 
    $$
    \left\{ \begin{array}{rcl}  
            u_1 + 4 u_2 + 2 u_3 & = & - u_1 \\
            2 u_2 & = & - u_2 \\
            u_1 + u_2 & = & - u_3 
            \end{array} \right.
    \quad \Leftrightarrow \quad
    \left\{ \begin{array}{rcl}  
            u_1 + 2 u_3 & = & - u_1 \\
            u_2 & = & 0 \\
            u_1& = & - u_3 
            \end{array} \right.
    \quad \Leftrightarrow \quad
    \left\{ \begin{array}{rcl}  

    u_2 & = & 0 \\
            u_1 & = & - u_3 
            \end{array} \right.
    $$
    qui définit un sous-espace propre de dimension 1 : $A$ n'est donc pas diagonalisable.
  \end{description}
  }
\end{enumerate}


%-------------------------------------------------------------------------------
\subsection{Propriétés générales du déterminant}%-------------------------------------------------------------------------------

%-------------------------------------------------------------------------------
\paragraph{Polynôme caractéristique et trace}
On définit la trace $\tr(A)$ d'une matrice carrée $A \in \Mcal_n$ comme la somme de ses termes diagonaux : $\tr(A) = \sum_{i=1}^n a_{ii}$.
\begin{enumerate}
  \item Montrer que le coefficient d'ordre $n-1$ du polynôme caratéristique de $A$ (noté $P_A$) vaut
  $$
  [\lambda^{n-1}] P_A(\lambda) = (-1)^{n-1} \tr(A).
  $$
  On pourra procéder par récurrence.
  \solution{
    On vérifie facilement que la propriété est vraie pour $n=2$. En la supposant vraie au rang $n-1$, on peut calculer le polynôme caractéristique de $A \in \Mcal_n$ en développant par la dernière ligne : 
    $$
    P_A(\lambda) = 
    \left| A - \lambda I\right|
    = \sum_{j=1}^{n-1} (-1)^{n+j} a_{nj} \left| (A - \lambda I)^{(n, j)} \right|
    + (a_{nn} -\lambda) \left| (A - \lambda I)^{(n, n)} \right|
    $$
    en notant $B^{(i,j)}$ la matrice $B$ privée de sa $i$ème ligne et $j$ème colonne. On peut alors remarquer que les termes de la première somme sont tous de degré au plus $n-2$ et que $\left|(A - \lambda I)^{(n, n)} \right| = P_{A^{(n, n)}}(\lambda)$. On a donc, en notant $Q_m$ un polynôme quelconque de degré inférieur ou égal à $m$,
    \begin{align*}
      P_A(\lambda) 
      & = (a_{nn} -\lambda) P_{A^{(n, n)}}(\lambda) + Q_{n-2}(\lambda) \\
      & = (a_{nn} -\lambda) \left( (-1)^{n-1} \lambda^{n-1} + (-1)^{n-2} \tr(A^{(n, n)} \lambda^{n-2} + Q_{n-3}(\lambda) \right) 
      & & (\text{par hypothèse}) \\
      & = (-1)^n \lambda^n + (-1)^{n-1} \underset{\tr(A)}{\underbrace{(a_{nn} + \tr(A^{(n, n)})}} \lambda^{n-1} + Q_{n-2}(\lambda)
    \end{align*}
    donc
    $$
    [\lambda^{n-1}] P_A(\lambda) = (-1)^{n-1} \tr(A).
    $$
  }
  \item En déduire que la trace est égale à la somme des raçines $\{\lambda_1, \dots \lambda_n\}$ de $P_A(\lambda)$ :
  $$
  \tr(A) = \sum_{i=1} \lambda_i
  $$
  On pourra utiliser la factorisation du polynôme caractéristique.
  \solution{
  On utilise cette fois la version factorisée de $P_A$, soit 
  $$
  P_A(\lambda) 
  = \prod_{i=1}^n (\lambda_i - \lambda)
  $$
  où les $\lambda_i$ sont les $n$ valeurs propres (pas nécessairement distinctes ni réelles) de $A$. Lors du développement de $P_A(\lambda)$, les termes en $\lambda^{n-1}$ apparaissent en multipliant un terme $\lambda_i$ par $n-1$ termes $(-\lambda)$, c'est-à-dire
  $$
  \left( \sum_{i=1}^n \lambda_i \right) (-\lambda)^{n-1}
  = (-1)^{n-1} \left( \sum_{i=1}^n \lambda_i \right) \lambda^{n-1}.
  $$
  On a donc
  $$
  [\lambda^{n-1}] P_A(\lambda) 
  = (-1)^{n-1} \left( \sum_{i=1}^n \lambda_i \right) 
  \qquad \Leftrightarrow \qquad
  \tr(A) = \sum_{i=1}^n \lambda_i.
  $$
  }
\end{enumerate}

 
%-------------------------------------------------------------------------------
\paragraph{Calcul par la méthode des cofacteurs}.
  Montrer que, pour toute matrice $A \in \Mcal_n$ et pour tout $i_0, j_0 \in \{1, \dots, n\}$, on a 
  \begin{align*}
    |A| 
    & = \sum_{j=1}^n a_{i_0j} (-1)^{i_0+j} |A^{(i_0j)}| & (\text{développement par rapport à la ligne $i_0$}) \\
    & = \sum_{i=1}^n a_{ij_0} (-1)^{i+j_0} |A^{(ij_0)}| & (\text{développement par rapport à la colonne $j_0$})
  \end{align*}

\solution{
  \todo{Ajouter une question intermédiaire avec $i_0 = 1$ ?} \\
  On considère le développement par rapport à la ligne $i_0$. Par multilinéarité, on a
  $$
  |A| = \sum_{j=1}^n a_{i_0j} 
    \left|\begin{array}{ccccc}
      a_{11} & \cdots & a_{1j} & \cdots & a_{1n} \\
      \vdots & & \vdots & & \vdots \\
      0 & \cdots & 1 & \cdots & 0 \\
      \vdots & & \vdots & & \vdots \\
      a_{n1} & \cdots & a_{nj} & \cdots & a_{nn} \\
    \end{array}\right|.
  $$
  On effectue ensuite les $j-1$ interversions de colonnes adjacentes amenant la colonne $j$ en colonne 1, {\em en préservant l'ordre des autres colonnes entres elles}. Du fait du caractère alterné du déterminant, chaque interversion engendre un changement de signe : 
  $$
  |A| = \sum_{j=1}^n (-1)^{j-1} a_{i_0j} 
    \left|\begin{array}{ccccccc}
      a_{1j} & a_{11} & \cdots & a_{1,j-1} & a_{1,j+1} & \cdots & a_{1n} \\
      \vdots & \vdots & & \vdots & \vdots & & \vdots \\
      1 & 0 & \cdots & 0 & 0 & \cdots & 0 \\
      \vdots & \vdots & & \vdots & \vdots & & \vdots\\
      a_{nj} & a_{n1} & \cdots & a_{n,j-1} & a_{n,j+1} & \cdots & a_{nn} \\
    \end{array}\right|.
  $$
  On effectue la même opération pour amener la ligne $i_0$ en premier
  $$
  |A| = \sum_{j=1}^n (-1)^{(j-1) + (i_0-1)} a_{i_0j} 
    \left|\begin{array}{ccccccc}
      1 & 0 & \cdots & 0 & 0 & \cdots & 0 \\
      a_{1j} & a_{11} & \cdots & a_{1,j-1} & a_{1,j+1} & \cdots & a_{1n} \\
      \vdots & \vdots & & \vdots & \vdots & & \vdots \\
      a_{i_0-1, j} & a_{i_0-1, 1} & \cdots & a_{i_0-1,j-1} & a_{i_0-1, j+1} & \cdots & a_{i_0-1, n} \\
      a_{i_0+1, j} & a_{i_0+1, 1} & \cdots & a_{i_0+1, j-1} & a_{i_0+1, j+1} & \cdots & a_{i_0+1, n} \\
      \vdots & \vdots & & \vdots & \vdots & & \vdots\\
      a_{nj} & a_{n1} & \cdots & a_{n,j-1} & a_{n,j+1} & \cdots & a_{nn} \\
    \end{array}\right|.
  $$
  On utilise alors la formule du déterminant par bloc
  $$
  |A| = \sum_{j=1}^n (-1)^{i_0 + j} a_{i_0j} |[1]| \times
    \left|\begin{array}{cccccc}
      a_{11} & \cdots & a_{1,j-1} & a_{1,j+1} & \cdots & a_{1n} \\
      \vdots & & \vdots & \vdots & & \vdots \\
      a_{i_0-1, 1} & \cdots & a_{i_0-1,j-1} & a_{i_0-1, j+1} & \cdots & a_{i_0-1, n} \\
      a_{i_0+1, 1} & \cdots & a_{i_0+1, j-1} & a_{i_0+1, j+1} & \cdots & a_{i_0+1, n} \\
      \vdots & & \vdots & \vdots & & \vdots\\
      a_{n1} & \cdots & a_{n,j-1} & a_{n,j+1} & \cdots & a_{nn} \\
    \end{array}\right|
  $$
  où on reconnaît les mineurs $A^{(i_0, j)}$ et les cofacteurs $(-1)^{i_0 + j} |A^{(i_0, j}|$. \\
  La démonstration pour le développement par rapport à une colonne est symétrique.
}
 
%-------------------------------------------------------------------------------
\subsection{Inverse d'une matrice orthonormale}%-------------------------------------------------------------------------------

\begin{exercise} \label{exo:AlgLin-Trace}
  Soit $P \in \Mcal_n$ orthonormale. Montrer que $P^{-1} = P^\top$.
\end{exercise}

\solution{
  En notant $P = [v_{ij}]$, $v_j$ le $j$ème vecteur colonne de $P$ et $B = [b_{ij}] = P^\top P$, on a
  $$
  P^\top = [v_{ji}] 
  \quad \Rightarrow \quad
  b_{ik} 
  = \sum_{k=1}^n [P^\top]_{ik} [P]_{kj} 
  = \sum_{k=1}^n v_{ki} v_{kj}
  = < v_i, v_j >
  = \left\{\begin{array}{rl} 1 & \text{si } i = j \\ 0 & \text{sinon} \end{array}\right.
  $$
  (puique les vecteurs $v_j$ sont orthonormés), donc $P^\top P = B = I$. La démonstration de $P P^\top = I$ est symétrique.
}

% %-------------------------------------------------------------------------------
% \subsection{Dynamique d'une population structurée}%-------------------------------------------------------------------------------
%  \todo{Voir notes}

%-------------------------------------------------------------------------------
\subsection{Analyse en composante principales}%-------------------------------------------------------------------------------
 \todo{Voir notes + $X_i = U + V_i \Rightarrow \Sigma = \sigma^2 J + \gamma^2 I \Rightarrow$ 1 seule vp non nulle $\lambda = \gamma^2 + n \sigma^2$.}
 
% %-------------------------------------------------------------------------------
% \subsection{Matrice de rotation}
% %-------------------------------------------------------------------------------
% \todo{}

%-------------------------------------------------------------------------------
\subsection{Modèles dynamique}%-------------------------------------------------------------------------------

%-------------------------------------------------------------------------------
\paragraph{Evolution de fréquences alléliques.}
Dans une population diploïde panmictique, on s’intéresse à un gène existant sous la forme de
$m$ allèles $a_1, \dots A_m$. On désigne par $p_i$ la proportion du pool gamétique portant l’allèle $A_i$ ($\sum_i p_i = 1$) et par $p'_i$ cette proportion à la génération qui suit. La dynamique de ces fréquences alléliques est donnée par
\begin{equation} \label{eq:dynFreqModele}
  V(p) p'_i = p_i \sum_{j=1}^m a_{ij} p_j, 
  \qquad \text{où} \quad
  V(p) = \sum_i \sum_j a_{ij} p_i p_j
\end{equation}
(de sorte que $\sum_i p'_i = 1$).
On note de plus $A = [a_{ij}]$ où les $a_{ij}$ sont positifs ou nuls, mais non tous nuls.

\begin{enumerate}
  \item Interprêter cette équation. Pourquoi $A$ est-elle symétrique ?
  \solution{$a_{ij}$ = avantage relatif du génotype $(A_i, A_j)$, symétrique par nature. \\
  On peut remarquer 
  \begin{itemize}
  \item que $V(p) = p^\top A p$ est l'avantage moyen d'un descendant, du fait de la reproduction panmictique et
  \item que ce modèle est paramétré à une constante près, c'est-à-dire qu'on aboutit à la même dynamique en remplaçant $A$ par $B = k A$ pour n'importe quel $k > 0$.
  \end{itemize}
  }
\end{enumerate}

On suppose maintenant qu'il existe une équilibre $p^* = [p^*_1, \dots p^*_m]^\top$ non trivial ($\forall i, p^*_i \neq 0$). On définit la viabilité à l'équilibre par $V^* = V(p^*)$.

\begin{enumerate}
  \item Montrer que $V^* = \sum_j a_{ij} p^*_j$.
  \solution{L'équilibre $p^*$ est un point stationnaire de \eqref{eq:dynFreqModele}, on a donc
  $$
  \forall i: \quad V^* p^*_i = p^*_i \sum_{j=1}^m a_{ij} p^*_j,
  $$
  soit $V^* = \sum_{j=1}^m a_{ij} p^*_j, \forall i$. }
  \item En écrivant tout vecteur de fréquence allélique $s = p^* + x$, montrer que $V^*$ est maximale ssi
  \begin{equation} \label{eq:dynFreqCondition}
    \forall x: \sum_i x_i = 0, \qquad x^\top A x \leq 0.
  \end{equation}
  \solution{  Si $s = p^* + x$, on a 
  $$
  V(s) 
  = s^\top A s
  = {p^*}^\top A p^* + 2 x^\top A p^* + x^\top A x
  = V^* + 2 x^\top A p^* + x^\top A x.
  $$
  De plus, $s$ et $p^*$ étant des vecteurs de fréquences (leurs coordonnées somment à 1), nécessairement, si $x = p^* +x$, alors $\sum_i x_i = 1_m^\top x = 0$. Or on a vu que $\sum_j a_{ij} p^*_j$ est indépendant de $i$ (et égale à $V^*$), donc $A p^* = V^* 1_m$, et donc
  $$
  x^\top A p^* = x^\top V^* 1_m = V^* x^\top 1_m = 0.
  $$
  Enfin, $V^*$ est maximale, ssi $\forall s, V(s) = V^* + x^\top A x \leq V^*$, c'est-à-dire ssi
  \eqref{eq:dynFreqCondition}.
  }
  \item Montrer qu'on peut écrire $A$ sous la forme $A = R \Lambda R^\top$ où
  $$
  \Lambda = \text{diag}(\lambda_1, \dots, \lambda_p, 
    \lambda_{p+1}, \dots, \lambda_{p+q}, 
    0, \dots, 0)
  $$
  où $p \geq 1$, $q \geq 0$, $p+q \leq m$, $\lambda_1, \dots \lambda_p > 0$ et $\lambda_{p+1}, \dots \lambda_{p+q} < 0$.
  \solution{
  $A$ étant symétrique, elle est diagonalisable et ses vecteurs propres sont orthogonaux. On peut donc l'écrire sous la forme $A = R \Lambda R^\top$ en mettant en premier les valeurs propres strictement positives, puis strictement négatives, puis, éventuellement, nulles. \\
  La seule propriété restant à démontrer est que $p \geq 1$ (et non $p = 0$). Pour cela, on remarque que, si $p = 0$, alors tous les $\lambda_i$ sont négatifs (ou nuls), donc
  $$
  \forall x \in \Rbb^m: \qquad 
  x^\top A x 
  = x^\top R \Lambda R^\top x 
  = y^\top \Lambda y  
  = \sum_i \lambda_i y_i^2 \leq 0 
  $$
  (en posant $y = R^\top x$) or $V^* = {p^*}^\top A p^* > 0$, donc on a nécessairement $p \geq 1$.
  }
  \item Montrer que \eqref{eq:dynFreqCondition} peut s'écrire 
  $$
  \forall y: \sum_i q_i y_i = 0, \qquad \sum_i \lambda_i y_i^2 \leq 0, 
  $$
  où $q$ est un vecteur à préciser.
  \solution{
  \todo{A reprendre : $q =$ 1er vecteur propre de $A$.}
  D'après la question précédente, en posant $y = R^\top x \Leftrightarrow x = R y$, on a que 
  $$
  \{x^\top A x \leq 0\} 
  \qquad \Leftrightarrow \qquad 
  \left\{y^\top \Lambda y = \sum_i \lambda_i y_i^2 \leq 0\right\}.
  $$
  Il reste à formuler la condition $\sum_i x_i = 0$ qui équivaut à 
  $$
  \left\{\sum_i \sum_j r_{ij} y_j = 0\right\}
  \quad \Leftrightarrow \quad 
  \left\{\sum_j y_j q_j = 0\right\} 
  \qquad \text{en posant $q_j = \sum_i r_{ij}$}.
  $$
  }
  \item En déduire que $V^\top$ est maximale si $p=1$ et $q \geq 1$.
  \solution{On a vu que $V^*$ est maximale si $x^\top A x \leq 0$ pour tout $x$ vérifiant $\sum_i x_i = 0$, qui définit une sous-espace de dimension $m-1$. $p$ ne peut donc pas être supérieur ou égal à 2 (car alors il existerait deux dimensions dans lesquelles $x^\top A x > 0$). \\
  La condition $q \geq 1$ assure seulement qu'il existe des vecteurs de fréquence $p$ donnant une viabilité $V(p) < V^*$.
  }
\end{enumerate}
