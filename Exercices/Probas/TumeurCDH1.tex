%-------------------------------------------------------------------------------
\subsection{Modèle d'apparition de tumeurs}
%-------------------------------------------------------------------------------

% Inspiré du modèle d'apparition des tumeurs CDH1 de type II, avec A. Bonnet
% https://www.overleaf.com/project/61fa6f1f2e1b2e66da2e825b
% Transféré dans /home/robin/Bureau/RECHERCHE/GENETIQUE/CDH1/Doc/CDH1_Poisson

\paragraph{Modèle.}
On considère l'apparition de tumeurs en deux étapes. Les sites tumoraux de type I (bénins) apparaissent selon en processus de Poisson homogène d'intensité $\lambda$. Chaque tumeur de type I se transforme ensuite en une tumeur de type II (malignes) au bout d'un temps exponentiel de paramètre $\mu$.

On note $N(t)$ le nombre total de tumeurs apparues au temps $t$ : 
$$
\{N(t)\}_{t \geq 0} \sim PP(\lambda).
$$
et $M(t)$ le nombre de tumeurs de type II au temps $t$. Le nombre de site de type I au temps $t$ vaut donc $N(t) - M(t)$.

\bigskip
\paragraph{Modèle.}
\begin{enumerate}
  \item Donner la loi du nombre total $N(t)$ de tumeurs au temps $t$.  
  \solution{Propriété du processus de Poisson: $N(t) \sim \Pcal(\lambda t)$}
\end{enumerate}
