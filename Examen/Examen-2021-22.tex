\documentclass[french, 12pt]{article}

%-------------------------------------------------------------------------------
\usepackage[a4paper,top=2cm,bottom=2cm,left=2cm,right=2cm,marginparwidth=1.75cm]{geometry}
\usepackage{amsmath,amsfonts,amssymb,amsthm}
\usepackage[french]{babel}
\usepackage[utf8]{inputenc}
\usepackage[T1]{fontenc}
\usepackage{enumerate}
\usepackage{natbib}
\usepackage{graphicx}
\usepackage{xspace}
\usepackage{color,xcolor}
\usepackage{tikz}
\usepackage{remreset}
\usepackage{url}
% \usepackage{extsizes} % Permet \documentclass[french, 14pt]{extreport}
% \usepackage[a4paper,top=1cm,bottom=2cm,left=1cm,right=1cm,marginparwidth=.75cm]{geometry}
% \usepackage{minitoc}

\graphicspath{{../Figures/}}
% Environnement
\newtheorem{theorem}{Théorème}
\newtheorem{definition}{Définition}
\newtheorem{lemma}{Lemme}
\newtheorem{proposition}{Proposition}
\newtheorem*{theorem*}{Théorème}
\newtheorem*{definition*}{Définition}
\newtheorem*{proposition*}{Proposition}
\newtheorem*{corollary*}{Corollaire}
\newtheorem*{assumption*}{Hypothèse}
\newtheorem*{algorithm*}{Algorithme}
\newtheorem*{lemma*}{Lemme}
\newtheorem*{remark*}{Remarque}
\newtheorem*{exercise*}{Exercice}
\newtheorem{exercise}{Exercice}
\newcommand{\remark}{\bigskip\noindent\textbf{\textsl{Remarque.}}\xspace}
\newcommand{\remarks}{\bigskip\noindent\textbf{\textsl{Remarques.}}\xspace}
\newcommand{\parSR}[1]{\paragraph*{\textsl{#1}}\xspace}
\renewcommand{\proof}{\bigskip\noindent\underline{\textsl{Démonstration}.}\xspace}
\newcommand{\eproof}{$\blacksquare$}

% Effets, couleurs
\newcommand{\emphase}[1]{\textcolor{red}{#1}}
\newcommand{\demoProp}[1]{\noindent{\textbf{\textsl{Démonstration de la proposition \ref{#1} :}}}}
\newcommand{\itemdot}{\textbullet}

% Moments
\DeclareMathOperator{\Esp}{\mathbb{E}}
\DeclareMathOperator{\diag}{diag}
\DeclareMathOperator{\Cov}{\mathbb{C}ov}
\DeclareMathOperator{\tr}{tr}
\DeclareMathOperator{\Var}{\mathbb{V}}
\let\Pr\relax\DeclareMathOperator{\Pr}{\mathbb{P}}
\renewcommand{\d}{\text{d}}

% R, N, ...
\newcommand{\cst}{\text{cst}}
\newcommand{\Cbb}{\mathbb{C}}
\newcommand{\Ibb}{\mathbb{I}}
\newcommand{\Nbb}{\mathbb{N}}
\newcommand{\Rbb}{\mathbb{R}}
\newcommand{\Zbb}{\mathbb{Z}}

% Indicateurs

% Lois et ensembles
\newcommand{\Acal}{\mathcal{A}}
\newcommand{\Bcal}{\mathcal{B}}
\newcommand{\Ccal}{\mathcal{C}}
\newcommand{\Ecal}{\mathcal{E}}
\newcommand{\Gcal}{\mathcal{G}}
\newcommand{\Ical}{\mathcal{I}}
\newcommand{\Lcal}{\mathcal{L}}
\newcommand{\Mcal}{\mathcal{M}}
\newcommand{\Ncal}{\mathcal{N}}
\newcommand{\Pcal}{\mathcal{P}}
\newcommand{\Rcal}{\mathcal{R}}
\newcommand{\Scal}{\mathcal{S}}
\newcommand{\Ucal}{\mathcal{U}}
\newcommand{\Xcal}{\mathcal{X}}
\newcommand{\Ycal}{\mathcal{Y}}

% Comments
\newcommand{\SR}[2]{\textcolor{gray}{#1}\textcolor{red}{#2}}
\newcommand{\todo}[1]{\textcolor{red}{\`A faire~: {\sl #1}}}
\newcommand{\dessin}[1]{
\begin{center}\framebox{\begin{minipage}{\textwidth}
  \textcolor{purple}{#1}
\end{minipage}}\end{center}
\bigskip
}
\newcommand{\progres}[1]{
\begin{center}\framebox{\begin{minipage}{\textwidth}
  \textcolor{blue}{{\sl #1}}
\end{minipage}}\end{center}
\bigskip
}
\newcommand{\solution}[1]{
\begin{center}\framebox{\begin{minipage}{\textwidth}
  \noindent{\sl Solution :}
  #1
\end{minipage}}\end{center}
\bigskip
}
% \newcommand{\exemple}[1]{
% \begin{center}\framebox{\begin{minipage}{\textwidth}
%   \parSR{Exemple.}
%   #1
% \end{minipage}}\end{center}
% \bigskip
% }
\newcommand{\exemple}[1]{
\begin{breakbox}
  \parSR{Exemple.}
  #1
\end{breakbox}
\bigskip
}

\newcommand{\SRcorrect}[2]{\textcolor{gray}{#1}\textcolor{blue}{#2}}
\newcommand{\SRcomment}[1]{\textcolor{blue}{[{\sl SR: #1}]}}



% Section numbering
\usepackage{chngcntr}
\renewcommand{\thepart}{\Roman{part}}
% \counterwithout{section}{part}
\setcounter{secnumdepth}{3}
\setcounter{tocdepth}{1}

% Proposition numbering
% \numberwithin{proposition}{section}
% \numberwithin{exercise}{section}
% \numberwithin{equation}{section}

% %-------------------------------------------------------------------------------
% %-------------------------------------------------------------------------------
% \title{\Huge{Ce qu'un biologiste doit savoir en mathématiques}}
% \author{SR d'après \cite{Lam20} : Annexes}
% \date{\today}


\title{}

%-------------------------------------------------------------------------------
%-------------------------------------------------------------------------------
\begin{document}
%-------------------------------------------------------------------------------
%-------------------------------------------------------------------------------

\begin{centering}
  \footnotesize{\sc École Normale Supérieure de Paris} 
  
  \bigskip
  \footnotesize{\sc Licence de Biologie L3}
  
  \bigskip
  \footnotesize{\sc Année 2021–22}
  
  \bigskip
  {\bf Mathématiques I : ce qu’un biologiste ne doit pas ignorer} 
  
  \bigskip
  {\sc Examen}
  
\end{centering}

\bigskip
L'épreuve dure 2 heures. 
Les notes de cours individuelles sont autorisées.
L’usage de tout appareil électronique est interdit à l’exception d’une calculatrice.

%-------------------------------------------------------------------------------
%-------------------------------------------------------------------------------
% \section*{Sujet}
%-------------------------------------------------------------------------------

% %-------------------------------------------------------------------------------
% \subsection{Algèbre linéaire}
% %-------------------------------------------------------------------------------

\begin{exercise}[Algèbre linéaire] ~ \label{exam:AlgLin}
  \begin{enumerate}
% [Ex 1.3.3]
  \item On considère la matrice 
  $$
  A = \left[\begin{array}{cc} a & b \\ c & d\end{array}\right]
  $$
  Déterminer son polynôme caractéristique et donner les conditions sur $p = \tr(A)$ et $q = \det(A)$ pour que $A$ admette deux valeurs propres réelles.
  %
  \item Montrer le polynôme caractéristique de la matrice 
  \begin{align*}
    B & = \left[\begin{array}{rrr}
      -1 & 3 & 1 \\ 0 & 2 & 1 \\ \alpha & \beta & 2
      \end{array}\right].  
  \end{align*}
  est
  $$
  P_B(\lambda) = -(1+\lambda)(\lambda^2 - 4 \lambda + 4 - \alpha - \beta)
  $$
  \item En déduire une condition sur $\alpha$ et $\beta$ pour que $B$ soit diagonalisable.
  \end{enumerate}
\end{exercise}

% %-------------------------------------------------------------------------------
% \subsection{Fonction de plusieurs variables}
% %-------------------------------------------------------------------------------

% %-------------------------------------------------------------------------------
% \subsection{\'Equations différentielles}
% %-------------------------------------------------------------------------------

\begin{exercise}[Dynamique de population] ~ \label{exam:DynPop}
  On considère une population sexuée panmictique, au sein de laquelle on désigne
  respectivement par $x(t)$, $y(t)$ et $z(t)$ les densités au temps $t$ de femelles flottantes, de mâles flottants, et de couples. On suppose que la dynamique de la population respecte le système dynamique suivant
  \begin{equation} \label{eq:DynPop}
    \left\{\begin{array}{rcl}
            \dot x(t) & = & - \alpha x y + r z, \\
            \dot y(t) & = & - \alpha x y + r z, \\
            \dot z(t) & = & + \alpha x y - c z^2,
           \end{array} \right.
  \end{equation}
  où les coefficients $\alpha$, $r$ et $c$ sont strictement positifs.
  \begin{enumerate}
   \item Interpréter ces équations et la signification de chacun des coefficients $\alpha$, $r$ et $c$.
   \item En notant $S = x(0) - y(0)$, montrer que $x(t) - y(t) = S$ pour tout temps $t$. En déduire les fonctions $y$ et $z$ satisfont le système 
  \begin{equation} \label{eq:DynPop2}
    \left\{\begin{array}{rcl}
            \dot y(t) & = & - \alpha (y^2 + Sy) + r z, \\
            \dot z(t) & = & + \alpha (y^2 + Sy) - c z^2.
           \end{array} \right.
  \end{equation}
  Dans la suite on supposera que $S > 0$.
  \item Déterminer les points d'équilibre du système \eqref{eq:DynPop2}.
  \item \'Ecrire la matrice jacobienne du système \eqref{eq:DynPop2} et étudier la nature du ou des équilibres non triviaux.
  \end{enumerate}
\end{exercise}

% %-------------------------------------------------------------------------------
% \subsection{Probabilités}
% %-------------------------------------------------------------------------------

\begin{exercise}[Probabilités] ~ \label{exam:Proba}
  Les organismes d’une espèce asexuée se reproduisent suivant un processus de Bienaymé–Galton–Watson, où chaque individu laisse à la génération suivante un nombre aléatoire $X$ d’individus. \\
  On note $p_k = \Pr\{X = k\}$ et on suppose que 
  $$
  p_0 = b
  \qquad \text{et} \qquad 
  p_k = (1 - b) (1 - a) a^{k-1} \text{ pour } k \geq 1.
  $$
  \begin{enumerate}
    \item Montrer que la fonction génératrice de $X$ vaut
    $$
    f(s) := \Esp(s^X) = b + (1-b)(1-a) \frac{s}{1 - as}.
    $$
    \item Déterminer la valeur de la probabilité, notée $q$, qu'une population de cette espèce issue d’un seul individu fondateur s'éteigne.
  \end{enumerate}
  On introduit un nombre aléatoire $N$ d'individus de cette espèce dans une île. On note $g$ la fonction génératrice de $N$. On suppose que les descendances des $N$ individus fondateurs évoluent indépendamment les unes des autres. On s'intéresse à la probabilité $p$ de colonisation, c'est-à-dire à la probabilité pour que cette population ne s'éteigne pas.
  \begin{enumerate}
    \setcounter{enumi}{2}
    \item A quelle condition $p$ est-elle non nulle ?
    \item Exprimer $p$ en fonction de $q$. 
    \item Déterminer $p$ si $N = 1$ et si $N$ est distribué comme $X$. Commenter.
  \end{enumerate}
\end{exercise}


% \newpage
% %-------------------------------------------------------------------------------
%-------------------------------------------------------------------------------
% \section*{Correction}
%-------------------------------------------------------------------------------

% %-------------------------------------------------------------------------------
% \subsection{Algèbre linéaire}
% %-------------------------------------------------------------------------------

\paragraph{Solution de l'exercice \ref{exam:AlgLin}.}
  \begin{enumerate}
% [Ex 1.3.3]
  \item On a 
  $$
  |A - \lambda I_2| 
  = \left|\begin{array}{cc} a - \lambda & b \\ c & d - \lambda \end{array}\right|
  = (a - \lambda)( d - \lambda) - bc
  = \lambda^2 - (a+d) \lambda + (ad - bc)
  = \lambda^2 - p \lambda + q
  $$
  qui admet deux solutions réelles (éventuellement égales) ssi
  $$
  p^2 - 4 q \geq 0
  \qquad \Leftrightarrow \qquad
  q \leq p^2 / 4.
  $$
  \item On développe par rapport à la première colonne.
  \item La premièrre valeur propre est donc $\lambda_1 = -1$. Le discriminant du polynôme 
  $$
  Q(\lambda) = \lambda^2 - 4 \lambda + 4 - (\alpha+\beta)
  $$
  vaut $\Delta = 16 - 4(4 - (\alpha+\beta)) = \alpha + \beta$. Les deux valeurs propres restantes $\lambda_2$ et $\lambda_3$ sont donc réelles et distinctes ssi $\alpha + \beta > 0$ et valent alors
  $$
  \lambda_2 = \frac{4 + \sqrt{\alpha + \beta}}{2}, 
  \qquad \lambda_3 = \frac{4 - \sqrt{\alpha + \beta}}{2}.
  $$
  Elles sont de plus distinctes de $\lambda_1 = -1$ ssi 
  $$
  \lambda_3 \neq -1 \qquad \leftrightarrow \qquad
  \alpha + \beta \neq 9.
  $$
  Dans ce cas ($\alpha + \beta > 0$ et $\alpha + \beta \neq 9$), les 3 valeurs propres sont réelles et distinctes et $B$ est donc diagonalisable. \\
  Reste à étudier les cas 
  \begin{description}
   \item[$\alpha + \beta = 0 \Leftrightarrow \lambda_2 = \lambda_3$:] \todo{déterminer la dimension du sous-espace propre associé.} 
   \item[$\alpha + \beta = 9 \Leftrightarrow \lambda_1 = \lambda_3$:] \todo{déterminer la dimension du sous-espace propre associé.} 
  \end{description}

  \end{enumerate}

% %-------------------------------------------------------------------------------
% \subsection{Fonction de plusieurs variables}
% %-------------------------------------------------------------------------------

% %-------------------------------------------------------------------------------
% \subsection{\'Equations différentielles}
% %-------------------------------------------------------------------------------

\paragraph{Solution de l'exercice \ref{exam:DynPop}.}
\begin{enumerate}
  \item $xy$ est la densité de couples possible parmi les individus flottants. Le paramètre $\alpha$ mesure la proportion de couples qui se forment effectivement. $r$ est le taux de naissance instantané de mâles et de femelles (le taux de naissance par couple vaut donc $2r$, à proportion égales entre mâles et femelles). $c$ est le taux de mort des couples.
  \item Selon le système \eqref{eq:DynPop}, on a $\dot x(t) - \dot y(t) = 0$ pour tout $t$, ce qui implique que la différence $x(t) - y(t)$ reste constante au cours du temps, donc que $x(t) - y(t) \equiv S$. \\
  Pour obtenir le système \eqref{eq:DynPop2}, il suffit de remplacer $x$ par $y + S$.
  \item $(y, z) = (0, 0)$ est clairement un point d'équilibre. \\
  Si $z > 0$, alors
  $$
  \dot y(t) = \dot z(t) = 0 (= - \dot z (t)) 
  \qquad \Rightarrow \qquad 
  -\alpha (y^2 + Sy) + r z = -\alpha (y^2 + Sy) + c z^2
  $$
  soit $z = r/c$, qui implique à son tour que
  $$
  y^2 + Sy - r^2/(\alpha c) = 0.
  $$ 
  Le discriminant de cette équation vaut $\Delta = S^2 + 4r^2/(\alpha c) > S^2$. La seule solution positive est donc $y = (\sqrt{\Delta} - S)/2$. Le second équilibre est donc $(y^*, z^*) = ((\sqrt{\Delta} - S)/2, r/c)$.
  \item La matrice jacobienne en $(y, z)$ vaut
  $$
  J_{(y, z)} = \left[\begin{array}{rr}
                      - \alpha (2y + S) & r \\
                      \alpha (2y + S) & -2 cz
                     \end{array}\right].
  $$
  Au point d'équilibre $(y^*, z^*)$, elle vaut donc
  $$
  J_{(y^*, z^*)} = \left[\begin{array}{rr}
                          - \alpha \sqrt{\Delta} & r \\
                          \alpha \sqrt{\Delta} & -2 r
                        \end{array}\right]
  $$
  donc le polynôme caractéristique vaut
  \begin{align*}
    P(\lambda) 
    = \left|\begin{array}{rr}
                - \alpha \sqrt{\Delta} - \lambda & r \\
                \alpha \sqrt{\Delta} & -2 r - \lambda
            \end{array}\right|
    & = (\alpha \sqrt{\Delta} + \lambda) (2 r + \lambda) - \alpha r \sqrt{\Delta} \\
    & = \lambda^2 + (\alpha \sqrt{\Delta} + 2r) \lambda + \alpha r \sqrt{\Delta} \\
    & = \lambda^2 - \tr(J_{(y^*, z^*)}) + \det(J_{(y^*, z^*)}).
  \end{align*}
  On observe d'abord que le discriminant de cette équation vaut $(\alpha^2 \Delta + 4r^2 > 0$ et que la jacobienne admet donc deux valeurs propres réelles distinctes. \\
  On utilisant ensuite le fait que le déterminant de $J_{(y^*, z^*)}$ est le produit de ses valeurs propres et sa trace $J_{(y^*, z^*)}$ est leur somme, on observe que les deux valeurs propres sont de même signe (puisque $\det(J_{(y^*, z^*)}) = \alpha r \sqrt{\Delta} > 0$) et qu'elles sont toutes les deux négatives (puisque $\tr(J_{(y^*, z^*)}) = - \alpha \sqrt{\Delta} - 2r < 0$), d'où on conclue que l'équilibre $(y^*, z^*)$ est stable.
\end{enumerate}

\paragraph{Solution de l'exercice \ref{exam:Proba}.}

\begin{enumerate}
  \item La fonction génératrice est définie par 
  \begin{align*}
    f(s) 
    & = \Esp(s^X) = \sum_{k \geq 0} p_k s^k
    = p_0 + \sum_{k \geq 1} p_k s^k
    = b + (1 - b) (1 - a) \sum_{k \geq 1} a^{k-1} s^k \\
    & = b + \frac{(1 - b) (1 - a)}a \sum_{k \geq 1} (as) s^k .
  \end{align*}
  La formule donnée vient de ce que $\sum_{k \geq 1} (as) s^k = as / (1 - as)$. 
  \item On sait que la probabilité d'extinction $q$ est la plus petite solution de l'équation de point fixe
  $$
  f(s) = s
  \qquad \Leftrightarrow \qquad
  a s^2 - (a+b) s + b = 0.
  $$
  Le discriminant de cette équation vaut $(a+b)^2 - 4ab = (a-b)^2$. Les deux solutions sont
  donc $s = 1$ et $s = b/a$. La probabilité d'extinction $q$ vaut donc $b/a$ si $b < a$ et 1 sinon.
  \item La colonisation réussit si la descendance d'au moins un individu fondateur ne s'éteint pas, ce qui se produit avec probabilité $1 - q^N$. $p$ est donc nulle pour tout $N$ si $q = 1$ et non nulle ssi $q < 1$, c'est à dire si $b < a$.
  \item $1 - q^N$ est la probabilité de colonisation conditionnelle à l'effectif initial $N$. La probabilité de colonisation vaut donc
  $$
  p = \Esp(1 - q^N) = 1 - \Esp(q^N) = 1 - g(q).
  $$
  \item Si $N = 1$, on a immédiatement $p = 1 - q$. Si $N$ est distribué comme $X$, alors $g = f$ et donc
  $$
  p = 1 - g(q) = 1 - f(q) = 1 - q
  $$
  puisque $q$ est un point fixe de $f$ ($f(q) = q$). \\
  L'égalité des résultats n'est pas surprenante puisque le second cas correspond au premier décalé d'une génération.
\end{enumerate}


% %-------------------------------------------------------------------------------
% \subsection{Probabilités}
% %-------------------------------------------------------------------------------


%-------------------------------------------------------------------------------
%-------------------------------------------------------------------------------
\end{document}
%-------------------------------------------------------------------------------
%-------------------------------------------------------------------------------


