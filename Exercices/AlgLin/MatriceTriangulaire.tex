%-------------------------------------------------------------------------------
\subsubsection{Matrice triangulaire}
%-------------------------------------------------------------------------------

Soit $A = [a_{ij}]_{1 \leq i, j \leq n} \in \Mcal_n(\Rbb)$ une matrice triangulaire supérieure, c'est-à-dire que $a_{ij} = 0$ si $i > j$ : 
$$
A = \left(\begin{array}{cccccc}
            a_{11} & \dots & a_{1i} & \dots & a_{1n} \\
            0 & \ddots & & a_{ij} & \vdots \\
            \vdots & \ddots & a_{ii} & & a_{in} \\
            \vdots & 0 & \ddots & \ddots & \vdots \\
            0 & \dots & \dots & 0 & a_{nn} 
          \end{array}\right)
$$

\begin{enumerate}
  \item Montrer que toutes les valeurs propres de $A$ sont réelles.
  \solution{Puisque le déterminant d'une matrice diagonale est égal au produit de ses termes diagonaux, le polynôme caractéristique de $A$ est
  $$
  P_A(\lambda) 
  = \left|\begin{array}{cccccc}
            a_{11} - \lambda & \dots & a_{1i} & \dots & a_{1n} \\
            0 & \ddots & & a_{ij} & \vdots \\
            \vdots & \ddots & a_{ii} - \lambda & & a_{in} \\
            \vdots & 0 & \ddots & \ddots & \vdots \\
            0 & \dots & \dots & 0 & a_{nn}- \lambda  
          \end{array}\right|
  = \prod_{i=1}^n (a_{ii} - \lambda)                       
  $$
  dont les racines sont les $a_{ii}$, qui sont tous réels.
  }
  \item Donner une condition suffisante simple pour que $A$ soit diagonalisable.
  \solution{Il suffit que les termes diagonaux $(a_{ii})_{1 \leq i \leq n}$ soient tous distincts.  }
\end{enumerate}
