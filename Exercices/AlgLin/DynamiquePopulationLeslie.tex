%-------------------------------------------------------------------------------
\subsubsection{Dynamique d'une population de Leslie \todo{}}
%-------------------------------------------------------------------------------

On considère une population structurée en $n$ classes d'âge. Pour $1 \leq i \leq n$, on note $x_{ti}$ le nombre d'individus de la classe $i$ à la génération $t$ et $x_t = [x_{t1} \dots x_{tn}]^\top$ le vecteur décrivant l'ensemble de la population à cette même génération. On suppose que l'évolution de cette population est régit par la récurrence
$$
x_{t+1} = A x_t
$$
où $A$ est la matrice de Leslie
$$
A = \left[\begin{array}{cccccc}
            f_1 & f_2 & \cdots  & \cdots & f_n \\
            s_1 & 0 & \cdots  & \cdots & 0 \\
            0 & \ddots  & \ddots & & \vdots \\
            \vdots & \ddots & \ddots & \ddots & \vdots \\
            0 & \cdots & 0 & s_{n-1} & 0 \\
          \end{array}\right]
$$
où tous les coefficients $f_i$ et $s_i$ sont supposés strictement positifs. On note de plus
$$
\ell_1 = 1 \qquad \text{et} \qquad 
\ell_k = \prod_{i=1}^{k-1} s_i \quad \text{pour $2 \leq k \leq n$}.
$$

\begin{enumerate}
  \item Interprêter les coefficients $f_i$ et $s_i$.
  \solution{$f_i$ est le taux de fertilité de la classe $i$ (qui alimente la classe 1). $S_i$ est le taux de survie de la classe $i$ (qui alimente la classe $i+1$).}
  %
  \item Soient $B_{k1} \in \Mcal_{k-1}$, $B_{k2} \in \Mcal_{n-k}$ définies par :
  $$
  B_{k1} = \left[\begin{array}{cccc}
            s_1 & -\lambda & &  \\
            & \ddots & \ddots & \\
            & & \ddots & -\lambda \\
            & & & s_{k-1}
          \end{array}\right], \qquad
  B_{k2} = \left[\begin{array}{cccc}
            -\lambda & & & \\
            s_{k+1} & \ddots & & \\
            & \ddots & \ddots & \\
            & & s_{n-1} & -\lambda
          \end{array}\right].
  $$
  En notant $0_{p,q}$ la matrice $p \times q$ dont tous les éléments sont nuls, calculer le déterminant de la matrice $B_k \in \Mcal_{n-1}$ :
  $$
  B_k = \left[\begin{array}{cc}
            B_{k1} & 0_{k-1, n-k} \\
            0_{n-k, k-1} & B_{k2}
          \end{array}\right].
  $$
  \solution{On utilise le calcul du déterminant par bloc pour obtenir
  $$
  |B| = |B_{k1}| \times |B_{k2}|
  $$
  et on remarque que, puisque $B_{k1}$ et $B_{k2}$ sont respectivement triangulaires supérieure et inférieure, leurs déterminants sont égaux au produit de leurs termes diagonaux, soit
  $$
  |B_{k1}| = \ell_k, \qquad |B_{k2}| = (-\lambda)^{n-k}
  $$
  }
  %
  \item Montrer que le polynôme caractéristique de $A$ est
  $$
  P_A(\lambda) = (f_1 - \lambda) (-\lambda)^{n-1} + \sum_{k=2}^{n} (-1)^{k-1} f_k \ell_k (-\lambda)^{n-k}.
  $$
  \solution{Les termes successifs du développement du déterminant 
  $$
  |A - \lambda I| 
  = \left|\begin{array}{cccccc}
            f_1-\lambda & f_2 & \cdots  & \cdots & f_n \\
            s_1 & -\lambda & 0  & \cdots & 0 \\
            0 & s_2 & \ddots & \ddots & \vdots \\
            \vdots & \vdots & \ddots & \ddots & 0 \\
            0 & \cdots & 0 & s_{n-1} & -\lambda \\
          \end{array}\right|
  $$ 
  par rapport à la première ligne sont
  \begin{align*}
  (f_1 - \lambda) |B_1| & = (f_1-\lambda) (-\lambda^{n-1}), &
  - f_2 |B_2| & = -f_2 s_1 (-\lambda)^{n-2}, \\
  f_3 |B_3| & = f_3 s_1 s_2 (-\lambda)^{n-3}, & 
  -f_4 |B_4| & = -f_4 s_1 s_2 s_3 (-\lambda)^{n-4}, \qquad \dots
  \end{align*}
  }
  %
  \item En déduire que la plus grande valeur propre en module $\lambda_1$ de la matrice $A$ vérifie
  $$
  \sum_{k=1}^n \ell_k f_k \lambda_1^{-k} = 1.
  $$
  \solution{Toutes les valeurs propres, dont $\lambda_1$, sont solutions de $P_A(\lambda) = 0$, soit
  \begin{align*}
  (f_1 - \lambda) (-\lambda)^{n-1} + \sum_{k=2}^{n} (-1)^{k-1} f_k \ell_k (-\lambda)^{n-k} & = 0 \\
  \Leftrightarrow \qquad \sum_{k=1}^{n} (-1)^{k-1} f_k \ell_k (-\lambda)^{n-k} & = (-\lambda)^{n-1} & 
  \Leftrightarrow \qquad \sum_{k=1}^{n} f_k \ell_k \lambda^k & = 1.
  \end{align*}
  }
\end{enumerate}

On note 
$$
a = \sum_{k=1}^n \ell_k \lambda_1^{-k}, \qquad
b = \sum_{k=1}^n k \ell_k f_k \lambda_1^{-k}.
$$

\begin{enumerate}
  \setcounter{enumi}{4}
  \item Montrer que le vecteur $v$ de coordonnées
  $$
  v_k = \frac1a \ell_k \lambda_1^{-k}, \qquad 1 \leq k \leq n,
  $$
  est un vecteur propre à droite de $A$ associé à la valeur propre $\lambda_1$.
  \solution{Soit le vecteur $w = Av$. Ses coordonnées sont
  \begin{align*}
    w_1 & = \sum_{k=1}^n f_k v_k = \frac1a \sum_{k=1}^n f_k \ell_k \lambda_1^{-k} = \frac1a = \lambda_1 v_1, \\
    w_k & = s_{k-1} v_{k-1} = \frac1a s_{k-1} \ell_{k-1} \lambda_1^{-k+1}  = \frac1a \ell_k \lambda_1^{-k+1} = \lambda_1 v_k, \qquad \text{pour $2 \leq k \leq n$}.
  \end{align*}
  }
  \item Montrer que le vecteur $u$ de coordonnées
  $$
  u_k = \frac1{b v_k} \sum_{j=k}^n \ell_j f_j \lambda_1^{-j}, \qquad 1 \leq k \leq n,
  $$
  est un vecteur propre à gauche de $A$ associé à la valeur propre $\lambda_1$.
  \solution{Soit le vecteur $w^\top = u^\top A$. Ses coordonnées sont
  \begin{align*}
    w_k & = f_k u_1 + s_k u_{k+1} 
    = \frac{a \lambda_1^k}{b \ell_k} \sum_{j=k}^n \ell_j f_j \lambda_1^{-j}, \qquad \text{pour $1 \leq k \leq n-1$}, \\
    w_n & = f_n u_1
    = f_n \frac{a \lambda_1}{b} \sum_{j=1}^n \ell_j f_j \lambda_1^{-j}
  \end{align*}
  \todo{pas fini}
  }
\end{enumerate}

