\progres{Rappel cours 2 :
\begin{enumerate}
  \item Fin algèbre linéaire: 
  \begin{itemize}
    \item Matrices définies positives ($A \succcurlyeq 0$)
    \item Matrices positives ($A \geq 0$) / theorème de Perron-Frobenius
  \end{itemize}
  \item TD : 
  \begin{itemize}
    \item Matrice de covariance $\succcurlyeq 0$ et ACP
    \item Rappel : $AB = I \Rightarrow BA = I$
  \end{itemize}
\end{enumerate}
Programme cours 3 :
\begin{enumerate}
  \item Normes = mesure d'une approximation
  \begin{itemize}
    \item Normes classiques
    \item \'Equivalence
  \end{itemize}
  \item Différentiabilité = possibilité d'une approximation linéaire
  \begin{itemize}
    \item Application linéaire tangente
    \item Matrice jacobienne
    \item Composition
  \end{itemize}
\end{enumerate}
}


%-------------------------------------------------------------------------------
%-------------------------------------------------------------------------------
\section{Préliminaire : normes} \label{sec:Multivar-Norme}
%-------------------------------------------------------------------------------
%-------------------------------------------------------------------------------

%-------------------------------------------------------------------------------
%-------------------------------------------------------------------------------
\subsection{Quelques normes classiques} 
%-------------------------------------------------------------------------------

\begin{definition*}
  Soit $\Ecal$ un espace vectoriel. Une application $\|\cdot\|$ de $\Ecal$ dans $\Rbb^+$ est une norme si
  \begin{enumerate}[(i)]
    \item $\{\|x\| = 0 \Leftrightarrow x = 0]\}$ ;
    \item $\forall x \in \Ecal, \forall \lambda \in \Rbb: \; \|\lambda x\| = |\lambda| \|x\|$ ;
    \item $\forall x, y \in \Ecal: \; \|x+y\| \leq \|x\| + \|y\|$ (inégalité triangulaire);
  \end{enumerate}
\end{definition*}

\remarks
\begin{itemize}
  \item Une norme induit une distance:
  $$
  d(x, y) = \|x - y\|.
  $$
  \item Une norme induit une définition de la convergence : $(u_n)_{n \geq 0}$ converge vers $\ell$ ssi
  $$
  \lim_{n \rightarrow \infty} \|u_n - \ell\| = 0;
  $$
  On dit alors que $(u_n)_{n \geq 0}$ converge vers $\ell$ {\em au sens de $\|\cdot\|$}.
\end{itemize}

%-------------------------------------------------------------------------------
\paragraph*{Trois normes classiques: $\|\cdot\|_1$, $\|\cdot\|_2$ et $\|\cdot\|_\infty$.}
\begin{description}
 \item[Vecteurs :] pour $x = [x_1 \dots x_n]^\top \in \Rbb^n$
 $$
 \|x\|_1 = \sum_{i=1}^n |x_i|, \qquad 
 \|x\|_2 = \sqrt{\sum_{i=1}^n x_i^2}, \qquad 
 \|x\|_\infty = \max_{i = 1 \dots n} |x_i|. 
 $$
 \item[Matrices:] pour $A = [a_{ij}] \in \Mcal_{n, k}(\Rbb)$
 $$
 \|A\|_1 = \sum_{i=1}^n \sum_{j=1}^k |a_{ij}|, \qquad 
 \|A\|_2 = \sqrt{\sum_{i=1}^n \sum_{j=1}^k a_{ij}^2}, \qquad 
 \|A\|_\infty = \max_{i = 1 \dots n, j = 1 \dots k} |a_{ij}|. 
 $$
 \item[Fonctions:] pour $f : [a, b] \mapsto \Rbb$ continue
 $$
 \|f\|_1 = \int_a^b |f(x)| \d x, \qquad 
 \|f\|_2 = \sqrt{\int_a^b f(x)^2 \d x}, \qquad 
 \|f\|_\infty = \sup_{x \in [a, b]} |f(x)|.
 $$
\end{description}

\dessin{Tracer de la boule unité pour $\|\cdot\|_1$, $\|\cdot\|_2$ et $\|\cdot\|_\infty$ dans $\Rbb^2$.}

\remark
On montre facilement que $\|\cdot\|_1$, $\|\cdot\|_2$ et $\|\cdot\|_\infty$ sont des normes pour $\Rbb^n$. En effet, les conditions ($i$) et ($ii$) sont vlairement vérifiées par ces trois normes et pour ($iii$), on a :
\begin{description}
  \item[$\|\cdot\|_1:$] 
  $$\|x+y\|_1 = \sum_i |x_i + y_i| \leq \sum_i |x_i| + |y_i| =  \|x\|_1 + \|y\|_1.$$
  \item[$\|\cdot\|_\infty:$] 
  $$\|x+y\|_\infty = \max_i |x_i + y_i| \leq \max_i |x_i| + \max_i |y_i| =  \|x\|_\infty + \|y\|_\infty.$$
  \item[$\|\cdot\|_2:$] 
  $$\|x+y\|^2_2 = \sum_i (x_i + y_i)^2 = \sum_i x_i^2 + \sum_i y_i^2 + 2 \sum_i x_i y_i,$$ 
  or, par Cauchy-Schwarz, on sait que $|x^\top y| < \|x\|_2 \|y\|_2$, donc 
  $$\|x+y\|^2_2 \leq \|x\|_2^2 + \|y\|_2^2 + 2 \|x\|_2 \|y\|_2 = (\|x\|_2 + \|y\|_2)^2.$$
\end{description}

%-------------------------------------------------------------------------------
%-------------------------------------------------------------------------------
\subsection{\'Equivalence} 
%-------------------------------------------------------------------------------

\begin{theorem} \label{thm:equivalenceNormes}
  Les normes définies sur un espace $\Ecal$ de dimension finie sont toutes équivalentes. Plus précisément : soit $\|\cdot\|_a$ et $\|\cdot\|_b$ deux normes pour $\Ecal$, il existe 2 constantes positives $c_1$ et $c_2$, telles que, pour tout $x \in \Ecal$ :
  $$
  \|x\|_a \leq c_1 \|x\|_b, \qquad
  \|x\|_b \leq c_2 \|x\|_a.
  $$
\end{theorem}

Ce résultat n'est pas démontré ici. 

\remark
Le théorème \ref{thm:equivalenceNormes} fournit un encadrement de chaque norme par l'autre : 
$$
\forall x \in \Ecal: \qquad
\frac1c_2 \|x\|_b \leq \|x\|_a \leq c_1 \|x\|_b, \qquad
\frac1c_1 \|x\|_a \leq \|x\|_b \leq c_2 \|x\|_a.
$$

\exemple{
  \begin{align*}
    \|x\|_2 & = \sqrt{\sum_i x_i^2}  \leq \sqrt{n \max_i x_i^2} = \sqrt{n} \max_i |x_i| = \sqrt{n} \|x\|_\infty, \\
    \|x\|_\infty & = \max_i |x_i| = \sqrt{\max_i x_i^2} \leq \sqrt{\sum_i x_i^2} = \|x\|_2.
  \end{align*}
}

\remark
Le théorème \ref{thm:equivalenceNormes} assure aussi l'équivalence des convergences : si la suite $(u_n)_{n \geq 0}$ converge vers $\ell$ au sens de la norme $\|\cdot\|_a$ :
$$
\lim_{n \rightarrow \infty} \|u_n - \ell\|_a = 0
$$
alors elle converge aussi au sens de la norme $\|\cdot\|_b$ puisque
$$
0 = \frac1{c_1} \lim_{n \rightarrow \infty} \|u_n - \ell\|_a
\leq \lim_{n \rightarrow \infty} \|u_n - \ell\|_b \leq
c_2 \lim_{n \rightarrow \infty} \|u_n - \ell\|_a = 0.
$$

\begin{definition}[Convergence coordonnée par coordonnée.]
  Une suite $(u(n))_{n \geq 0}$ d'élément de $\Rbb^k$ converge coordonnée par coordonnée vers $\ell \in \Rbb^k$ si, pour chaque $1 \leq i \leq k$:
  $$
  \lim_{n \rightarrow \infty} u_i(n) = \ell_i.
  $$
\end{definition}

\begin{proposition} \label{prop:convergenceParCoordonnee}
  La convergence coordonnée par coordonnée est équivalente à la convergence de la suite $(u(n))_{n \geq 0}$ vers $\ell$ au sens d'une norme $\|\cdot\|$ quelconque.
\end{proposition}

\begin{lemma}[Inversion de la limite et du maximum] \label{lem:inversionLimiteMaximum}
  Soient $k$ suites réelles $v_1, \dots v_k$, on a
  $$
  \lim_{n \rightarrow \infty} \max_{i=1 \dots k} v_i(n) = 0
  \qquad \Leftrightarrow \qquad
  \max_{i=1 \dots k} \lim_{n \rightarrow \infty} v_i(n) = 0.
  $$
\end{lemma}

\parSR{Démonstration du lemme \ref{lem:inversionLimiteMaximum}.}
\begin{description}
  \item[$\Rightarrow$:] $\lim_{n \rightarrow \infty} \max_{i=1 \dots k} v_i(n) = 0$
  \begin{align*}
    \Rightarrow \quad & \{\forall \epsilon, \exists N_\epsilon: n > N_\epsilon \Rightarrow |\max_i v_i(n)| = \max_i |v_i(n)| < \epsilon\} \\
    \Rightarrow \quad & \{\forall \epsilon, \exists N_\epsilon: n > N_\epsilon \Rightarrow \forall i, |v_i(n)| < \epsilon\} \\
    \Rightarrow \quad & \{\forall i, \forall \epsilon, \exists N_{\epsilon, i}(\equiv N_\epsilon): n > N_{\epsilon, i} \Rightarrow |v_i(n)| < \epsilon\} \\
    \Rightarrow \quad & \forall i, \lim_{n \rightarrow \infty} v_i(n) = 0 \\
    \Rightarrow \quad & \max_i \lim_{n \rightarrow \infty} v_i(n) = 0.
  \end{align*}
  \item[$\Leftarrow$:] $\max_{i=1 \dots k} \lim_{n \rightarrow \infty} v_i(n) = 0$
  \begin{align*}
    \Rightarrow \quad & \forall i, \lim_{n \rightarrow \infty} v_i(n) = 0 \\
    \Rightarrow \quad & \{\forall i, \forall \epsilon, \exists N_{\epsilon, i}: n > N_{\epsilon, i} \Rightarrow |v_i(n)| < \epsilon\}\\
    \Rightarrow \quad & \{\forall \epsilon, \exists N_\epsilon (= \max_i N_{\epsilon, i}): \forall i, n > N_\epsilon \Rightarrow |v_i(n)| < \epsilon\}\\
    \Rightarrow \quad & \{\forall \epsilon, \exists N_\epsilon: n > N_\epsilon \Rightarrow \max_i |v_i(n)| < \epsilon\}\\
    \Rightarrow \quad & \lim_{n \rightarrow \infty} \max_i v_i(n) = 0
  \end{align*}
\end{description}
\eproof

\parSR{Démonstration de la proposition \ref{prop:convergenceParCoordonnee}.}
Toutes les normes étant équivalentes dans $\Rbb^k$, nous pouvons démontrer chaque implication avec la norme de notre choix.
\begin{description}
  \item[$\Rightarrow$:]
    \begin{align*}
%       \forall 1 \leq i \leq k: \lim_{n \rightarrow \infty} u_i(n) = \ell_i
%       & \qquad \Rightarrow \qquad & & 
      \forall 1 \leq i \leq k: \lim_{n \rightarrow \infty} |u_i(n) - \ell_i| & = 0 
      & \Rightarrow & & \sum_{i=1}^k \lim_{n \rightarrow \infty} |u_i(n) - \ell_i| & = 0 \\
      & & \Rightarrow & & \lim_{n \rightarrow \infty} \sum_{i=1}^k |u_i(n) - \ell_i| & = 0 \\
      & & \Rightarrow & &  \lim_{n \rightarrow \infty} \|u - \ell\|_1 & = 0. 
    \end{align*}
  \item[$\Leftarrow$:]
  \begin{align*}
    \lim_{n \rightarrow \infty} \|u - \ell\|_\infty & = 0
    & \Rightarrow & & \lim_{n \rightarrow \infty} \max_{i=1 \dots k} |u_i(n) - \ell_i| & = 0 \\
    & & \Rightarrow & & \max_{i=1 \dots k} \lim_{n \rightarrow \infty} |u_i(n) - \ell_i| & = 0 \qquad \text{(d'après le lemme)}\\
    & & \Rightarrow & & \forall 1 \leq i \leq k: \lim_{n \rightarrow \infty} |u_i(n) - \ell_i| & = 0.
  \end{align*}
\end{description} 
\eproof

\begin{proposition} \label{prop:normeProduitMatriceVecteur}
  Soit $A \in \Mcal_{n, k}$, $x \in \Rbb^k$ et $y = Ax \in \Rbb^n$, montrer que 
  $$
  \|y\|_\infty \leq \|A\|_\infty \|x\|_1, \qquad
  \|y\|_1 \leq \|A\|_1 \|x\|_\infty.
  $$
\end{proposition}

\parSR{Démonstration de la proposition \ref{prop:normeProduitMatriceVecteur}.}
On a pour tout $1 \leq j \leq n$ :
$$
|y_j| 
\; = \; \left|\sum_{i=1}^k a_{ij} x_i \right| 
\; \leq \; \sum_{i=1}^k |a_{ij} x_i| 
\; = \; \sum_{i=1}^k |a_{ij}| \; |x_i|
$$
On a donc
\begin{itemize}
  \item d'une part
  $$
  \forall 1 \leq j \leq n: \qquad |y_j| 
  \; \leq \; \max_{i=1 \dots k} |a_{ij}| \times \sum_{i=1}^k |x_i| 
  \; \leq \; \|A\|_\infty \; \|x\|_1
  $$
  donc $\|y\|_\infty \leq \|A\|_\infty \|x\|_1$,
  \item et d'autre part
  $$
  \forall 1 \leq j \leq n: |y_j| 
  \leq \max_{i=1 \dots k} |x_i| \times \sum_{i=1}^k |a_{ij}| = \|x\|_\infty \sum_{i=1}^k |a_{ij}| 
  $$
  qui implique que
  $$
  \sum_{j=1}^n |y_j| 
  \leq \|x\|_\infty \sum_{i=1}^k \sum_{j=1}^n |a_{ij}| = \|x\|_\infty \|A\|_1,
  $$
  c'est-à-dire $\|y\|_1 \leq \|A\|_1 \|x\|_\infty$.
\end{itemize}
\eproof

