%-------------------------------------------------------------------------------
\subsubsection{Loi marginale d'une chaîne de Markov \todo{corrigé}}
%-------------------------------------------------------------------------------

On considère une chaîne de Markov constituée d'un couple $((X_n, Y_n))_{n \geq 0}$ prenant ses valeurs dans $\{0, 1\} \times \{0, 1 \}$. On suppose que 
\begin{itemize}
 \item $X_n$ change d'état ($0 \to 1$ où $1 \to 0$) avec probabilité $a$ et reste dans le même état avec probabilité $1-a$, soit :
 $$
 \Pr\{X_{n+1} = X_n\} = 1-a, 
 $$ 
 \item si $X_n$ vaut $0$, $Y_n$ change d'état avec probabilité $b$ (et reste dans le même état avec probabilité $1 - b$), soit :
 $$
 \Pr\{Y_{n+1} = Y_n \mid X_n = 0\} = 1-b, 
 $$ 
 \item mais que si $X_n$ vaut $1$, alors $Y_n$ reste dans le même état ($Y_{n+1} = Y_n$), soit :
 $$
 \Pr\{Y_{n+1} = Y_n \mid X_n = 1\} = 1.
 $$
\end{itemize}
Intuitivement, $Y_n$ peut varier lorsque $X_n$ vaut 0, alors qu'elle est figée quand $X_n$ vaut 1.
La matrice de transition de la chaîne $((X_n, Y_n))_{n \geq 0}$ est donc
$$
P = 
\left(
\begin{array}{c|cccc}
  (x, y) & (0, 0) & (0, 1) & (1, 0) & (1, 1) \\
  \hline
  (0, 0) & (1-a)(1-b) & (1-a)b & a(1-b) & ab \\
  (0, 1) & (1-a)b & (1-a)(1-b) & ab & a(1-b) \\
  (1, 0) & a & 0 & 1-a & 0 \\
  (1, 1) & 0 & a & 0 & 1-a \\
\end{array}
\right).
$$

\bigskip
\paragraph{Distribution stationnaire.}
\begin{enumerate}
  \item Montrer que l'unique distribution stationnaire de cette chaîne de Markov est la distribution uniforme sur $\{0, 1\} \times \{0, 1 \}$, qu'on notera $\nu$.
  \solution{
    La somme de chacune des colonnes de $P$ vaut 1, ce qui assure que $\nu^\top P = \nu^\top$. 
    Comme, de plus, la chaîne est irréductible et apériodique, elle admet une unique distribution stationnaire, quie est $\nu$.
  }
  \item En supposant que $(X_0, Y_0)$ est tiré uniformément : $(X_0, Y_0) \sim \nu$, calculer $\Pr\{X_n = 1, Y_n = 1\}$, $\Pr\{X_n = 1\}$ et $\Pr\{Y_n = 1\}$ pour $n \geq 0$.
  \solution{
  Puisque $\nu$ est la distribution stationnaire, on a $(X_n, Y_n) \sim \nu$ pour tout $n \geq 0$, et les couples $(x, y)$ sont tous équiprobables, soit
  $$
  \Pr\{X_n = 1, Y_n = 1\} = \frac14, \qquad
  \Pr\{X_n = 1\} = \frac12, \qquad
  \Pr\{Y_n = 1\} = \frac12.
  $$}
\end{enumerate}

\bigskip
\paragraph{Loi marginale du processus $(Y_n)_{n \geq 0}$.}
On suppose maintenant que $(X_0, Y_0)$ est tiré uniformément et on s'intéresse à la loi du processus constitué par la seule seconde coordonnée $Y$. 
\begin{enumerate}
  \setcounter{enumi}{2}
  \item Montrer que 
  \begin{equation} \label{eq:margMC-pcond1}
  \Pr\{Y_{n+1} = 1 \mid Y_n = 1\} = 1 - \frac{b}2.
  \end{equation}
  \solution{Puisque la chaîne est en régime stationnaire, $X_n$ vaut 0 ou 1 avec probabilité $1/2$ et
  \begin{itemize}
    \item si $X_n = 0$, la transition $Y_n = 1 \to Y_{n+1} = 1$ se fait avec probabilité $1-b$ alors que
    \item si $X_n = 1$,  la transition $Y_n = 1 \to Y_{n+1} = 1$ est certaine (probabilité $1$), 
  \end{itemize}
  soit au total
  $$
  \Pr\{Y_{n+1} = 1 \mid Y_n = 1\} 
  = \frac{1-b}2 + \frac12
  = 1 - \frac{b}2.
  $$}
  \item Calculer $\Pr\{Y_{n+1} = 1, Y_n = 1, Y_{n-1} = 1\}$.
  \solution{On envisage toutes les combinaisons $(x, x')$ possibles pour le couple $(X_{n-1}, X_n)$ et, pour chacune d'elle, on on utilise le caractère markovien du couple $(X_n, Y_n)$ pour décomposer
  \begin{align*}
    & \Pr\{Y_{n+1} = 1, Y_n = 1, X_n=x', Y_{n-1} = 1, X_{n-1}=x\} \\
    = & \Pr\{Y_{n+1} = 1 \mid Y_n = 1, X_n=x'\}
    \times \Pr\{Y_n = 1, X_n=x' \mid Y_{n-1} = 1, X_{n-1}=x\} \\
    & \times \Pr\{Y_{n-1} = 1, X_{n-1}=x\}.
  \end{align*}
  Comme la chaîne est en régime stationnaire, la probabilité $\Pr\{Y_{n-1} = 1, X_{n-1}=x\}$ vaut 1/4 pour tout $x$.Il reste à calculer le produit
  $$
  \Pr\{Y_{n+1} = 1 \mid Y_n = 1, X_n=x'\}
  \times \Pr\{Y_n = 1, X_n=x' \mid Y_{n-1} = 1, X_{n-1}=x\}  
  $$
  pour chaque couple $(x, x')$ :
  \begin{align*}
    (x, x')=(0, 0): & \quad (1-a)(1-b) \times (1-b), &
    (x, x')=(0, 1): & \quad a(1-b) \times 1, \\
    (x, x')=(1, 0): & \quad a \times (1-b) &
    (x, x')=(1, 1): & \quad (1-a) \times 1, 
  \end{align*}
  soit au total
  \begin{align*}
    \Pr\{Y_{n+1} = 1, Y_n = 1, Y_{n-1} = 1\}
    & = \frac14 \left[(1-a)(1-b)^2 + 2a(1-b) + (1-a)\right] \\
    & = \frac14 \left(1 - 2b + b^2 - ab^2\right).
  \end{align*}
  }
  \item En déduire que
  $$
  \Pr\{Y_{n+1} = 1 \mid Y_{n-1} = 1, Y_n = 1\}
  = \frac{1 - 2b + b^2 - ab^2}{2 - b}.
  $$
  \solution{
    D'après l'équation \eqref{eq:margMC-pcond1}, on a 
    $$
    \Pr\{Y_{n-1} = 1, Y_n = 1\} = 
    \Pr\{Y_{n-1} = 1\} \times \Pr\{Y_n = 1 \mid Y_{n-1} = 1\} = \frac14(2 - b),
    $$
    d'où on tire
    $$
    \Pr\{Y_{n+1} = 1 \mid Y_{n-1} = 1, Y_n = 1\}
    = \frac{\Pr\{Y_{n+1} = 1, Y_n = 1, Y_{n-1} = 1\}}{\Pr\{Y_{n+1} = 1, Y_n = 1\}} 
%     = \frac{\left(1 - b + a^2 - ab\right)/4}{(2 - b)/4} 
    = \frac{1 - 2b + b^2 - ab^2}{2 - b}. 
  $$
}
  \item Le processus $(Y_n)_{n \geq 0}$ constitue-t-il une chaîne de Markov ?
  \solution{Non, puisque $\Pr\{Y_{n+1} = 1 \mid Y_{n-1} = 1, Y_n = 1\} \neq     \Pr\{Y_{n+1} = 1 \mid Y_n = 1\}$ (la première dépend de $a$ et pas la seconde). \\
  Ainsi le couple $((X_n, Y_n))_{n \geq 0}$ constitue une chaîne de Markov, mais pas $(Y_n)_{n \geq 0}$ seule. \\
  (Dans le cas présent, on peut montrer que, par contre, $(X_n)_{n \geq 0}$ constitue une chaîne de Markov.)
}
\end{enumerate}
