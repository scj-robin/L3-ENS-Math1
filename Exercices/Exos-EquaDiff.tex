%-------------------------------------------------------------------------------
\subsection{Calcul différentiel}
%-------------------------------------------------------------------------------

% \todo{Voir L3 Bio SU : TD2, exercice 1}

\input{EquaDiff/SystDynDim1-L3bioSU}

\input{EquaDiff/}

\input{EquaDiff/}

\input{EquaDiff/}

\input{EquaDiff/}


%-------------------------------------------------------------------------------
\subsection{Systèmes dynamiques en dimension 2}
%-------------------------------------------------------------------------------

% %-------------------------------------------------------------------------------
% \paragraph{Système dynamique linéaire.} 
% On considère le système dynamique suivant
% $$
% \left\{\begin{array}{rcl}
%         \dot x & = & -a_1 x + b_1 y + c_1 \\ 
%         \dot y & = & -a_2 x + b_2 y + c_2
%         \end{array}\right.
% $$
% où tous les coefficients constants sont strictement positifs.
% \begin{enumerate}
%   \item À quelle condition y a-t-il un unique équilibre ? Lorsque c’est le cas, à quelle condition est-il stable ?
%   \item Lorsqu’il n’existe pas d’équilibre unique, représenter les deux isoclines (c'est à dire les ensemble de points ou s'annule $\dot x$ d'une part et $\dot y$ d'autre part). Y a-t-il une infinité d’équilibres ou aucun équilibre ?
%   \item Représenter les orbites dans le plan de phase.
% \end{enumerate}
% 
% \solution{\todo{}}

% Cycle limite \todo{Voir exemple 1, p 195, Perko, 2001}

