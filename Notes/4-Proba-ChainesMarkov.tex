%-------------------------------------------------------------------------------
%-------------------------------------------------------------------------------
\section{Chaînes de Markov}  \label{sec:Proba-Markov}
%-------------------------------------------------------------------------------

\newcommand{\cM}{{chaîne de Markov}\xspace}
\newcommand{\CM}{{Chaîne de Markov}\xspace}
\newcommand{\per}{\text{per}}

% Chaînes de Markov : classes de communication, irréductibilité, états transients,
% récurrents, absorbants. 
% 
% Période d’une chaîne de Markov irréductible.

% \todo{Voir 1-AlgLin-MatricesPositives / Chaîne de Markov}

\paragraph*{Objectif.}
Etudier une suite de variable aléatoires $(X_0, X_1, \dots X_n, \dots)$ non indépendantes. L'indice $n$ se réfère souvent à un temps et une telle suite est alors appelées {\em processus stochastique} ou {\em chaîne}. On note alors $X = (X_t)_{t \geq 0}$.

On suppose ici que les variables $X_n$ sont toutes à valeurs dans un même ensemble $\Ecal$ fini ou dénombrable.

%-------------------------------------------------------------------------------
\subsection{Définitions et premières proriétés}  
%-------------------------------------------------------------------------------

%-------------------------------------------------------------------------------
\subsubsection{Hypothèse de Markov}

\begin{definition*}[\CM]
  $X$ est une \cM si
  $$
  \Pr\{X_{n+1} = x_{n+1} \mid X_0 = x_0, X_1 = x_1, \dots X_n = x_n\} 
  = 
  \Pr\{X_{n+1} = x_{n+1} \mid X_n = x_n\}.
  $$
\end{definition*}

\begin{definition*}[\CM homogène]
  Une \cM $X$ est homogène si
  $$
  \forall n \geq 1: \qquad 
  \Pr\{X_{n+1} = j \mid X_n = i\} = p_{ij} 
  \qquad (\text{indépendant de $n$}).
  $$
\end{definition*}

\begin{exercise*}
  Montrer que si $X$ est un \cM, 
  $$
  \Pr\{X_{n+1} = k \mid X_0 = i, X_n = j\} 
  = 
  \Pr\{X_{n+1} = k \mid X_n = j\}.
  $$
\end{exercise*}

\solution{
  On utilise le théorème des probabilités totales pour sommer sur toutes les valeurs intermédiaires
  \begin{align*}
    \Pr\{X_0 = i, X_n = j, X_{n+1} = k\}
    & = \sum_{x_1, x_2, \dots x_{n-1}}
    \Pr\{X_0 = i, X_1 = x_1, \dots X_{n-1} = x_{n-1}, X_n = j, X_{n+1} = k\} \\
    & = \sum_{x_1, x_2, \dots x_{n-1}}
    \Pr\{X_0 = i, X_1 = x_1, \dots X_{n-1} = x_{n-1}, X_n = j, X_{n+1} = k\} \\
    & = \sum_{x_1, x_2, \dots x_{n-1}}
    \Pr\{X_{n+1} = k \mid X_0 = i, X_1 = x_1, \dots X_{n-1} = x_{n-1}, X_n = j\} \\
    & \qquad \times \Pr\{X_0 = i, X_1 = x_1, \dots X_{n-1} = x_{n-1}, X_n = j\} \\
    & = \Pr\{X_{n+1} = k \mid X_n = j\} \\
    & \qquad \times \sum_{x_1, x_2, \dots x_{n-1}} \Pr\{X_0 = i, X_1 = x_1, \dots X_{n-1} = x_{n-1}, X_n = j\} \\
    & = \Pr\{X_{n+1} = k \mid X_n = j\} \Pr\{X_0 = i, X_n = j\}.
  \end{align*}
  Il suffit alors de diviser par $\Pr\{X_0 = i, X_n = j\}$ des deux côtés. 
}

%-------------------------------------------------------------------------------
\subsubsection{Matrice de transition}

\begin{definition*}[Probabilité et matrice de transition]
  Soit $X$ un \cM homogène, on appelle probabilités de transition les probabilités :
  $$
  p_{ij} = \Pr\{X_1 = j \mid X_0 = i\}, \qquad \forall i, j \in \Ecal.
  $$
  On appelle de plus matrice de transition la matrice $P$ de terme général $p_{ij}$:
  $$
  P = [p_{ij}].
  $$
\end{definition*}

\remark
\begin{enumerate}
  \item On utilise la notation matricielle même si $\Ecal$ est infini (mais dénombrable).
  \item On rappelle que $P$ est une {\em matrice stochastique} : 
  $$
  \forall i \in \Ecal: \qquad \sum_{j \in \Ecal} p_{ij} = 1.
  $$
  En conséquence, chaque ligne $(p_{ij})_{j \in \Ecal}$ de $P$ définie une loi de probabilité.
\end{enumerate}

\begin{definition*}[Transition de $n$ étapes]
  Soit $X$ un \cM homogène, on appelle probabilités de transition en $n$ étapes les probabilités :
  $$
  p_{ij}(n) = \Pr\{X_n = j \mid X_0 = i\}, \qquad \forall i, j \in \Ecal.  
  $$
\end{definition*}

\begin{proposition*}
  Soit $X$ une \cM homogène, les probabilités $p_{ij}(n)$ sont les termes généraux de la matrice $P^n$ : 
  $$
  p_{ij}(n) = [P^n]_{ij}.
  $$
\end{proposition*}

\proof
Le démonstration se fait par récurrence en supposant la propriété vraie au rang $n$. Il suffit alors de sommer sur tous les états précédents possibles : 
\begin{align*}
  p_{ik}(n+1) 
  & = \Pr\{X_{n+1} = k \mid X_0=i\} \\
  & = \sum_{k \in \Ecal} \Pr\{X_{n+1} = k, X_n = j \mid X_0=i\} \\
  & = \sum_{k \in \Ecal} \Pr\{X_n = j \mid X_0=i\} \; \Pr\{X_{n+1} = k \mid X_n = j, X_0=i\} \\
  & = \sum_{k \in \Ecal} \Pr\{X_n = j \mid X_0=i\} \; \Pr\{X_{n+1} = k \mid X_n = j\} \qquad (\text{cf exercice précédent})\\
  & = \sum_{k \in \Ecal} p_{ij}(n) \; p_{jk} = \sum_{k \in \Ecal} [P^n]_{ij} p_{jk} = [P^{n+1}]_{ik}.
\end{align*}
\eproof

\remark
La loi d'une \cM homogène $X$ est donc entièrement spécifiée 
\begin{itemize}
  \item sa matrice de transition $P$ et
  \item la distribution de sa valeur initiale $X_0$.
\end{itemize}

\paragraph*{Notation.}
Dans la suite on notera $\Pr_\mu$ la loi d'une \cM $X$
\begin{itemize}
  \item de distribution initiale $\mu$: $X_0 \sim \mu$ et
  \item de matrice de transition $P$.
\end{itemize}
On notera notamment $\Pr_i$ la loi de $X$ si $x_0 = i$, c'est à dire
$$
\Pr_\mu = \sum_{i \in \Ecal} \mu_i \Pr_i.
$$

\remark
On peut identifier la distribution $\mu$ à un vecteur ligne $\mu^\top$ (éventuellement infini). On peut alors écrire
$$
\Pr_\mu\{X_n = j\} 
= \sum_{i \in \Ecal} \Pr\{X_0 = i\} \Pr\{X_n = j \mid X_0 = i\}
= \sum_{i \in \Ecal} \mu_i p_{ij}(n)
= \sum_{i \in \Ecal} \mu_i [P^n]_{ij}
= [\mu^\top P^n]_j.
$$

%-------------------------------------------------------------------------------
\subsection{Distribution stationnaire}  
%-------------------------------------------------------------------------------

\begin{definition*}[Convergence en loi]
  Une suite de v.a. $(Y_n)_{n \geq 0}$ à valeur dans $\Ecal$ converge en loi vers une distribution $\nu$ sur $\Ecal$ :
  $$
  (Y_n)_{n \geq 0} \overset{\Lcal}{\underset{n \rightarrow \infty}{\longrightarrow}} \nu
  $$
  si
  $$
  \forall i \in \Ecal: \qquad \lim_{n \rightarrow \infty} \Pr\{Y_n = i\} = \nu_i.
  $$
\end{definition*}

\begin{exercise*}
  Montrer que, si
  $$
  (Y_n)_{n \geq 0} \overset{\Lcal}{\underset{n \rightarrow \infty}{\longrightarrow}} \nu
  $$
  alors, pour toute fonction $f : \Ecal \mapsto \Rbb$ bornée, 
  $$
  \lim_{n \rightarrow \infty} \Esp(f(Y_n)) = \sum_{i \in \Ecal} \nu_i f(i)
  $$
  (c'est à dire $\lim_{n \rightarrow \infty} \Esp(f(Y_n)) = \Esp(Y)$ pour $Y \sim \nu$).
\end{exercise*}

\proof
\begin{description}
  \item[$\Ecal$ fini :] la démonstration vient en intervertissant la somme et la limite : 
  $$
  \lim_{n \rightarrow \infty} \Esp(f(Y_n)) 
  = \lim_{n \rightarrow \infty} \left(\sum_{i \in \Ecal} \Pr\{Y_n = i\} f(i)\right)
  = \sum_{i \in \Ecal} \left(\lim_{n \rightarrow \infty} \Pr\{Y_n = i\} \right) f(i) 
  = \sum_{i \in \Ecal} \nu_i f(i).
  $$
  \item[$\Ecal$ infini dénombrable :] sans perte de généralité, on identifie $\Ecal$ à $\Nbb$. Comme pour le cas fini, on montre facilement que 
  $$
  \lim_{n \rightarrow \infty} \Pr\{Y_n \leq i\} = \Pr\{Y \leq i\}
  \qquad \Rightarrow \qquad 
  \lim_{n \rightarrow \infty} \Pr\{Y_n > i\} = \Pr\{Y > i\}.
  $$
  On note $M$ un majorant de $f$ (qui est bornée). On peut écrire que
  \begin{align*}
    \left|\Esp(f(Y_n)) - \Esp(Y)\right|
    & = \left|\sum_{j \geq 0} \Pr\{Y_n=j\} f(j) - \sum_{j \geq 0} \nu_j f(j)\right| \\
    & \leq \left|\sum_{j=0}^i \Pr\{Y_n=j\} f(j) - \sum_{j=0}^i \nu_j f(j)\right|
    + \left|\sum_{j > i} \Pr\{Y_n=j\} f(j) - \sum_{j > i} \nu_j f(j)\right| \\
%     & \leq \left|\sum_{j=0}^i \Pr\{Y_n=j\} f(j) - \sum_{j=0}^i \nu_j f(j)\right|
%     + \left|\sum_{j > i} \Pr\{Y_n=j\} f(j)\right| + \left|\sum_{j > i} \nu_j f(j)\right| \\
    & \leq M \sum_{j=0}^i \left|\Pr\{Y_n=j\} - \nu_j\right| + M \Pr\{Y_n > i\} + M \Pr\{Y > i\}.
  \end{align*}
  Maintenant, pour tout $\epsilon > 0$, il existe un $i$ et un $n_0$ tels que 
  $$
  \Pr\{Y > i\} < \epsilon / 3M
  \qquad \text{et} \qquad 
  \forall n > n_0: \quad \Pr\{Y_n > i\} < \epsilon / 3M.
  $$
  Puisque $i$ est fini, il existe de plus $n_1 > n_0$ tel que 
  $$
  \forall n > n_1, \; \forall j \leq i: \quad \left|\Pr\{Y_n=j\} - \nu_j\right| < \frac\epsilon{3M(i+1)}.
  $$
  On a donc, 
 \begin{align*}
   \forall n > n_1: \qquad 
   \left|\Esp(f(Y_n)) - \Esp(Y)\right| \leq \frac{M\epsilon}{3M} \underset{=1}{\underbrace{\sum_{j=0}^i \frac1{i+1}}} + \frac{M\epsilon}{3M} + \frac{M\epsilon}{3M} = \epsilon
  \end{align*}
  ce qui démontre le résultat.
\end{description}
\eproof

\begin{definition*}
  Une distribution $\nu$ sur $\Ecal$ est dite stationnaire pour la \cM $X$ si il existe une distribution $\mu$ telle que 
  $$
  \lim_{n \rightarrow \infty} \mu^\top P^n = \nu.
  $$
  C'est à dire que la \cM $X$ de transition $P$ et de distribution initiale $\mu$ converge en loi vers $\nu$.
\end{definition*}

\begin{definition*}
  Une distribution $\nu$ sur $\Ecal$ est dite invariante pour la \cM $X$ si 
  $$
  \nu^\top P = \nu^\top.
  $$
\end{definition*}

\begin{proposition*}
  Une distribution $\nu$ sur $\Ecal$ est stationnaire pour la \cM $X$ ssi elle est invariante.
\end{proposition*}

\proof
\begin{description}
  \item[$\Leftarrow$ :] il suffit de prendre $\mu = \nu$ pour s'assurer que $\mu^\top P^n = \nu^\top P^n = \nu$ pour tout $n$. La \cM est alors dite {\em stationnaire}.
  \item[$\Rightarrow$ :] si $\nu$ est stationnaire alors il existe $\mu$ telle que $\mu^\top P^n$ converge vers $\nu$ (i.e. $\lim_{n \rightarrow \infty} [\mu^\top P]_j = \nu_j$), ce qui implique que $\mu^\top P^{n+1}$ converge également vers $\nu$, i.e.
  $$
  \lim_{n \rightarrow \infty} [\mu^\top P^{n+1}]_j = \nu_j.
  $$
  Or, par l'exercice précédent, en prenant $f(i) = p_{ij}$, on a
  $$
  \lim_{n \rightarrow \infty} [\mu^\top P^{n+1}]_j
  = \lim_{n \rightarrow \infty} \sum_{i \in \Nbb} [\mu^\top P^n]_i p_{ij}
  = \sum_{i \in \Nbb} \nu_i p_{ij}
  = [\nu^\top P]_j,
  $$
  soit $\nu^\top P = \nu^\top$.
\end{description}
\eproof


% %-------------------------------------------------------------------------------
% \subsubsection{Chaîne de Markov}
% %-------------------------------------------------------------------------------
% 
% On revient maintenant aux chaînes de Markov introduites à la section \ref{sec:MatStoch} et qui seront étudiées plus en détail à la section \ref{sec:Proba-Markov}.
% 
% \begin{definition*}
%   Une chaîne de Markov homogène à espace fini est une suite de variables aléatoires $\{X(n)\}_{n \geq 0}$ à valeur dans $\Xcal$ (identifié à $\{1, \dots k\}$) non indépendantes mais telles que
%   $$
%   \Pr\{X(n+1) = j \mid X(0)=x_0, X(1) = x_1) \dots X(n) = i\}
%   = \Pr\{X(n+1) = j \mid X(n) = i\}
%   = p_{ij}.
%   $$
%   La matrice $P = [p_{ij}]$ est appelée matrice de transition de la chaîne de Markov.
% \end{definition*}
% 
% La matrice $P$ est une matrice stochastique car tous ses éléments sont positifs ou nuls et leur somme en ligne vaut 1 : 
% $$
% \sum_{j = 1}^k p_{ij} = 1.
% $$
% 
% \begin{proposition*}
%   Soit $\mu_0$ la distribution de $X(0)$ : $\mu_{0i} = \Pr\{X(0) = i\}$ et $mu_n^{\mu_0}$ la distribution de $X(n)$ sachant la distribution initiale $\mu_0$, on a 
%   $$
%   \mu_n^{\mu_0} = \mu_0 A^n 
%   $$
% \end{proposition*}
% 
% \proof
% Par récurrence, partant de $\mu_1^{\mu_0} = A \mu_0$.
% \eproof
% 
% Comme pour les modèles de dynamique des populations, le comportement de la chaîne de Markov en temps long est gouverné par par celui de $A^n$ (et par $\mu_0$).
% 
% \begin{proposition*}
%   Si $A$ est diagonalisable, ses valeurs propres sont aussi les valeurs propres de $A^\top$ et les vecteurs propres de $A^\top$ sont les vecteurs lignes de $P^{-1}$.
% \end{proposition*}
% 
% \proof
%   Il suffit de remarquer que, puique $A = P D P^{-1}$, on a 
%   $$
%   A^\top 
%   = (P D P^{-1})^\top 
%   = (P^{-1})^\top D^\top  P^\top
%   = (P^{-1})^\top D P^\top 
%   $$
%   donc $A^\top$ est aussi diagonalisable et possède les même valeurs propres (contenues dans $D$) que $A$. Pour les mêmes raisons, les vecteurs propres de $A^\top$ sont les vecteurs colonnes de $(P^{-1})^\top$, c'est à dire les vecteurs lignes de $P^{-1}$.
% \eproof
% 
% \remark
% Si $A$ est diagonalisable et que $u$ est un vecteur propre de $A^\top$, il existe $\lambda$ tel que
% $$
% A^\top u  = \lambda u 
% \qquad \Leftrightarrow \qquad
% (A^\top u)  = \lambda u^\top 
% \qquad \Leftrightarrow \qquad
% u^\top A  = \lambda u^\top 
% $$
% $u^\top$ est appelé vecteur propre {\em à gauche} de $A$ (les vecteurs propres précédemment définis étant donc des vecteurs propres {\em à droite}). Cette remarque est notamment utile pour étudier les matrice stochastiques.
% 
% Le théorème de Perron-Frobénius donne des conditions garantissant que la distribution $\mu_n$ converge vers le vecteur propre (à gauche) associé à la valeur propre 1, quelque soit la distribution initiale $\mu_0$. Ces propriétés seront étudiées en détail à la section \ref{sec:Proba-Markov}.
% 
% %-------------------------------------------------------------------------------
% \paragraph*{Exemple.}
% Succession d'espèces d'arbres (R. Arditi)
% \dessin{\url{TreeSpeciesMC}}

%-------------------------------------------------------------------------------
\subsection{Comportement en temps long}  
%-------------------------------------------------------------------------------

\begin{proposition*}[Exercice 1.1.14+]
  Si $\Ecal$ est fini, $1$ est la plus grande valeur propre en module de $P$. \\
  Si $P$ est de plus régulière, $1$ la seule valeur propre de module 1 et est d'ordre de multiplicité : elle donc la valeur propre dominante de $P$.
\end{proposition*}

\proof
\begin{enumerate}
 \item On montre facilement que 1 est valeur propre de $P$ (car $P$ est stochastique), donc $\lambda_1 \geq 1$.
 \item Soit $\lambda$ une valeur propre de $P$ et $v$ un vecteur propre associé, on a $P^n v = \lambda^n v$ où $P^n$ est stochastique (voir section \ref{sec:MatStoch}), donc les coordonnées de $A^n v$ sont bornées par la plus grande coordonnée de $v$, ce qui impose que $\lambda \leq 1$, donc $\lambda_1 \leq 1$.
 \item Le fait que $P$ soit régulière assure, par le théorème de Perron-Frobenius, que $\lambda_1$ est la valeur propre unique de module 1, toutes les autres ayant des modules strictement inférieurs. 
\end{enumerate}
\eproof

\remark
Le théorème de Perron-Frobenius nous assure que les vecteurs propres (à gauche et à droite) associés à $\lambda_1 = 1$ ont toutes leurs coordonnées positives. On a ainsi démontré que, pour $\Ecal$ fini, si $P$ est régulière, 
\begin{itemize}
  \item elle admet une unique distribution stationnaire $\nu$ et que $\nu_i > 0$ pour tout $i \in \Ecal$, 
  \item quelque soit la distribution initiale $\mu$, $\mu P^n \rightarrow \nu$, c'est-à-dire que $X_n$ converge en loi vers $\nu$.
\end{itemize}

\begin{definition*}[Communication entre états]
  On dit qu'il existe un {\em chemin de $i$ vers $j$} (ou que $j$ est {\em accessible depuis $i$}) s'il existe $n \geq 0: p_{ij}(n) > 0$. \\
  S'il existe une chemin de $i$ vers $j$ et un chemin de $j$ vers $i$, on dit que $i$ et $j$ communique et on note $i \sim j$.
\end{definition*}

\remark
Par définition, tout état communique avec lui même (en prenant $n= 0 : P^0 = I \Rightarrow p_{ii}(0) = 1$).

La relation de communication est une relation d'équivalence, c'est à dire qu'elle est :
\begin{itemize}
  \item reflexive: $\forall i \in \Ecal: i \Rcal i$,
  \item symétrique: $\forall i, j \in \Ecal: i \Rcal j \Rightarrow j \Ecal i$,
  \item transitive: $\forall i, j, k \in \Ecal: i \Rcal j, j \Rcal k \Rightarrow i \Ecal k$.
\end{itemize}
Elle définit donc des classes d'équivalences $\Ccal_1, \Ccal_2, \dots $, qui constituent une partition de $E$:
$$
\cup_{k \geq 0} \Ccal_k = \Ecal, \qquad \forall j \neq k: \quad \Ccal_j \cap \Ccal_k = \varnothing.
$$
Les classes $\Ccal_k$ sont appelées {\em classes de communication}.

\begin{definition*}
  Une \cM est dite {\em irréductible} si elle admet une seule classe de communication.
\end{definition*}

\remark
L'hypothèse de régularité de $P$ implique que $X$ est irréductible (puisque $\exists n: P^n > 0$) mais l'inverse, puisque les états peuvent tous communiquer pour des $n$ différents (c exemple suivant).


%-------------------------------------------------------------------------------
\paragraph*{Exemple pour $\Ecal = \{0, 1\}$.}
On considère la \cM $X$ de matrice de transition 
$$
P = \left[\begin{array}{ccc}
      1 - p & & p \\
      q & & 1 - q 
    \end{array}\right]
$$
\begin{description}
  \item[$p = q = 0$ :] alors $P = I$ et toute distribution est stationnaire, mais $P$ n'est pa régulière et la loi de $X_n$ est la même que celle de $X_0$.
  \item[$0 < pq < 1$ :] alors on montre que la seule distribution stationnaire est 
  $$
  \left[ \frac{p}{p+q} \;  \frac{q}{p+q} \right],
  $$
  et comme $P$ est régulière, $X_n$ converge en loi vers $\nu$ quelque soit la distribution initiale $\mu$.
  \item[$p = q = 1$ :] alors la seule distribution stationnaire est 
  $$
  \left[ \frac12 \;  \frac12 \right],
  $$
  mais comme $P$ n'est pas régulière, $X_n$ ne converge pas nécessairement vers $\nu$. En fait $X_n$ alterne entre les états $0$ et $1$ et ne converge en loi vers $\nu$ que si $\mu = \nu$. Si $X_0 = i$, alors $X_n$ ne converge pas du tout.
\end{description}

%-------------------------------------------------------------------------------
\subsection{Classification des états}  
%-------------------------------------------------------------------------------

% Loi du nombre de visites d’un état transient, théorème ergodique pour les chaînes récurrentes à espace d’états fini. 

On cherche maintenant à établir une classification des comportements possibles d'une \cM. La notion de temps de retour est centrale pour effectuer cette description.

\begin{definition*}[Temps de retour]
  Le temps de retour dans l'état $i$ est défini pour $X_0 = i$ par
  $$
  T_i := \min\{n : X_n = i\} \leq + \infty \qquad (\text{i.e. possiblement } + \infty).
  $$
\end{definition*}

\begin{definition*}[\'Etat récurrent]
  L'état $i$ est dit récurrent si $\Pr_i\{T_i < + \infty\} = 1$. Dans le cas contraire, $i$ est dit transient. \\
  L'état $i$ est dit récurrent positif s'il est récurrent et $\Esp(T_i) < + \infty$. Dans le cas contraire il est dit récurrent nul.
\end{definition*}

\begin{definition*}[\'Etat transient]
  L'état $i$ est transient si $\Pr_i\{T_i < + \infty\} = 1$.
\end{definition*}

\begin{definition*}[Période d'un état]
  La période de l'état $i$ est le plus grand diviseur commun des temps de retour possible dans l'état :
  $$
  \per(i) = PGCD\{n: p_{ii}(n) > 0\}.
  $$
  Une \cM est dite {\em apériodique} si $\per(i) = 1$ pour tout $i$.
\end{definition*}

\dessin{
Dessinner qques graphes de transition de \cM avec ou sans période.}

\begin{proposition*}[Cas $\Ecal$ fini]
  Si $\Ecal$ est fini, il n'existe pas d'état récurrent nul et il existe au moins un état récurrent positif.
\end{proposition*}

\proof Non démontré. \eproof

\begin{proposition*}
  Tous les états d'une même classe de communication sont de même nature (récurrent positif, récurrent nul, transient). \\
  Si $\Ecal$ est fini et $X$ irréductible, son unique classe de communication est récurrente positive.
\end{proposition*}

%-------------------------------------------------------------------------------
\paragraph*{Exemple : marche aléatoire sur $\Zbb$.}
On considère la \cM $X$ à valeur dans $\Ecal = \Zbb$, de probabilités de transition
$$
p_{i, j} = \left\{\begin{array}{rl}
                   p & \text{si } j = i+1, \\
                   1 - p & \text{si } j = i-1, \\
                   0 & \text{sinon.}
                  \end{array}\right.
$$
\begin{description}
  \item[$p = 0$ ou $p = 1$:] chaque singleton constitue une classe transiente (la \cM y passe une fois et une seule).
  %
  \item[$p \in (0, 1)$:] $X$ est irréductible ($\Zbb$ forme une unique classe de communication). En notant $E_n$ le déplacement $X_n - X_{n-1}$, on voit que les $E_n$ sont iid à valeur dans $\{-1, +1\}$ et de loi
  $$
  \Pr\{E_n = 1\} = p = 1 - \Pr\{E_n = -1\}.
  $$
  On a donc $\Esp(E_n) = 2p-1$ et, puisque
  $$
  X_n = X_0 + \sum_{i=1}^n X_i,
  $$
  la loi des grands nombre nous assure que 
  $$
  \lim_{n \rightarrow \infty} \frac{X_n}n = \lim_{n \rightarrow \infty} \left(\frac{X_0}n + \frac1n \sum_{i=1}^n X_i\right) = 2p - 1.
  $$
  \item[$p \in (0, 1)$ et $p \neq 1/2$:] alors $X_n$ se comporte comme $(2p-1)n$ : la trajectoire est dite {\em balistique} (elle s'écarte de son origine à vitesse linéaire). $X_n$ tend vers $+\infty$ si $p > 1/2$ et vers $-\infty$ si $p < 1/2$. La \cM est transiente.
  \item[$p = 1/2$:] on parle alors de marche aléatoire {\em symétrique}. En remarquant que 
  \begin{align*}
      \Var(E_n) & = \Esp(E_n^2) - (2p - 1)^2 = 1 - (4p^2 - 4p + 1) = 4p(1-p) \\
      & = 1 \text{ si } p = 1/2,
  \end{align*}
  le théorème centrale limite nous assure que 
  $X_n/n$ converge en loi vers $\Ncal(0, 1)$, c'est-à-dire que 
  $$
  \lim_{n \rightarrow \infty} \Pr\left\{\frac{X_n}n \in [a, b]\right\} = \Pr\{\Ncal(0, 1) \in [a, b]\}.
  $$
  $X_n$ si situe alors typiquement à une distance $\sqrt{n}$ de son origine. \\
  On peut montrer que $X$ est récurrente nulle : la \cM repasse un nombre infini de fois dns chaque état, mais les temps de passage sont de plus en plus espacés.
\end{description}

%-------------------------------------------------------------------------------
\subsection{Théorème ergodique}  
%-------------------------------------------------------------------------------

\begin{theorem*}[Théorème ergodique]
  Si $X$ est une \cM irreductible et récurrente positive, alors 
  \begin{itemize}
   \item $X$ admet une unique distribution stationnaire $\nu$,
   \item $X$ est ergodique, c'est à dire que, pour toute distribution initiale $\mu$, on a
   $$
   \Pr_\mu\left\{\lim_{n \rightarrow \infty} \frac1n \left|\{1 \leq k \leq n: X_k = i\}\right| = \nu_i\right\} = 1,
   $$
   \item si de plus $X$ est apériodique alors $X$ converge en distribution vers $\nu$ : 
   $$
   \lim_{n \rightarrow \infty} \Pr_\mu\{X_n = i\} = \nu_i.
   $$
  \end{itemize}
\end{theorem*}

\remark
Ce théorème est plus fort que la convergence en loi car il nous informe que la trajectoire de $X = (X_n)_{n \geq 0}$ elle même en assurant qu'elle passe par chaque état $i$ un nombre de fois proportionnelle à $\nu_i$.

\remark
La convergence en loi n'est garantie que dans le cas apériodique.

%-------------------------------------------------------------------------------
\paragraph*{Retour à l'exemple pour $\Ecal = \{0, 1\}$.}
\begin{description}
  \item[$p = q = 0$ :] la chaîne possède 2 classes récurrente ($\{0\}$ et $\{1\}$). Elle reste en fait constante et toute mesure $\nu$ est invariante.
  \item[$p = 0$ et $q \neq 0$ :] la chaîne possède une classe récurrente ($\{0\}$) et une classe transiente ($\{1\}$). La chaîne est 'absorbée' par l'état $0$ en un temps fini et la seule distribution invariante est $\nu = [1 \; 0]$.
  \item[$p = q = 1$ :] la chaîne est irréductible (on vérifie que $P^2 = I$) et récurrente positive, donc elle est ergodique (chaque état est visité la moitié du temps) mais elle est périodique et ne converge pas en loi.
  \item[$p + q = 1$ et $pq < 1$:] la chaîne est irréductible, récurrente positive et périodique donc elle est ergodique et converge en loi vers $\nu$.
\end{description}

