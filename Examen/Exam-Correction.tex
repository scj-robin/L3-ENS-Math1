%-------------------------------------------------------------------------------
%-------------------------------------------------------------------------------
% \section*{Correction}
%-------------------------------------------------------------------------------

% %-------------------------------------------------------------------------------
% \subsection{Algèbre linéaire}
% %-------------------------------------------------------------------------------

\paragraph{Solution de l'exercice \ref{exam:AlgLin}.}
  \begin{enumerate}
% [Ex 1.3.3]
  \item On a 
  $$
  |A - \lambda I_2| 
  = \left|\begin{array}{cc} a - \lambda & b \\ c & d - \lambda \end{array}\right|
  = (a - \lambda)( d - \lambda) - bc
  = \lambda^2 - (a+d) \lambda + (ad - bc)
  = \lambda^2 - p \lambda + q
  $$
  qui admet deux solutions réelles (éventuellement égales) ssi
  $$
  p^2 - 4 q \geq 0
  \qquad \Leftrightarrow \qquad
  q \leq p^2 / 4.
  $$
  \item On développe par rapport à la première colonne.
  \item La premièrre valeur propre est donc $\lambda_1 = -1$. Le discriminant du polynôme 
  $$
  Q(\lambda) = \lambda^2 - 4 \lambda + 4 - (\alpha+\beta)
  $$
  vaut $\Delta = 16 - 4(4 - (\alpha+\beta)) = \alpha + \beta$. Les deux valeurs propres restantes $\lambda_2$ et $\lambda_3$ sont donc réelles et distinctes ssi $\alpha + \beta > 0$ et valent alors
  $$
  \lambda_2 = \frac{4 + \sqrt{\alpha + \beta}}{2}, 
  \qquad \lambda_3 = \frac{4 - \sqrt{\alpha + \beta}}{2}.
  $$
  Elles sont de plus distinctes de $\lambda_1 = -1$ ssi 
  $$
  \lambda_3 \neq -1 \qquad \leftrightarrow \qquad
  \alpha + \beta \neq 9.
  $$
  Dans ce cas ($\alpha + \beta > 0$ et $\alpha + \beta \neq 9$), les 3 valeurs propres sont réelles et distinctes et $B$ est donc diagonalisable. \\
  Reste à étudier les cas 
  \begin{description}
   \item[$\alpha + \beta = 0 \Leftrightarrow \lambda_2 = \lambda_3$:] \todo{déterminer la dimension du sous-espace propre associé.} 
   \item[$\alpha + \beta = 9 \Leftrightarrow \lambda_1 = \lambda_3$:] \todo{déterminer la dimension du sous-espace propre associé.} 
  \end{description}

  \end{enumerate}

% %-------------------------------------------------------------------------------
% \subsection{Fonction de plusieurs variables}
% %-------------------------------------------------------------------------------

% %-------------------------------------------------------------------------------
% \subsection{\'Equations différentielles}
% %-------------------------------------------------------------------------------

\paragraph{Solution de l'exercice \ref{exam:DynPop}.}
\begin{enumerate}
  \item $xy$ est la densité de couples possible parmi les individus flottants. Le paramètre $\alpha$ mesure la proportion de couples qui se forment effectivement. $r$ est le taux de naissance instantané de mâles et de femelles (le taux de naissance par couple vaut donc $2r$, à proportion égales entre mâles et femelles). $c$ est le taux de mort des couples.
  \item Selon le système \eqref{eq:DynPop}, on a $\dot x(t) - \dot y(t) = 0$ pour tout $t$, ce qui implique que la différence $x(t) - y(t)$ reste constante au cours du temps, donc que $x(t) - y(t) \equiv S$. \\
  Pour obtenir le système \eqref{eq:DynPop2}, il suffit de remplacer $x$ par $y + S$.
  \item $(y, z) = (0, 0)$ est clairement un point d'équilibre. \\
  Si $z > 0$, alors
  $$
  \dot y(t) = \dot z(t) = 0 (= - \dot z (t)) 
  \qquad \Rightarrow \qquad 
  -\alpha (y^2 + Sy) + r z = -\alpha (y^2 + Sy) + c z^2
  $$
  soit $z = r/c$, qui implique à son tour que
  $$
  y^2 + Sy - r^2/(\alpha c) = 0.
  $$ 
  Le discriminant de cette équation vaut $\Delta = S^2 + 4r^2/(\alpha c) > S^2$. La seule solution positive est donc $y = (\sqrt{\Delta} - S)/2$. Le second équilibre est donc $(y^*, z^*) = ((\sqrt{\Delta} - S)/2, r/c)$.
  \item La matrice jacobienne en $(y, z)$ vaut
  $$
  J_{(y, z)} = \left[\begin{array}{rr}
                      - \alpha (2y + S) & r \\
                      \alpha (2y + S) & -2 cz
                     \end{array}\right].
  $$
  Au point d'équilibre $(y^*, z^*)$, elle vaut donc
  $$
  J_{(y^*, z^*)} = \left[\begin{array}{rr}
                          - \alpha \sqrt{\Delta} & r \\
                          \alpha \sqrt{\Delta} & -2 r
                        \end{array}\right]
  $$
  donc le polynôme caractéristique vaut
  \begin{align*}
    P(\lambda) 
    = \left|\begin{array}{rr}
                - \alpha \sqrt{\Delta} - \lambda & r \\
                \alpha \sqrt{\Delta} & -2 r - \lambda
            \end{array}\right|
    & = (\alpha \sqrt{\Delta} + \lambda) (2 r + \lambda) - \alpha r \sqrt{\Delta} \\
    & = \lambda^2 + (\alpha \sqrt{\Delta} + 2r) \lambda + \alpha r \sqrt{\Delta} \\
    & = \lambda^2 - \tr(J_{(y^*, z^*)}) + \det(J_{(y^*, z^*)}).
  \end{align*}
  On observe d'abord que le discriminant de cette équation vaut $(\alpha^2 \Delta + 4r^2 > 0$ et que la jacobienne admet donc deux valeurs propres réelles distinctes. \\
  On utilisant ensuite le fait que le déterminant de $J_{(y^*, z^*)}$ est le produit de ses valeurs propres et sa trace $J_{(y^*, z^*)}$ est leur somme, on observe que les deux valeurs propres sont de même signe (puisque $\det(J_{(y^*, z^*)}) = \alpha r \sqrt{\Delta} > 0$) et qu'elles sont toutes les deux négatives (puisque $\tr(J_{(y^*, z^*)}) = - \alpha \sqrt{\Delta} - 2r < 0$), d'où on conclue que l'équilibre $(y^*, z^*)$ est stable.
\end{enumerate}

\paragraph{Solution de l'exercice \ref{exam:Proba}.}

\begin{enumerate}
  \item La fonction génératrice est définie par 
  \begin{align*}
    f(s) 
    & = \Esp(s^X) = \sum_{k \geq 0} p_k s^k
    = p_0 + \sum_{k \geq 1} p_k s^k
    = b + (1 - b) (1 - a) \sum_{k \geq 1} a^{k-1} s^k \\
    & = b + \frac{(1 - b) (1 - a)}a \sum_{k \geq 1} (as) s^k .
  \end{align*}
  La formule donnée vient de ce que $\sum_{k \geq 1} (as) s^k = as / (1 - as)$. 
  \item On sait que la probabilité d'extinction $q$ est la plus petite solution de l'équation de point fixe
  $$
  f(s) = s
  \qquad \Leftrightarrow \qquad
  a s^2 - (a+b) s + b = 0.
  $$
  Le discriminant de cette équation vaut $(a+b)^2 - 4ab = (a-b)^2$. Les deux solutions sont
  donc $s = 1$ et $s = b/a$. La probabilité d'extinction $q$ vaut donc $b/a$ si $b < a$ et 1 sinon.
  \item La colonisation réussit si la descendance d'au moins un individu fondateur ne s'éteint pas, ce qui se produit avec probabilité $1 - q^N$. $p$ est donc nulle pour tout $N$ si $q = 1$ et non nulle ssi $q < 1$, c'est à dire si $b < a$.
  \item $1 - q^N$ est la probabilité de colonisation conditionnelle à l'effectif initial $N$. La probabilité de colonisation vaut donc
  $$
  p = \Esp(1 - q^N) = 1 - \Esp(q^N) = 1 - g(q).
  $$
  \item Si $N = 1$, on a immédiatement $p = 1 - q$. Si $N$ est distribué comme $X$, alors $g = f$ et donc
  $$
  p = 1 - g(q) = 1 - f(q) = 1 - q
  $$
  puisque $q$ est un point fixe de $f$ ($f(q) = q$). \\
  L'égalité des résultats n'est pas surprenante puisque le second cas correspond au premier décalé d'une génération.
\end{enumerate}


% %-------------------------------------------------------------------------------
% \subsection{Probabilités}
% %-------------------------------------------------------------------------------
