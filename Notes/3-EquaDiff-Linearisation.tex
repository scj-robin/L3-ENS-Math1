%-------------------------------------------------------------------------------
%-------------------------------------------------------------------------------
\section{ODE non linéaires} \label{sec:EquaDiff-nonLineaire}
%-------------------------------------------------------------------------------

%-------------------------------------------------------------------------------
\subsection{ODE linéaire en dimension $n$.}
%-------------------------------------------------------------------------------

\begin{definition*}
  Une ODE $\dot y = F(y)$ est dite linéaire si l'application $F : \Rbb \mapsto \Rbb$ est linéaire : 
  $$
  \dot y (t) = A y(t).
  $$
\end{definition*}

\remark
On remarque immédiatement que les points d'équilibre $x^*$ sont les élément du noyau $\ker(A)$, i.e. les vecteurs propres associés à la valeur propre 0.

%-------------------------------------------------------------------------------
\paragraph{Cas diagonalisable.}

\begin{proposition*}[ODE linéaire diagonalisable]
  Si la matrice $A$ est diagonalisable : 
  $$
  A = P D P^{-1}
  \qquad \text{avec}  \quad 
  D = \diag(\lambda_1, \dots \lambda_n),
  $$
  la solution du problème
  $$
  \{y(0) = y_0, \; \dot y = A y\}
  $$
  est
  $$
  y(t) = P E(t) P^{-1} y_0  
  $$
  avec
  $$
  E(t) = \diag\left(e^{\lambda_1 t} \dots e^{\lambda_n t} \right)
  \qquad \text{et} \qquad
  z_0 = P^{-1} y_0.
  $$
\end{proposition*}

\proof
En posant $z(t) = P^{-1} y(t)$ et puisque la dérivation est linéaire, il vient 
$$
\dot z (t) = P^{-1} \dot y(t) = P^{-1} A y(t) = P^{-1} P D P^{-1} y(t) = D z(t).
$$
On obtient ainsi que les fonctions $z_i(t)$ sont respectivement solutions de $\dot z_i = \lambda_i z_i$, soit 
$$
z_i(t) = z_{0i} e^{\lambda_i t}
$$
avec $z_0 = z(0) = P^{-1} y(0) = P^{-1} y_0$. La solution s'obtient en revenant à $y(t) = P z(t)$.
\eproof

\remark
\begin{enumerate}
  \item La proposition dit que, au prix du changement de base $z = P^{-1} y$, le système $\dot Y = A y$ est équivalent à $n$ équations différentielles linéaires unidimensionnelles.
  \item La condition initiale $y_0$ se décompose sous la forme
  $$
  y_0 = \sum_i z_{i0} v_i
  $$
  où les $v_i$ sont les vecteur propres de $A$, i.e. les vecteurs colonnes de $P$.
  \item En définissant l'exponentielle matricielle
  $$
  \exp(B) = \sum_{k \geq 0} \frac1{k!} B^k,
  \qquad \Rightarrow \qquad
  \exp(P B P^{-1}) = P \exp(B) P^{-1}
  $$
  on obtient ainsi que
  $$
  E(t) = \exp(t D) 
  \qquad \text{et} \qquad 
  P E(t) P^{-1} = P \exp(t D) P^{-1} = \exp(t P D P^{-1}) = \exp(t A).
  $$
  On retrouve ainsi que la solution du problème
  $$
  \{y(0) = y_0, \; \dot y = A y\}
  $$
  s'écrit
  $$
  y(t) = \underset{\in \Mcal_n}{\underbrace{\exp(t A)}} \; y_0.
  $$
  \item On peut étudier le comportement en temps long du système en considérant
  \begin{align*}
    \lim_{t \rightarrow \infty} y(t)
    & = \lim_{t \rightarrow \infty} \exp(t A) y_0 
    = P \left(\lim_{t \rightarrow \infty} \exp(t D)\right) \underset{z_0}{\underbrace{P^{-1} y_0}}
    = \sum_{i=1}^n \left(z_{0i} \lim_{t \rightarrow \infty} e^{\lambda_i t}\right) v_i.
  \end{align*}
  Comme
  $$
  \lim_{t \rightarrow \infty} e^{\lambda_i t} = 
  \left\{\begin{array}{rr}
          0 & \text{si } \lambda_i < 0 \\
          1 & \text{si } \lambda_i = 0 \\
          +\infty & \text{si } \lambda_i > 0
         \end{array}\right.
  $$
  on peut donc distinguer deux cas : 
  \begin{itemize}
  \item soit il existe $\lambda_i > 0$ telle que $z_{0i} \neq 0$, et on a alors
  $$
  \lim_{t \rightarrow \infty} \|y(t)\| = + \infty
  $$
  (le système explose dans la direction $v_i$), 
  \item soit le système tend vers une combinaison linéaire d'éléments du noyau de $A$ :
  $$
  \sum_{i : \lambda_i = 0} z_{0i} v_i \in \ker(A)
  $$
  qui dépend de la condition initiale $y_0 = P z_0$.
  \end{itemize}
  En conclusion, s'il existe au moins un $\lambda_i \geq 0$, un perturbation autour d'un point d'équilibre $x^* \in \ker(A)$ éloignera le système de $x_ *$ dans la direction $v_i$. A l'opposé, si toutes les $\lambda_i$ sont strictement négatives, quelque soit $y_0$, $\lim_{t \rightarrow \infty} y(t) = 0$ qui est donc le seul équilibre stable.
\end{enumerate}

%-------------------------------------------------------------------------------
\paragraph{Cas non diagonalisable.}

\begin{theorem*}
  Soit $A$ non diagonalisable et $\lambda_1, \dots \lambda_n$ les racines (possiblement complexes, non nécessairement distinctes) de son polynôme caractéristique, quelque soit $y_0$, il existe $n^2$ polynômes (à coefficients complexes) $q_{ij}(t)$ tels quel la solution du problème
  $$
  \{y(0) = y_0, \; \dot y = A y\}
  $$
  soit
  $$
  y_i(t) = \sum_{j=1}^n q_{ij}(t) e^{\lambda_j t}.
  $$
  En conséquence, 0 est le seul équilibre stable possible et il l'est à condition que
  $$
  forall 1 \leq i \leq n: \quad \Re(\lambda_i) < 0
  $$
  en notant $\Re(u)$ la partie réelle du nombre complexe $u$.
\end{theorem*}

%-------------------------------------------------------------------------------
\subsection{Linéarisation} 
%-------------------------------------------------------------------------------

On souhaite maintenant étudier la nature des équilibres d'un système non linéaire. En dimension 1, cet analyse s'est fondée sur la dérivée de la fonction $F$. Le théorème suivant généralise cette représentation fondée sur l'approximation au premier ordre aux abords d'un équilibre $x^*$ en posant $y(t) = x^* + h(t)$: 
$$
\dot h(t) = \dot y (t) = F(y(t)) = F(x^* + h(t)) = (D_{x^*} F)(h(t)) = J_{x^*} h(t).
$$

%-------------------------------------------------------------------------------
\begin{theorem*}[Hartman-Grobman]
  Le point d'équilibre $x^*$ de l'ODE $\dot y = F(y)$ a la même nature que l'origine $0$ pour le système linéaire 
  $$
  \dot y = J_{x^*} y
  $$
  à condition que toutes les valeurs propres de $J_{x^*}$ aient une partie réelle non nulle. \\
  En particulier, $x^*$ est stable si $\forall i: \Re(\lambda_i) < 0$.
\end{theorem*}

%-------------------------------------------------------------------------------
\paragraph{Démarche.}
On peut ainsi définir une démarche systématique pour étudier les équilibres d'un système $\dot y = F(y)$ :
\begin{enumerate}
  \item Trouver les points équilibres en déterminant d'abord les {\em isoclines}
  $$
  \Ical_k = \{x : F_k(x) = 0\}
  $$
  puis leur intersection $\bigcap_k \Ical_k$ pour résoudre $F(x) = 0$ ;
  \item Calculer la matrice jacobienne $J_{x^*} F$ pour tout équilibre $x^*$ ;
  \item Déterminer les racines du polynôme caractéristique $P_{J_{x^*}}(\lambda)$ et en déduire la nature de $x^*$ si toutes les racines ont une partie réelle non nulle.
\end{enumerate}

\vspace{.1\textheight}
\progres{
  Début Cours 8. Rappels :
  \begin{enumerate}[\itemdot]
    \item Système dynamique défini ppour $y(t) \Rbb^n$:
    $$
    \{y(t_0 = y_0, \; \dot y = F(y)\}
    $$
    \item Point d'équilibre $x^*$ : $F(x) = 0$
    \item ($n = 1$) Nature de $x^*$: stable si $F'(x^*) < 0$, instable si $F'(x^*) > 0$.
    \item ($n \geq 1$) Cas linéaire diagonalisable : caractérisation en fonction du signe des valeurs propres \\
    $\Rightarrow$ 0 seul équilibre stable possible (à condition que tous les $\lambda_i < 0$) \\
    $\Rightarrow$ idem cas non-diagonalisable
    \item ($n \geq 1$) Cas non linéaire : nature d'une équilibre $x^*$ = même nature que l'origine pour le système linéaire avec $A = J_{x^*} F$.
  \end{enumerate}
}
