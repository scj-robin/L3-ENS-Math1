%-------------------------------------------------------------------------------
\subsubsection{Système de Lorenz en 2 dimensions} 
%-------------------------------------------------------------------------------

On considère une version réduite à deux dimensions du système de Lorenz, c'est-à-dire à un couple $(x(t), y(t))_{t \geq 0}$ satisfaisant les conditions initiales
$$
x(0) = x_0, \qquad y(0) = y_0
$$
et vérifiant le système d'équations différentielles ordinaires
\begin{equation*} % \label{eq:lorenz2D}
  \left\{\begin{array}{rcl}
    \dot x & = a x - xy \\
    \dot y & = x^2 - by
  \end{array}\right.
\end{equation*}
où les coefficients $a$ et $b$ sont strictement positifs.

\paragraph{Détermination des points stationnaires et de leur stabilité.}
\begin{enumerate}
  \item Etudier les signes de $\dot x$ et $\dot y$ et tracer le champ de vecteur gradient.
  \solution{
  $\dot x$ est positif si $x \geq 0$ et $y \leq a$ ou si $x \leq 0$ et $y \geq a$.
  $\dot y$ est positif si $y \leq x^2/b$.
    $$
    \includegraphics[width=.5\textwidth, trim=30 30 20 50, clip=]{Lorenz2D-gradientField}
    $$
  }
  %
  \item Montrer que le système admet trois points stationnaires.
  \solution{
    En annulant simultanément
    $$
    F_1(x, y) = a x - xy, \qquad F_2(x, y) = x^2 - by,
    $$
    on détermine que le vecteur gradient est nul aux points 
    $$
    O = (0, 0), \qquad A = (\sqrt{ab}, a) \quad \text{et} \quad B = (-\sqrt{ab}, a).
    $$
  }
  \item Déterminer la matrice Jacobienne du système.
  \solution{
  On a
  $$
  J f = \left[\begin{array}{cc} a-y & -x \\ 2x & -b \end{array} \right].
  $$
  }
  \item En déduire la nature (stable ou instable) de chacun des points stationnaires.
  \solution{
  \begin{description}
    \item[$O = (0, 0)$ :] On a
    $$
    J_0 f = \left[\begin{array}{cc} a & 0 \\ 0 & -b \end{array} \right],
    $$
    dont les valeurs propres ($\lambda_1 = a, \lambda_2 = b)$ sont de signes opposés : $O$ est donc instable. Plus précisément, il est instable le long de l'axe des $x$ et stable le long de l'axe des $y$.
    \item[$A = (\sqrt{ab}, a)$ :] On a
    $$
    J_A f = \left[\begin{array}{cc} 0 & -\sqrt{ab} \\ 2\sqrt{ab} & -b \end{array} \right],
    $$
    donc $\tr(J_A f) = -b < 0$ et $\det(J_A f) = 2ab > 0$. Les deux valeurs propres sont donc de même signe et négatives : $A$ est donc un équilibre stable. \\
    {\sl Alternative :} Le polynôme caractéristique
    $$
    P_A(\lambda) = \lambda^2 + b \lambda + 2 ab
    $$
    a pour discriminant $\Delta = b^2 - 8 ab$. 
    \begin{itemize}
      \item Si $0 < a < b/8$, $\Delta$ est positif et les valeurs propres sont 
      $$
      \lambda_1 = \frac12 \left(-b + \sqrt{\Delta}\right) 
      \quad \text{et} \quad
      \lambda_2 = \frac12 \left(-b - \sqrt{\Delta}\right).
      $$
      comme de plus $\Delta < b^2$, les deux valeurs propres sont négatives, donc $A$ est stable.
      \item Si $a > b/8$, $\Delta$ est négatif et les valeurs propres sont 
      $$
      \lambda_1 = \frac12 \left(-b + i \sqrt{|\Delta|}\right) 
      \quad \text{et} \quad
      \lambda_2 = \frac12 \left(-b - i \sqrt{|\Delta|}\right),
      $$      
      dont la partie entière commune est $-b < 0$, donc $A$  est également stable.
    \end{itemize}
    \item[$B = (-\sqrt{ab}, a)$ :] On a alors    
    $$
    J_B f = \left[\begin{array}{cc} 0 & \sqrt{ab} \\ -2\sqrt{ab} & -b \end{array} \right]
    $$
    qui a la même trace et le même déterminant que $J_A f$ : $B$ est donc également stable. \\
    {\sl Alternative :} Le polynôme caractéristique de $J_B f$ est le même que celui de $J_A f$ donc $B$ est de même nature que $A$.
  \end{description}
  }
  \item Tracer l'allure des trajectoirs partant de $(0, a/2)$, $(\varepsilon, a/2)$ et $(-\varepsilon, a/2)$.
  \solution{
    $$
    \includegraphics[width=.5\textwidth, trim=30 30 20 50, clip=]{Lorenz2D-paths}
    $$
  }
\end{enumerate}

\paragraph{Cas de paramètres négatifs.}
On étudie maintenant le cas où les paramètres $a$ et $b$ peuvent prendre des valeurs négatives. On ne s'attardera pas sur les cas limites $a = 0$ et/ou $b = 0$.

\bigskip
\begin{enumerate}
  \setcounter{enumi}{3}
  \item Déterminer le ou les points stationnaires et étudier sa (leur) stabilité quand $a$ est négatif ($b$ étant toujours strictement positif).
  \solution{
    Si $a < 0$ (et $b > 0$), alors $ab < 0$ donc $O$ est le seul point stationnaire, et les deux valeurs propres de $J_O f$ sont négatives, donc $O$ est stable.
  }
  \item Mener la même étude pour $a > 0$ et $b < 0$, d'une part, et pour $a < 0$ et $b < 0$, d'autre part.
  \solution{
  \begin{description}
    \item[$a > 0, b < 0$ :] Alors $ab < 0$ donc $O$ est le seul point stationnaire, mais les deux valeurs propres de $J_O f$ sont positives, donc $O$ est instable. 
    \item[$a < 0, b < 0$ :] Alors $ab > 0$ donc $O$, $A$ et $B$ sont stationnaires.
    \begin{itemize}
      \item Le valeurs propres de $J_O f$ sont alors de signes opposés, donc $O$ est instable.
      \item On a $\tr(J_A f) = \tr(J_B f) = -b > 0$ et $\det(J_A f) = \det(J_B f) = 2 ab > 0$. Les deux valeurs propres sont donc positives et $A$ et $B$ sont donc instables. \\
      {\sl Alternative:} Le discriminant de $P_A(\lambda)$ (et donc de $P_B(\lambda)$) vaut alors $\Delta = b^2 - 8ab < b^2$. Les deux valeurs propres
      $$
      \lambda_1 = \frac12 \left(-b + \sqrt{\Delta}\right) 
      \quad \text{et} \quad
      \lambda_2 = \frac12 \left(-b - \sqrt{\Delta}\right).
      $$
      sont alors positives : $A$ et $B$ sont donc instables.
    \end{itemize}
  \end{description}
  }
\end{enumerate}

