%-------------------------------------------------------------------------------
\subsubsection{Processus de Galton–Watson quadratique \todo{}} %-------------------------------------------------------------------------------
  On considère un processus de Bienaymé-Galton-Watson partant d'une population de taille 1 et dans laquelle le nombre de descendants par individu suit la loi suivante :
  $$=
  \Pr\{X = k\} = \left\{
    \begin{array}{ll}
      1 - a - b & \text{si $k = 0$} \\
      a & \text{si $k = 1$} \\
      b & \text{si $k = 2$}
    \end{array}\right..
  $$
  \begin{enumerate}
    \item Montrer que la fonction génératrice de $X$ est
    $$
    f(s) = 1 - a - b + as + bs^2.
    $$
    \solution{\todo{}}
    \item En déduire le nombre de descendants moyen par individu $m$
    $$
    m = a + 2b.
    $$
    \solution{On a $f'_X(s) = a + 2bs$, soit $m = f'_X(1) = a + 2b$.}
    \item Montrer que les solutions de l'équation $s = f(s)$ sont de la forme
    $$
    s = \frac{(1-a) \pm |m-1|}{2(1 -(a+b))}
    $$
    où $m$ est l'espérance du nombre de descendants d'un individu.
    \solution{On résoud
    $$
    s = f(s)
    \qquad \Leftrightarrow \qquad
    (1 - a -b) + (a-1)s + bs^2 = 0
    $$
    dont le discriminant est
    $$
    \Delta = (a-1)^2 - 4 b (1 - a -b) = ... = (a + 2b - 1)^2 = (m-1)^2
    $$
    soit}
  \end{enumerate}
