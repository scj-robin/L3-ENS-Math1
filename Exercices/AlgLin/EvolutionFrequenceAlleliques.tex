%-------------------------------------------------------------------------------
\subsubsection{\'Evolution de fréquences alléliques}
%-------------------------------------------------------------------------------

\paragraph{Dynamique de fréquences alléliques.}
Dans une population diploïde panmictique, on s’intéresse à un gène existant sous la forme de $m$ allèles. On désigne par $p_i(t)$ la proportion du pool gamétique portant l’allèle $i$  à la génération $t$ et $p(t) = (p_i(t))_{1 \leq i \leq m}$ le vecteur des fréquences alléliques à la génération $t$, qui vérifie $\sum_{1 \leq i \leq m}p_i(t) = 1$. 

On définit la matrice, supposée symétrique, $A = [a_{ij}]_{1 \leq i, j \leq p}$ où les $a_{ij}$ sont tous positifs ou nuls, mais non tous nuls. La dynamique des fréquences alléliques est donnée par
\begin{align} \label{eq:dynFreqModele}
  \forall \; i = 1 \dots m: & & 
  p_i(t+1) & = V(t)^{-1} p_i(t) \sum_{j=1}^m a_{ij} p_j(t) \\
  \text{avec} & & 
  V(t) & = \sum_{i=1}^m \sum_{j=1}^m a_{ij} p_i(t) p_j(j) = p(t)^\top \; A \; p(t), \nonumber
\end{align}
de sorte que $\sum_{1 \leq i \leq m} p_i(t+1) = 1$.

\bigskip
\begin{enumerate}
  \item Interpréter l'équation \eqref{eq:dynFreqModele}. Pourquoi $A$ est-elle supposée symétrique ? Que représente $V(t)$ ?
  \solution{$a_{ij}$ représente l'avantage relatif du génotype $(A_i, A_j)$, qui est symétrique par nature. \\
  $V(p) = p^\top A p$ est l'avantage moyen d'un descendant, du fait de la reproduction panmictique. 
  
  \remark 
  Ce modèle est paramétré à une constante près, c'est-à-dire qu'on aboutit à la même dynamique en remplaçant $A$ par $B = k A$ pour n'importe quel $k > 0$.
  }
\end{enumerate}

\paragraph{Caractérisation d'un équilibre.}
On suppose à partir de maintenant qu'il existe un équilibre $p^* = (p^*_i)_{1 \leq i \leq m}$ pour le système \eqref{eq:dynFreqModele}, c'est-à-dire vérifiant : 
$$
p(t) = p^* \qquad \Rightarrow \qquad p(t+1) = p^*,
$$
et qu'il est ``non trivial'', c'est-à-dire dans lequel tous les allèles sont présents : 
$$
\forall \; i = 1 \dots m: \qquad p^*_i > 0
$$
et on note $V^* = V(p^*) > 0$.

\begin{enumerate}
  \setcounter{enumi}{1}
  \item Montrer que, pour tout $i$ : $\sum_{1 \leq j \leq m} a_{ij} p^*_j = V^*$.
  \solution{L'équilibre $p^*$ est un point stationnaire de la dynamique \eqref{eq:dynFreqModele}, donc, en posant $p(t+1) = p(t) = p^*$, il vient
  $$
  \forall \; i = 1 \dots m: \quad V^* p^*_i = p^*_i \sum_{j=1}^m a_{ij} p^*_j,
  $$
  soit $V^* = \sum_{j=1}^m a_{ij} p^*_j, \forall i$, puisque les $p_i^*$ sont tous non nuls.. }
\end{enumerate}

\paragraph{Condition d'optimalité d'un équilibre.}
On cherche maintenant à déterminer sous quelles conditions l'équilibre (non trivial) $p^*$ correspond à un optimum pour $V$.

\begin{enumerate}
  \setcounter{enumi}{2}
  \item En écrivant tout vecteur de fréquences alléliques $p$ sous la forme $p = p^* + x$, montrer que $V^*$ est maximal ssi
  \begin{equation*} % \label{eq:dynFreqCondition}
    \forall x \in \Rbb^m \text{ tel que } \sum_{i=1}^m x_i = 0: \qquad x^\top A x \leq 0.
  \end{equation*}
  \solution{Si $p = p^* + x$, on a 
  $$
  V(p) 
  = p^\top A p
  = {p^*}^\top A p^* + 2 x^\top A p^* + x^\top A x
  = V^* + 2 x^\top A p^* + x^\top A x.
  $$
  De plus, $p$ et $p^*$ étant des vecteurs de fréquences ($1_m^\top p = 1_m^\top p^* = 1$), si $p = p^* +x$, alors $1_m^\top x = \sum_i x_i = 0$. Or on a vu que $\sum_j a_{ij} p^*_j$ est indépendant de $i$ (et égal à $V^*$), donc $A p^* = V^* 1_m$, donc
  $$
  x^\top A p^* = x^\top V^* 1_m = V^* x^\top 1_m = 0
  $$
  donc
  $$
  V(p) 
  = V^* + x^\top A x,
  $$
  $V^*$ est donc maximal, ssi $\forall p, V(p) \leq V^*$, c'est-à-dire ssi $x^\top A x \leq 0$.
  }
  %
  \item Montrer qu'on peut écrire $A$ sous la forme $A = R \Lambda R^\top$ où
  $$
  \Lambda = \text{diag}(\lambda_1, \dots, \lambda_k, 
    \lambda_{k+1}, \dots, \lambda_{k+\ell}, 
    0, \dots, 0)
  $$
  avec $k \geq 0$, $\ell \geq 0$, $k+\ell \leq m$, $\lambda_1, \dots \lambda_k > 0$ et $\lambda_{k+1}, \dots \lambda_{k+\ell} < 0$.
  \solution{
  $A$ étant symétrique, elle est diagonalisable et ses vecteurs propres sont orthogonaux. On peut donc l'écrire sous la forme $A = R \Lambda R^\top$ en mettant en premier les valeurs propres strictement positives, puis strictement négatives, puis, éventuellement, nulles. 
  }
  %
  \item Montrer l'existence de $p^*$ assure que $k \geq 1$.
  \solution{
  Si $k = 0$, alors tous les $\lambda_i$ sont négatifs (ou nuls), c'est-à-dire que $A$ est négative ($A \preceq 0$), donc
  $$
  \forall x \in \Rbb^m: \qquad 
  x^\top A x 
  = x^\top R \Lambda R^\top x 
  = y^\top \Lambda y  
  = \sum_i \lambda_i y_i^2 \leq 0 
  $$
  en posant $y = R^\top x$, or $V^* = {p^*}^\top A p^* > 0$ (car $A \geq 0$ et $p^*_i > 0$, $\forall i$), donc on a nécessairement $k \geq 1$.
  }
  \item En déduire que $V^*$ est maximal ssi $k=1$ et $\ell \geq 1$. 
  \solution{
  On a vu que, pour que $V^*$ soit maximal, il faut que $x^\top A x \leq 0$ pour tout $x$ vérifiant $\sum_i x_i = 0$, qui définit une sous-espace de dimension $m-1$. $k$ ne peut donc pas être supérieur ou égal à 2 (car alors il existerait alors au moins deux dimensions dans lesquelles $x^\top A x > 0$). \\
  La condition $\ell \geq 1$ assure seulement qu'il existe des vecteurs de fréquences $p = p^* + x$ donnant une viabilité $V(p) = V^* + x^\top A x < V^*$. 
  }
\end{enumerate}
