\documentclass[french, 11pt]{article}

%-------------------------------------------------------------------------------
\usepackage[a4paper,top=1cm,bottom=2cm,left=1cm,right=1cm,marginparwidth=0.5cm]{geometry}
\usepackage{amsmath,amsfonts,amssymb,amsthm}
\usepackage[french]{babel}
\usepackage[utf8]{inputenc}
\usepackage[T1]{fontenc}
\usepackage{enumerate}
\usepackage{natbib}
\usepackage{graphicx}
\usepackage{xspace}
\usepackage{color,xcolor}
\usepackage{tikz}
\usepackage{remreset}
\usepackage{url}
\usepackage{boites}
% \usepackage{extsizes} % Permet \documentclass[french, 14pt]{extreport}
% \usepackage[a4paper,top=1cm,bottom=2cm,left=1cm,right=1cm,marginparwidth=.75cm]{geometry}
% \usepackage{minitoc}

\graphicspath{{../Figures/}}
% Environnement
\newtheorem{theorem}{Théorème}
\newtheorem{definition}{Définition}
\newtheorem{lemma}{Lemme}
\newtheorem{proposition}{Proposition}
\newtheorem*{theorem*}{Théorème}
\newtheorem*{definition*}{Définition}
\newtheorem*{proposition*}{Proposition}
\newtheorem*{corollary*}{Corollaire}
\newtheorem*{assumption*}{Hypothèse}
\newtheorem*{algorithm*}{Algorithme}
\newtheorem*{lemma*}{Lemme}
\newtheorem*{remark*}{Remarque}
\newtheorem*{exercise*}{Exercice}
\newtheorem{exercise}{Exercice}
\newcommand{\remark}{\bigskip\noindent\textbf{\textsl{Remarque.}}\xspace}
\newcommand{\remarks}{\bigskip\noindent\textbf{\textsl{Remarques.}}\xspace}
\newcommand{\parSR}[1]{\paragraph*{\textsl{#1}}\xspace}
\renewcommand{\proof}{\bigskip\noindent\underline{\textsl{Démonstration}.}\xspace}
\newcommand{\eproof}{$\blacksquare$}

% Effets, couleurs
\newcommand{\emphase}[1]{\textcolor{red}{#1}}
\newcommand{\demoProp}[1]{\noindent{\textbf{\textsl{Démonstration de la proposition \ref{#1} :}}}}
\newcommand{\itemdot}{\textbullet}

% Moments
\DeclareMathOperator{\Esp}{\mathbb{E}}
\DeclareMathOperator{\diag}{diag}
\DeclareMathOperator{\Cov}{\mathbb{C}ov}
\DeclareMathOperator{\tr}{tr}
\DeclareMathOperator{\Var}{\mathbb{V}}
\let\Pr\relax\DeclareMathOperator{\Pr}{\mathbb{P}}
\renewcommand{\d}{\text{d}}

% R, N, ...
\newcommand{\cst}{\text{cst}}
\newcommand{\Cbb}{\mathbb{C}}
\newcommand{\Ibb}{\mathbb{I}}
\newcommand{\Nbb}{\mathbb{N}}
\newcommand{\Rbb}{\mathbb{R}}
\newcommand{\Zbb}{\mathbb{Z}}

% Indicateurs

% Lois et ensembles
\newcommand{\Acal}{\mathcal{A}}
\newcommand{\Bcal}{\mathcal{B}}
\newcommand{\Ccal}{\mathcal{C}}
\newcommand{\Ecal}{\mathcal{E}}
\newcommand{\Gcal}{\mathcal{G}}
\newcommand{\Ical}{\mathcal{I}}
\newcommand{\Lcal}{\mathcal{L}}
\newcommand{\Mcal}{\mathcal{M}}
\newcommand{\Ncal}{\mathcal{N}}
\newcommand{\Pcal}{\mathcal{P}}
\newcommand{\Rcal}{\mathcal{R}}
\newcommand{\Scal}{\mathcal{S}}
\newcommand{\Ucal}{\mathcal{U}}
\newcommand{\Xcal}{\mathcal{X}}
\newcommand{\Ycal}{\mathcal{Y}}

% Comments
\newcommand{\SR}[2]{\textcolor{gray}{#1}\textcolor{red}{#2}}
\newcommand{\todo}[1]{\textcolor{red}{\`A faire~: {\sl #1}}}
\newcommand{\dessin}[1]{
\begin{center}\framebox{\begin{minipage}{\textwidth}
  \textcolor{purple}{#1}
\end{minipage}}\end{center}
\bigskip
}
\newcommand{\progres}[1]{
\begin{center}\framebox{\begin{minipage}{\textwidth}
  \textcolor{blue}{{\sl #1}}
\end{minipage}}\end{center}
\bigskip
}
\newcommand{\solution}[1]{
\begin{center}\framebox{\begin{minipage}{\textwidth}
  \noindent{\sl Solution :}
  #1
\end{minipage}}\end{center}
\bigskip
}
% \newcommand{\exemple}[1]{
% \begin{center}\framebox{\begin{minipage}{\textwidth}
%   \parSR{Exemple.}
%   #1
% \end{minipage}}\end{center}
% \bigskip
% }
\newcommand{\exemple}[1]{
\begin{breakbox}
  \parSR{Exemple.}
  #1
\end{breakbox}
\bigskip
}

\newcommand{\SRcorrect}[2]{\textcolor{gray}{#1}\textcolor{blue}{#2}}
\newcommand{\SRcomment}[1]{\textcolor{blue}{[{\sl SR: #1}]}}



% Section numbering
\usepackage{chngcntr}
\renewcommand{\thepart}{\Roman{part}}
% \counterwithout{section}{part}
\setcounter{secnumdepth}{3}
\setcounter{tocdepth}{1}

% Proposition numbering
\renewcommand{\subsubsection}{\section}
% \numberwithin{exercise}{section}
% \numberwithin{equation}{section}

% Suppression des solutions
\renewcommand{\solution}[1]{}

\newcommand{\alglin}{/home/robin/ENSEIGN/Cours/MathBiologie/L3-ENS-Math1/Exercices/AlgLin}
\newcommand{\multivar}{/home/robin/ENSEIGN/Cours/MathBiologie/L3-ENS-Math1/Exercices/MultiVar}
\newcommand{\equadiff}{/home/robin/ENSEIGN/Cours/MathBiologie/L3-ENS-Math1/Exercices/EquaDiff}
\newcommand{\probas}{/home/robin/ENSEIGN/Cours/MathBiologie/L3-ENS-Math1/Exercices/Probas}


% %-------------------------------------------------------------------------------
% %-------------------------------------------------------------------------------
% \title{\Huge{Ce qu'un biologiste doit savoir en mathématiques}}
% \author{SR d'après \cite{Lam20} : Annexes}
% \date{\today}


\title{\normalsize{\sc École Normale Supérieure de Paris, Licence de Biologie L3}
  
  \bigskip
  \normalsize{\sc Année 2025–26}
  
  \bigskip
  \large{\bf Mathématiques I : ce qu’un biologiste ne doit pas ignorer} 
  
}
\date{Examen, le 16 janvier 2026}

%-------------------------------------------------------------------------------
%-------------------------------------------------------------------------------
\begin{document}
%-------------------------------------------------------------------------------
%-------------------------------------------------------------------------------

\maketitle

\bigskip
L'épreuve dure 2 heures. 
Les notes de cours individuelles sont autorisées.
L’usage de tout appareil électronique est interdit à l’exception d’une calculatrice.

%-------------------------------------------------------------------------------
% \section{Algèbre linéaire}
%-------------------------------------------------------------------------------
\subsubsection{Polynôme caractéristique d'une matrice paramétrée}
%-------------------------------------------------------------------------------

On considère la matrice
\begin{align*}
  B & = \left[\begin{array}{rrr}
    -1 & 3 & 1 \\ 0 & 2 & 1 \\ \alpha & \beta & 2
    \end{array}\right].  
\end{align*}

\begin{enumerate}
  \item Montrer son polynôme caractéristique est
  $$
  P_B(\lambda) = -(1+\lambda)(\lambda^2 - 4 \lambda + 4 - \alpha - \beta).
  $$
  \solution{
  On développe par rapport à la première colonne et on développe : 
  \begin{align*}
    P_B(\lambda) 
    & = |B - \lambda I_3|
    = (-1 -\lambda) \left|\begin{array}{cc} 2 - \lambda & 1 \\ \beta & 2 - \lambda \end{array}\right| 
    - \alpha \left|\begin{array}{cc} 3 & 1 \\ 2 - \lambda & 1 \end{array}\right| \\
    & = (-1 - \lambda) ((2 - \lambda)^2 - \beta) - \alpha (3 - (2 - \lambda)) \\
    & = - (1 + \lambda) (\lambda^2 - 4 \lambda + 4 - \beta) - \alpha (1 + \lambda) \\
    & = - (1 + \lambda) (\lambda^2 - 4 \lambda + 4 - \beta + \alpha).
  \end{align*}
  }
  \item En déduire une condition sur $\alpha$ et $\beta$ pour que $B$ possède trois valeurs propres réelles distinctes.
  \solution{La premièrre valeur propre est donc $\lambda_1 = -1$. Le discriminant du polynôme 
  $$
  Q(\lambda) = \lambda^2 - 4 \lambda + 4 - (\alpha+\beta)
  $$
  vaut $\Delta = 16 - 4(4 - (\alpha+\beta)) = \alpha + \beta$. Les deux valeurs propres restantes $\lambda_2$ et $\lambda_3$ sont donc réelles et distinctes ssi $\alpha + \beta > 0$ et valent alors
  $$
  \lambda_2 = \frac{4 + \sqrt{\alpha + \beta}}{2} > 0 
  \qquad \text{et} \qquad
  \lambda_3 = \frac{4 - \sqrt{\alpha + \beta}}{2}.
  $$
  Elles sont de plus distinctes de $\lambda_1 = -1$ ssi 
  $$
  \lambda_3 \neq -1 \qquad \Leftrightarrow \qquad
  \alpha + \beta \neq 36.
  $$
  Dans ce cas ($\alpha + \beta > 0$ et $\alpha + \beta \neq 36$), les 3 valeurs propres sont réelles et distinctes (et $B$ est donc diagonalisable). 
  }
  %
%   \item En déduire une condition sur $\alpha$ et $\beta$ pour que $B$ soit diagonalisable.
%   \solution{La premièrre valeur propre est donc $\lambda_1 = -1$. Le discriminant du polynôme 
%   $$
%   Q(\lambda) = \lambda^2 - 4 \lambda + 4 - (\alpha+\beta)
%   $$
%   vaut $\Delta = 16 - 4(4 - (\alpha+\beta)) = \alpha + \beta$. Les deux valeurs propres restantes $\lambda_2$ et $\lambda_3$ sont donc réelles et distinctes ssi $\alpha + \beta > 0$ et valent alors
%   $$
%   \lambda_2 = \frac{4 + \sqrt{\alpha + \beta}}{2} > 0 
%   \qquad \text{et} \qquad
%   \lambda_3 = \frac{4 - \sqrt{\alpha + \beta}}{2}.
%   $$
%   Elles sont de plus distinctes de $\lambda_1 = -1$ ssi 
%   $$
%   \lambda_3 \neq -1 \qquad \Leftrightarrow \qquad
%   \alpha + \beta \neq 36.
%   $$
%   Dans ce cas ($\alpha + \beta > 0$ et $\alpha + \beta \neq 36$), les 3 valeurs propres sont réelles et distinctes et $B$ est donc diagonalisable. \\
%   Reste à étudier les cas 
%   \begin{description}
%    \item[$\alpha + \beta = 0 \Leftrightarrow \lambda_2 = \lambda_3 = 2$:] \todo{déterminer la dimension du sous-espace propre associé.} 
%    \item[$\alpha + \beta = 36 \Leftrightarrow \lambda_1 = \lambda_3 = -1$:] \todo{déterminer la dimension du sous-espace propre associé.} 
%   \end{description}}
\end{enumerate}


%-------------------------------------------------------------------------------
% \section{Fonction de plusieurs variables}
%-------------------------------------------------------------------------------
\subsubsection{Application linéaire tangente à une forme quadratique} 
%-------------------------------------------------------------------------------

On considère une matrice $A \in \Mcal_n$ symétrique, un vecteur $v \in \Rbb^n$ et la fonction 
$$
\begin{array}{rlll}
  f : & \Rbb^n & \mapsto & \Rbb \\
  & x & \to & f(x) = x^\top A x + v^\top x.
\end{array}
$$

\begin{enumerate}
  \item Montrer qu'il existe un vecteur $g(x) \in \Rbb^n$, qu'on précisera, tel que l'application linéaire tangente à $f$ en $x$ s'écrit
  $$
  \begin{array}{rlll}
    D_xf : & \Rbb^n & \mapsto & \Rbb^n \\
    & h & \to & D_xf(h) = g(x)^\top h.
  \end{array}
  $$
  \solution{On écrit
  \begin{align*}
    f(x+h) 
    & = (x+h)^\top A (x+h) + v^\top (x+h)
    = f(x) + x^\top A h + h^\top A x + h^\top A h + v^\top h \\
    & = f(x) + (2 x^\top A + v^\top) h + h^\top A h
  \end{align*}
  puisque $x^\top A h = h^\top A x$. On remarque alors que $h^\top A h = o(\|h\|)$ pour conclure que, puisque $A$ est symétrique, l'application linéaire tangent $D_x f$ s'écrit bien
  $$
  D_xf(h) = g(x)^\top h
  \qquad \text{avec} \quad
  g(x) = 2 A x + v.
  $$}
  \item En supposant que $A$ est inversible, déterminer le point stationnaire $x^*$ où $g(x)$ s'annule.
  \solution{En supposant $A$ inversible, on a
  $$
  g(x^*) = 0
  \qquad \Leftrightarrow \qquad
  2 A x^* + v = 0
  \qquad \Leftrightarrow \qquad
  x^* = - \frac12 A^{-1} v.
  $$}
  \item Donner une condition sur$A$ pour que $A$ soit un minimum local. (On pourra calculer la matrice hessienne de $A$.)
  \solution{La matrice hessienne de l'application $f$ en tout point $x$ vaut $H_x = 2 A$ (il suffit de déterminer l'application linéaire tangente à $g(x)$).
  $x^*$ est donc un minimum ssi $A$ est strictement définie négative}
  \item Discuter l'utilité de l'hypothèse selon laquelle $A$ est symétrique.
  \solution{On peut décomposer $A$ en ses parties symétrique $S$ et anti-symétrique $T$ : 
  $$
  S = \frac12(A + A^\top), \qquad 
  T = \frac12(A - A^\top), \qquad 
  \Rightarrow \quad
  A = S + T
  $$
  et remarquer que
  $$
  f(x) 
  = x^\top A x + v^\top x
  = x^\top S x + \frac12 \underset{=0}{\underbrace{(x^\top A x - x^\top A^\top x)}} + v^\top x,  
  $$
  c'est-à-dire que seulle la partie symétrique de $A$ contribue à la fonction Ode $f$.}
\end{enumerate}




%-------------------------------------------------------------------------------
% \section{Systèmes dynamiques}
%-------------------------------------------------------------------------------
\subsubsection{Chemostat}
%-------------------------------------------------------------------------------

Voir \url{https://umr5558-shiny.univ-lyon1.fr/web/}


%-------------------------------------------------------------------------------
% \section{Probabilités}
%-------------------------------------------------------------------------------
\subsubsection{Processus Galton–Watson géométrique avec excès de zéro}
%-------------------------------------------------------------------------------

Les organismes d’une espèce asexuée se reproduisent suivant un processus de Galton–Watson, où chaque individu laisse à la génération suivante un nombre aléatoire $X$ d’individus. \\
On note $p_k = \Pr\{X = k\}$ et on suppose que 
$$
p_0 = b
\qquad \text{et} \qquad 
p_k = (1 - b) (1 - a) a^{k-1} \text{ pour } k \geq 1.
$$
\begin{enumerate}
  \item Montrer que la fonction génératrice de $X$ vaut
  $$
  f(s) := \Esp(s^X) = b + (1-b)(1-a) \frac{s}{1 - as}.
  $$
  \solution{Par définition, on a
  \begin{align*}
    f(s) 
    & =\Esp(s^X) = \sum_{k \geq 0} p_k s^k 
    = b + (1-b) \sum_{\textcolor{red}{k \geq 1}} (1 - a) a^{k-1} s^k \\
    & = b + (1 - b) s \sum_{k \geq 1} (1 - a) (as)^{k-1}
    = b + (1 - b) s \sum_{k \geq 0} (1 - a) (as)^k
  \end{align*}
  où on reconnaît la fonction génératrice d'une variable aléatoire géométrique : 
  $$
  \sum_{k \geq 0} (1 - a) (as)^k = \frac{1 - a}{1 - as},
  $$
  qui donne le résultat.
  }
  \item Déterminer la valeur de la probabilité, notée $q$, qu'une population de cette espèce issue d’un seul individu fondateur s'éteigne.
  \solution{On sait que $q$ est le plus petit point fixe de la fonction $f$ dans l'intervalle $(0, 1]$. Il faut donc résoudre $f(s) = s$
  \begin{align*}
    \Leftrightarrow \qquad 
    b(1 - as) + (1-b)(1-a) s & = a - as^2 \\
    \Leftrightarrow \qquad
    as^2 - (a+b) s + b & = 0
  \end{align*}
  dont le discriminant vaut $\Delta = (a+b)^2 - 4ab = (a-b)^2$ et les solutions sont donc
  $$
  s = \frac{(a+b) \pm (a-b)}{2}
  \qquad \Leftrightarrow \qquad
  s \in \left\{\frac{b}a, 1\right\}.
  $$
  La probabilité d'extinction est donc $q = b/a$ si $b < a$ et 1 sinon.
  }
  \item Par quel autre moyen peut-on déterminer à quelle condition la probabilité d'extinction est strictement inférieure à 1 ?
  \solution{
  On sait que la probabilité d'extinction est strictement inférieure à 1 si le nombre de descendant moyen $m$ est strictement supérieur à 1. Il suffit donc de calculer $m$ pour la loi $p$. \\
  On rappelle l'espérance d'une loi géométrique : si $Z \sim \Gcal(1-a)$, alors
  $$
  \Esp Z 
  = (1-a) \sum_{z \geq 1} z a^{z-1}
  = (1-a) \; \partial_a \left(\sum_{z \geq 0} a^z\right)
  = (1-a) \; \partial_a \left(\frac1{1-a}\right)
  % = (1-a) \frac1{(1-a)^2}
  = \frac1{1-a}.
  $$
  On a donc $m = \Esp X = b \times 0 + (1-b) \Esp Z = (1-b) / (1-a)$ et on retrouve $m > 1 \; \Leftrightarrow \; b < a$.
  }
\end{enumerate}


\renewcommand{\subsubsection}[1]{\section*{Bonus : #1}}
%-------------------------------------------------------------------------------
\subsubsection{Evolution de plusieurs populations de Galton–Watson}
%-------------------------------------------------------------------------------

On considère une espèce dont les générations sont régies par un processus de Bienaymé-Galton-Watson. On note $f$ la fonction génératrice du nombre de descendants et $q$ la probabilité d'extinction d'une population issue d'un unique individu.

On introduit un nombre aléatoire $N$ d'individus de cette espèce dans une île. On note $g$ la fonction génératrice de $N$. On suppose que les descendances des $N$ individus fondateurs évoluent indépendamment les unes des autres. On s'intéresse à la probabilité $p$ de colonisation, c'est-à-dire à la probabilité pour que cette population ne s'éteigne pas.
\begin{enumerate}
  \item A quelle condition $p$ est-elle non nulle ?
  \solution{La colonisation réussit si la descendance d'au moins un individu fondateur ne s'éteint pas, ce qui se produit avec probabilité $1 - q^N$. $p$ est donc nulle pour tout $N$ si $q = 1$ et non nulle ssi $q < 1$, c'est à dire si $b < a$.}
  \item Exprimer $p$ en fonction de $q$. 
  \solution{$1 - q^N$ est la probabilité de colonisation conditionnelle à l'effectif initial $N$. La probabilité de colonisation vaut donc
  $$
  p = \Esp\left(1 - q^N\right) = 1 - \Esp(q^N) = 1 - g(q).
  $$}
  \item Déterminer $p$, d'une part, si $N = 1$ et, d'autre part, si $N$ est distribué comme $X$. \\
  Commenter.
  \solution{
  \begin{description}
    \item[$N = 1$ :] on a immédiatement $p = 1 - q$. 
    \item[$N \overset{\Delta}{=} X$:] alors $g = f$ et donc
    $$
    p = 1 - g(q) = 1 - f(q) = 1 - q
    $$
    puisque $q$ est un point fixe de $f$ ($f(q) = q$). 
  \end{description}
  L'égalité des résultats n'est pas surprenante puisque le second cas correspond au premier décalé d'une génération.}
\end{enumerate}



%-------------------------------------------------------------------------------
%-------------------------------------------------------------------------------
\end{document}
%-------------------------------------------------------------------------------
%-------------------------------------------------------------------------------


