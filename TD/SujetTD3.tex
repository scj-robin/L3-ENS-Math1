\documentclass[french, 12pt]{article}

%-------------------------------------------------------------------------------
\usepackage[a4paper,top=2cm,bottom=2cm,left=2cm,right=2cm,marginparwidth=1.75cm]{geometry}
\usepackage{amsmath,amsfonts,amssymb,amsthm}
\usepackage[french]{babel}
\usepackage[utf8]{inputenc}
\usepackage[T1]{fontenc}
\usepackage{enumerate}
\usepackage{natbib}
\usepackage{graphicx}
\usepackage{xspace}
\usepackage{color,xcolor}
\usepackage{tikz}
\usepackage{remreset}
\usepackage{url}
\usepackage{boites}
% \usepackage{extsizes} % Permet \documentclass[french, 14pt]{extreport}
% \usepackage[a4paper,top=1cm,bottom=2cm,left=1cm,right=1cm,marginparwidth=.75cm]{geometry}
% \usepackage{minitoc}

\graphicspath{{../Figures/}}

% Environnement
\newtheorem{theorem}{Théorème}
\newtheorem{definition}{Définition}
\newtheorem{lemma}{Lemme}
\newtheorem{proposition}{Proposition}
\newtheorem*{theorem*}{Théorème}
\newtheorem*{definition*}{Définition}
\newtheorem*{proposition*}{Proposition}
\newtheorem*{corollary*}{Corollaire}
\newtheorem*{assumption*}{Hypothèse}
\newtheorem*{algorithm*}{Algorithme}
\newtheorem*{lemma*}{Lemme}
\newtheorem*{remark*}{Remarque}
\newtheorem*{exercise*}{Exercice}
\newtheorem{exercise}{Exercice}
\newcommand{\remark}{\bigskip\noindent\textbf{\textsl{Remarque.}}\xspace}
\newcommand{\remarks}{\bigskip\noindent\textbf{\textsl{Remarques.}}\xspace}
\newcommand{\parSR}[1]{\paragraph*{\textsl{#1}}\xspace}
\renewcommand{\proof}{\bigskip\noindent\underline{\textsl{Démonstration}.}\xspace}
\newcommand{\eproof}{$\blacksquare$}

% Effets, couleurs
\newcommand{\emphase}[1]{\textcolor{red}{#1}}
\newcommand{\demoProp}[1]{\noindent{\textbf{\textsl{Démonstration de la proposition \ref{#1} :}}}}
\newcommand{\itemdot}{\textbullet}

% Moments
\DeclareMathOperator{\Esp}{\mathbb{E}}
\DeclareMathOperator{\diag}{diag}
\DeclareMathOperator{\Cov}{\mathbb{C}ov}
\DeclareMathOperator{\tr}{tr}
\DeclareMathOperator{\Var}{\mathbb{V}}
\let\Pr\relax\DeclareMathOperator{\Pr}{\mathbb{P}}
\renewcommand{\d}{\text{d}}

% R, N, ...
\newcommand{\cst}{\text{cst}}
\newcommand{\Cbb}{\mathbb{C}}
\newcommand{\Ibb}{\mathbb{I}}
\newcommand{\Nbb}{\mathbb{N}}
\newcommand{\Rbb}{\mathbb{R}}
\newcommand{\Zbb}{\mathbb{Z}}

% Indicateurs

% Lois et ensembles
\newcommand{\Acal}{\mathcal{A}}
\newcommand{\Bcal}{\mathcal{B}}
\newcommand{\Ccal}{\mathcal{C}}
\newcommand{\Ecal}{\mathcal{E}}
\newcommand{\Gcal}{\mathcal{G}}
\newcommand{\Ical}{\mathcal{I}}
\newcommand{\Lcal}{\mathcal{L}}
\newcommand{\Mcal}{\mathcal{M}}
\newcommand{\Ncal}{\mathcal{N}}
\newcommand{\Pcal}{\mathcal{P}}
\newcommand{\Rcal}{\mathcal{R}}
\newcommand{\Scal}{\mathcal{S}}
\newcommand{\Ucal}{\mathcal{U}}
\newcommand{\Xcal}{\mathcal{X}}
\newcommand{\Ycal}{\mathcal{Y}}

% Comments
\newcommand{\SR}[2]{\textcolor{gray}{#1}\textcolor{red}{#2}}
\newcommand{\todo}[1]{\textcolor{red}{\`A faire~: {\sl #1}}}
\newcommand{\dessin}[1]{
\begin{center}\framebox{\begin{minipage}{\textwidth}
  \textcolor{purple}{#1}
\end{minipage}}\end{center}
\bigskip
}
\newcommand{\progres}[1]{
\begin{center}\framebox{\begin{minipage}{\textwidth}
  \textcolor{blue}{{\sl #1}}
\end{minipage}}\end{center}
\bigskip
}
\newcommand{\solution}[1]{
\begin{center}\framebox{\begin{minipage}{\textwidth}
  \noindent{\sl Solution :}
  #1
\end{minipage}}\end{center}
\bigskip
}
% \newcommand{\exemple}[1]{
% \begin{center}\framebox{\begin{minipage}{\textwidth}
%   \parSR{Exemple.}
%   #1
% \end{minipage}}\end{center}
% \bigskip
% }
\newcommand{\exemple}[1]{
\begin{breakbox}
  \parSR{Exemple.}
  #1
\end{breakbox}
\bigskip
}

\newcommand{\SRcorrect}[2]{\textcolor{gray}{#1}\textcolor{blue}{#2}}
\newcommand{\SRcomment}[1]{\textcolor{blue}{[{\sl SR: #1}]}}



% Proposition numbering
% \numberwithin{proposition}{section}
\numberwithin{exercise}{section}
\numberwithin{equation}{section}

% Suppression des solutions
\renewcommand{\solution}[1]{}

%-------------------------------------------------------------------------------
%-------------------------------------------------------------------------------
\begin{document}
%-------------------------------------------------------------------------------
%-------------------------------------------------------------------------------

\begin{center}
  \small{\sc \'Ecole normale supérieure de Paris, Licence de Biologie L3, Année 2022-23} \\
  \bigskip
  \large{\bf Ce qu'un biologiste doit savoir en mathématiques} \\
  \bigskip  
  {\bf TD n° 3}
\end{center}

%-------------------------------------------------------------------------------
%-------------------------------------------------------------------------------
\section{Processus markoviens} 
\newcommand{\probas}{/home/robin/ENSEIGN/Cours/MathBiologie/L3-ENS-Math1/Exercices/Probas}
%-------------------------------------------------------------------------------

%-------------------------------------------------------------------------------
\subsection{Chaîne de Markov}
%-------------------------------------------------------------------------------

%-------------------------------------------------------------------------------
\subsubsection{Exemple de chaîne de Markov}
%-------------------------------------------------------------------------------

  On considère une chaîne de Markov $(X_n)_{n \geq 0}$ prenant ses valeurs dans $\{1, 2, 3, 4\}$, et on note $p(i, j)$ la probabilité de transition de $i$ à $j$. On suppose que $p(1, 1) = p(4, 4) = 1$, que $p(2, 3) = 1 - p(2, 1) =: p$ et que $p(3, 2) = 1 - p(3, 4) =: q$.
  \begin{enumerate}
   \item Donner la matrice de transition de cette chaîne de Markov.
   \item Quelles sont les classes de communication de cette chaîne ? Donner leur nature. Quels sont les comportements asymptotiques possibles de cette chaîne ?
   \item On appelle $x_i$ la probabilité que la chaîne soit absorbée en 1 sachant que $X_0 = i$. Donner $x_1$ et $x_4$, ainsi que deux équations reliant $x_2$ et $x_3$, puis les calculer.
  \end{enumerate}

\solution{
  \begin{enumerate}
    \item La matrice de transition est
    $$
    P = \left( \begin{array}{cccc}
                1 & 0 & 0 & 0 \\
                1-p & 0 & p & 0 \\
                0 & q & 0 & 1-q \\
                0 & 0 & 0 & 1
              \end{array} \right)
    $$
    \item Les classes de communications sont 
    $$
    C_1 = \{1\}, \qquad C_2 = \{2, 3\}, \qquad C_3 = \{4\}.
    $$
    $C_1$ et $C_3$ sont absorbantes, $C_2$ est transiente. \\
    La chaîne finit nécessairement par être absorbé en 1 ou 4.
    \item D'après la question précédente, on a 
    $$
    x_1 = 1, \qquad x_4 = 0.
    $$
    Pour être absorbé en 1 partant de 3, la chaîne doit d'abord passer de 3 en 2 (avec probabilité $q$), puis être absorbée en 1 partant de 2, donc
    $$
    x_3 = q x_2.
    $$
    Partant de 2 la chaîne peut être immédiatement absorbée en 1 (avec probabilité $1-p$) ou passer en 3 (avec probabilité $p$), puis être absorbée en 1 depuis 3, donc
    $$
    x_2 = (1 - p) + p x_3.
    $$
    $x_2$ vérifie donc
    $$
    x_2 = (1 - p) + p q x_2 
    \qquad \Leftrightarrow \qquad x_2 = \frac{1-p}{1 - pq}
    \qquad \Rightarrow \qquad x_3 = \frac{q(1-p)}{1 - pq}.
    $$
    On peut aussi calculer directement $x_2$ en définissant le temps d'atteinte de l'état $1$ depuis l'état $2$:
    $$
    T_{21} = \inf\{t: X_t = 1 \mid X_0 = 2\}.
    $$
    On remarque alors que, pour $k \geq 0$,  
    $$
    \Pr\{T_{21} = 2k\} = 0, \qquad \Pr\{T_{21} = 2k + 1\} = (1-p) \; (pq)^k
    $$
    et on en déduit
    $$
    x_2 = \sum_{n \geq 0} \Pr\{T_{21} = n\} = \sum_{k \geq 0} \Pr\{T_{21} = 2k+1\} 
    = (1-p) \sum_{k \geq 0} (pq)^k = \frac{1-p}{1 - pq}.
    $$
    De même, on peut déterminer $x_3$ directement à partir de la loi du temps d'atteinte $T_{31}$.
  \end{enumerate}
}



%-------------------------------------------------------------------------------
\subsection{Processus de branchement}
%-------------------------------------------------------------------------------

%-------------------------------------------------------------------------------
\subsubsection{Processus de branchement} %-------------------------------------------------------------------------------
  On considère un processus de Bienaymé-Galton-Watson partant d'une population de taille 1 et dans laquelle le nombre de descendants par individu suit une loi géométrique $\Gcal(a)$ (pour mémoire, $X \sim \Gcal(a)$ ssi $\Pr\{X = k\} = (1-a) a^k$).
  \begin{enumerate}
    \item Calculer le nombre moyen $m$ de descendants par individus en fonction de $a$.
    \solution{On a
    $$
    m = \Esp(X) 
    = \sum_{k \geq 0} (1-a) \; a^k \; k 
    = a(1-a) \sum_{k \geq 0} k \; a^{k-1}
    $$
    où on reconnaît la dérivée de la fonction $f(a) = \sum_{k \geq 0} a^{k} = (1-a)^{-1}$, soit $f'(a) = (1-a)^{-2}$. On a donc
    $$
    m
    = a(1-a) (1-a)^{-2}
    = a / (1-a)
    $$
    (qui est une fonction monotone croissante de $a$).}
    \item Pour quelles valeurs de $a$ le processus est-il critique, sous-critique ou sur-critique ?
    \solution{On a $m \leq 1 \; \Leftrightarrow \; a \leq 1-a \; \Leftrightarrow \; a \leq 1/2$.  Le processus est donc critique pour $a = 1/2$, sous critique pour $a < 1/2$ et sur-critique pour $a > 1/2$.}
    \item Montrer que, pour $m > 1$, la probabilité d'extinction $q$ de cette population vaut $1/m$.
    \solution{On sait que la probabilité d'extinction $q$ partant d'une population de taille 1 est le plus petit point fixe de l'équation $s = f_X(s)$ où $f_X$ est la fonction génératrice des probabilités de la loi géométrique : 
    $$
    f_X(s) = (1 - a) / (1 - as).
    $$
    On cherche donc à résoudre $1 - a = (1 - as) s \; \Leftrightarrow \; as^2 - s + (1-a) = 0$ dont on sait que $s = 1$ est solution. On peut donc factoriser
    $$
    as^2 - s + (1-a) = a (s-1) \left(s  - \frac{1-a}a\right)
    $$
    où $(1-a)/a < 1$ dès que $a > 1/2$. La probabilité d'extinction vaut donc
    $$
    q = \frac{1-a}a = \frac1m.
    $$}
  \end{enumerate}


\input{\probas/BGWmultipop}

%-------------------------------------------------------------------------------
%-------------------------------------------------------------------------------
\end{document}
%-------------------------------------------------------------------------------
%-------------------------------------------------------------------------------


