%-------------------------------------------------------------------------------
\subsubsection{Analyse en composantes principales} 
%-------------------------------------------------------------------------------

On considère $n$ variables aléatoires $X_1, \dots X_n$ réelles de matrice de covariance $\Sigma$. On note $X$ le vecteur aléatoire $X = [X_1,  \dots X_n]^\top$. On cherche à déterminer la combinaison linéaire 
$$
Y = \sum_{i=1}^n a_i X_i = a^\top X
$$ 
de plus grande variance, où $a = [a_1, \dots a_n]^\top \in \Rbb^n$.

\begin{enumerate}
  \item Montrer que, pour tout réel $k$ et $b = k a$, on a $\Var(b^\top X) = k^2 \Var(a^\top X)$ et justifier que, dans la suite, on suppose que  $\|a\| = 1$.
  \solution{Soit $Z = b^\top X = k a^\top X = k Y$, on a $\Var(Z) = k^2 \Var(Y)$, d'où le résultat. \\
  La recherche de la combinaison ``de plus grande variance'' n'a donc pas de sens si on impose pas de contrainte au vecteur $a$.
  }
  \item Montrer qu'on peut décomposer $\Sigma$ sous la forme $\Sigma = P \Lambda P^\top$ et que, pour tout vecteur $a$ de norme 1, il existe un vecteur $b$, de norme 1 également, que l'on précisera, et tel que
  $$
  a^\top \Sigma a = \sum_{i=1}^n \lambda_i b_i^2.
  $$
  \solution{$\Sigma$ étant symétrique, elle est diagonalisable et ses vecteurs propres sont orthogonaux, on peut donc la décomposer sous la forme indiquée, où $P$ est une matrice orthonormale. \\
  Pour tout vecteur $a$ de norme 1, on a donc
  $$
  a^\top \Sigma a = a^\top P \Lambda P^\top a = b^\top \Lambda b = \sum_i \lambda_i b_i^2
  $$
  en posant $b = P^\top a$, qui vérifie bien
  $$
  \|b\|^2 = b^\top b = a^\top P P^\top a = a^\top a = 1.
  $$
  }
  \item Déterminer le vecteur $b^*$ de norme 1 qui maximise $\sum_{i=1}^n \lambda_i b_i^2$.
  \solution{Puisque $\Sigma$ est une matrice de covariance, elle est définie positive, donc toute ses valeurs propres sont réelles et positives ou nulles. \\
  En notant $\lambda_1$ sa plus grande valeur propre, pour tout $b$ de norme $1$, on a
  $$
  \sum_i \lambda_i b_i^2 \leq \lambda_1 \sum_i b_i^2 = \lambda_1 \|b\|^2 = \lambda_1.
  $$
  Cette borne supérieure est atteinte uniquement pour le vecteur $b^* = [1 \; 0 \; \dots \; 0]^\top$.
  }
  \item En déduire la combinaison linéaire $Y^* = {a^*}^\top X$ de variance maximale (pour $\|a^*\| = 1$) et donner sa variance.
  \solution{Puisque $\Var(Y) = a^\top \Sigma a$ et que pour tout $a$, $a^\top \Sigma a = \sum_i \lambda_i b_i^2$ en prenant 
  $$
  b = P^\top a \qquad \Leftrightarrow \qquad a = P b,
  $$
  la variance $\Var(a^\top X)$ maximale est atteinte pour
  $$
  a^* = P b^* = v_1
  $$
  en notant $P = [v_1 v_2 \dots v_n]$. \\
  $a^*$ est donc le premier vecteur propre de $\Sigma$, c'est-à-dire le vecteur propre associé à $\lambda_1$. La variance de $Y$ est alors
  $$
  \Var(Y) = a^\top \Sigma a = a^\top \lambda_1 a = \lambda_1.
  $$
  \remarks
  \begin{itemize}
  \item Si $\lambda_1$ est une valeur propre multiple, la solution du problème n'est pas unique.
  \item Si toutes les valeurs propres sont simples, la combinaison linéaire $Y' = a'^\top X$ de plus grande variance (pour $a'$ de norme 1) et non corrélée à $Y^* = a^* X$ ($\Cov(Y', Y^*) = 0$) est donnée par $a' = v_2$, le deuxième vecteur propre de $\Sigma$.
  \end{itemize}
  }
\end{enumerate}

\bigskip
\noindent On considère maintenant 
\begin{itemize}
  \item $n$ variables aléatoires $V_1, \dots V_n$ indépendantes et de variance $1$ et
  \item une variable $U$, indépendante des $\{V_i\}_{1 \leq i \leq n}$ et de variance $\sigma^2$,  On 
\end{itemize}
et on pose, pour $1 \leq i \leq n$,
$$
X_i = U + V_i.
$$

\begin{enumerate}
  \setcounter{enumi}{4}
  \item Déterminer la matrice de covariance $\Sigma$ de $(X_1, \dots X_n)$.
  \solution{On a
  \begin{align*}
    \Var(X_i) & = \Var(U) + \Var(V_i) = \sigma^2 + 1, \\
    \Cov(X_i, X_j) & = \Cov(U+V_i, U+V_j) = \Cov(U, U) = \sigma^2, & & \text{pour $i \neq j$},
  \end{align*}
  soit
  $$
  \Sigma 
  = \left[\begin{array}{cccc}
    \sigma^2 + 1 & \sigma^2 & \dots & \sigma^2 \\
    \sigma^2 & \ddots & \ddots & \vdots \\
    \vdots & \ddots & \ddots & \sigma^2 \\
    \sigma^2 & \dots & \sigma^2 & \sigma^2 + 1
  \end{array}\right]
  = \sigma^2 J + I 
  $$
  }
  \item Déterminer la combinaison linéaire $Y = \sum_{i=1}^n a_i X_i$ de plus grande variance et donner sa variance.
  \solution{Il s'agit de déterminer les valeurs et vecteurs propres de $\Sigma$. 
  Puisque $J = 1_n 1_n^\top$, on a, pour tout vecteur $v$
  $$
  \Sigma v = \lambda v 
  \qquad \Leftrightarrow \qquad 
  \sigma^2 1_n 1_n^\top v  + v = \lambda v.
  $$
  On peut distinguer deux cas.
  \begin{description}
    \item[$1_n^\top v = 0$:] on a alors $v = \lambda v$, soit $\lambda = 1$. Tout vecteur $v$ tel que $1_n^\top v = \sum_i v_i = 0$ est donc propre associé à $\lambda = 1$, ce qui définit un sous espace propre de dimension $n-1$. 
    \item[$1_n^\top v \neq 0$:] pour $v = 1_n$ (donc $1_n^\top v = n$), on a
    $$
    \Sigma v = n \sigma^2 v + v = \lambda v,
    $$
    soit $\lambda = n \sigma^2 + 1$, associée au vecteur propre $v = 1_n$.
  \end{description}
  La valeur propre dominante est $\lambda_1 = n \sigma^2 + 1$ et le vecteur propre de norme 1 associé est $a^* = 1/\sqrt{n} 1_n$, soit
  $$
  Y^* 
  = \frac1{\sqrt{n}} \sum_i X_i
  $$
  dont la variance est donnée par la valeur propre dominante $\lambda_1$, soit
  $$
  \Var(Y^*) = n \sigma^2 + 1.
  $$
  }
\end{enumerate}


