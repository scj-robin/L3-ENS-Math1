%-------------------------------------------------------------------------------
\subsubsection{Fonction de $\Rbb^2$ dans $\Rbb$}
%-------------------------------------------------------------------------------

Soit la fonction 
$$
\begin{array}{rrcl}
  f : & \Rbb^2 & \mapsto & \Rbb \\
  & (x, y) & \to & 2 x^3 + 6 x y - 3 y^2 + 2
\end{array}
$$

\begin{enumerate}
  \item Déterminer les points stationnaires de $f$.
  \solution{
    Le vecteur gradient au point $(x, y)$ est
    $$
    \nabla_{x, y}f = \left[
        \begin{array}{c} 6 x^2 + 6 y  \\ 6 x - 6 y \end{array}
      \right]
    $$
    s'annule ssi $x = y$ et $x^2 = -y$, c'est à dire aux points $A = (0, 0)$ et $B(-1, -1)$.
  }
  \item Déterminer la nature de chaque point stationnaire.
  \solution{
    La matrice hessienne au point $(x, y)$ est
    $$
    \nabla^2_{x, y}f = \left[
        \begin{array}{cc} 12 x & 6 \\ 6 & -6 \end{array}
      \right].
    $$
    Elle est symétrique, donc diagonalisable ; son déterminant vaut $|\nabla^2_{x, y}f| = - 36 (2x + 1)$ et sa trace vaut $\tr(\nabla^2_{x, y}) = 12x - 6$. 
    \begin{itemize}
    \item Au point $A = (0, 0)$, on a $|\nabla^2_{x, y}f| = - 36$, les deux valeurs propres sont donc de signe opposé, et $A$ est donc un point selle.
    \item Au point $B = (-1, -1)$, on a $|\nabla^2_{x, y}f| = 36$, donc les deux valeurs propres sont de même signe et, puisque $\tr(\nabla^2_{x, y}) = -18$, elles sont toutes les deux négatives, donc $B$ est un maximum.
    \end{itemize}
  }
  \item La fonction $f$ admet-elle un extremum (maximum ou minimum) global ?
  \solution{
    Le seul maximum (local) est le point $B$ où la fonction $f$ faut $-2 + 6 - 3 + 2 = 3$. Hors $\lim_{x \to +\infty} f(x, 0) = +\infty$, donc $B$ n'est pas un maximum global et $f$ n'admet donc pas d'extremum global.
  }
\end{enumerate}
