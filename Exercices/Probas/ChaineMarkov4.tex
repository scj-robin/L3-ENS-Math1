%-------------------------------------------------------------------------------
\subsubsection{Chaîne de Markov à 4 états}
%-------------------------------------------------------------------------------

On considère une chaîne de Markov $(X_n)_{n \geq 0}$ prenant ses valeurs dans $\{1, 2, 3, 4\}$, et on note $p(i, j)$ la probabilité de transition de $i$ à $j$. On suppose que $p(1, 1) = p(4, 4) = 1$, que $p(2, 3) = 1 - p(2, 1) =: p$ et que $p(3, 2) = 1 - p(3, 4) =: q$.
\begin{enumerate}
  \item Donner la matrice de transition de cette chaîne de Markov.
  \solution{La matrice de transition est
  $$
  P = \left( \begin{array}{cccc}
              1 & 0 & 0 & 0 \\
              1-p & 0 & p & 0 \\
              0 & q & 0 & 1-q \\
              0 & 0 & 0 & 1
            \end{array} \right)
  $$}
  \item Quelles sont les classes de communication de cette chaîne ? Donner leur nature. Quels sont les comportements asymptotiques possibles de cette chaîne ?
  \solution{Les classes de communications sont 
  $$
  C_1 = \{1\}, \qquad C_2 = \{2, 3\}, \qquad C_3 = \{4\}.
  $$
  $C_1$ et $C_3$ sont absorbantes, $C_2$ est transiente. \\
  La chaîne finit nécessairement par être absorbé en 1 ou 4.}
  \item On appelle $x_i$ la probabilité que la chaîne soit absorbée en 1 sachant que $X_0 = i$. Donner $x_1$ et $x_4$, ainsi que deux équations reliant $x_2$ et $x_3$, puis les calculer.
  \solution{D'après la question précédente, on a 
  $$
  x_1 = 1, \qquad x_4 = 0.
  $$
  Pour être absorbé en 1 partant de 3, la chaîne doit d'abord passer de 3 en 2 (avec probabilité $q$), puis être absorbée en 1 partant de 2, donc
  $$
  x_3 = q x_2.
  $$
  Partant de 2 la chaîne peut être immédiatement absorbée en 1 (avec probabilité $1-p$) ou passer en 3 (avec probabilité $p$), puis être absorbée en 1 depuis 3, donc
  $$
  x_2 = (1 - p) + p x_3.
  $$
  $x_2$ vérifie donc
  $$
  x_2 = (1 - p) + p q x_2 
  \qquad \Leftrightarrow \qquad x_2 = \frac{1-p}{1 - pq}
  \qquad \Rightarrow \qquad x_3 = \frac{q(1-p)}{1 - pq}.
  $$
  On peut aussi calculer directement $x_2$ en définissant le temps d'atteinte de l'état $1$ depuis l'état $2$:
  $$
  T_{21} = \inf\{t: X_t = 1 \mid X_0 = 2\}.
  $$
  On remarque alors que, pour $k \geq 0$,  
  $$
  \Pr\{T_{21} = 2k\} = 0, \qquad \Pr\{T_{21} = 2k + 1\} = (1-p) \; (pq)^k
  $$
  et on en déduit
  $$
  x_2 = \sum_{n \geq 0} \Pr\{T_{21} = n\} = \sum_{k \geq 0} \Pr\{T_{21} = 2k+1\} 
  = (1-p) \sum_{k \geq 0} (pq)^k = \frac{1-p}{1 - pq}.
  $$
  De même, on peut déterminer $x_3$ directement à partir de la loi du temps d'atteinte $T_{31}$.}
\end{enumerate}
