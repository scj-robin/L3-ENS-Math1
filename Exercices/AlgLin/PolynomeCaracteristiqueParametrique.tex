%-------------------------------------------------------------------------------
\subsubsection{Polynôme caractéristique d'une matrice paramétrée}
%-------------------------------------------------------------------------------

On considère la matrice
\begin{align*}
  B & = \left[\begin{array}{rrr}
    -1 & 3 & 1 \\ 0 & 2 & 1 \\ \alpha & \beta & 2
    \end{array}\right].  
\end{align*}

\begin{enumerate}
  \item Montrer son polynôme caractéristique est
  $$
  P_B(\lambda) = -(1+\lambda)(\lambda^2 - 4 \lambda + 4 - \alpha - \beta).
  $$
  \solution{
  On développe par rapport à la première colonne et on développe : 
  \begin{align*}
    P_B(\lambda) 
    & = |B - \lambda I_3|
    = (-1 -\lambda) \left|\begin{array}{cc} 2 - \lambda & 1 \\ \beta & 2 - \lambda \end{array}\right| 
    - \alpha \left|\begin{array}{cc} 3 & 1 \\ 2 - \lambda & 1 \end{array}\right| \\
    & = (-1 - \lambda) ((2 - \lambda)^2 - \beta) - \alpha (3 - (2 - \lambda)) \\
    & = - (1 + \lambda) (\lambda^2 - 4 \lambda + 4 - \beta) - \alpha (1 + \lambda) \\
    & = - (1 + \lambda) (\lambda^2 - 4 \lambda + 4 - \beta + \alpha).
  \end{align*}
  }
  \item En déduire une condition sur $\alpha$ et $\beta$ pour que $B$ possède trois valeurs propres réelles distinctes.
  \solution{La premièrre valeur propre est donc $\lambda_1 = -1$. Le discriminant du polynôme 
  $$
  Q(\lambda) = \lambda^2 - 4 \lambda + 4 - (\alpha+\beta)
  $$
  vaut $\Delta = 16 - 4(4 - (\alpha+\beta)) = \alpha + \beta$. Les deux valeurs propres restantes $\lambda_2$ et $\lambda_3$ sont donc réelles et distinctes ssi $\alpha + \beta > 0$ et valent alors
  $$
  \lambda_2 = \frac{4 + \sqrt{\alpha + \beta}}{2} > 0 
  \qquad \text{et} \qquad
  \lambda_3 = \frac{4 - \sqrt{\alpha + \beta}}{2}.
  $$
  Elles sont de plus distinctes de $\lambda_1 = -1$ ssi 
  $$
  \lambda_3 \neq -1 \qquad \Leftrightarrow \qquad
  \alpha + \beta \neq 36.
  $$
  Dans ce cas ($\alpha + \beta > 0$ et $\alpha + \beta \neq 36$), les 3 valeurs propres sont réelles et distinctes (et $B$ est donc diagonalisable). 
  }
  %
%   \item En déduire une condition sur $\alpha$ et $\beta$ pour que $B$ soit diagonalisable.
%   \solution{La premièrre valeur propre est donc $\lambda_1 = -1$. Le discriminant du polynôme 
%   $$
%   Q(\lambda) = \lambda^2 - 4 \lambda + 4 - (\alpha+\beta)
%   $$
%   vaut $\Delta = 16 - 4(4 - (\alpha+\beta)) = \alpha + \beta$. Les deux valeurs propres restantes $\lambda_2$ et $\lambda_3$ sont donc réelles et distinctes ssi $\alpha + \beta > 0$ et valent alors
%   $$
%   \lambda_2 = \frac{4 + \sqrt{\alpha + \beta}}{2} > 0 
%   \qquad \text{et} \qquad
%   \lambda_3 = \frac{4 - \sqrt{\alpha + \beta}}{2}.
%   $$
%   Elles sont de plus distinctes de $\lambda_1 = -1$ ssi 
%   $$
%   \lambda_3 \neq -1 \qquad \Leftrightarrow \qquad
%   \alpha + \beta \neq 36.
%   $$
%   Dans ce cas ($\alpha + \beta > 0$ et $\alpha + \beta \neq 36$), les 3 valeurs propres sont réelles et distinctes et $B$ est donc diagonalisable. \\
%   Reste à étudier les cas 
%   \begin{description}
%    \item[$\alpha + \beta = 0 \Leftrightarrow \lambda_2 = \lambda_3 = 2$:] \todo{déterminer la dimension du sous-espace propre associé.} 
%    \item[$\alpha + \beta = 36 \Leftrightarrow \lambda_1 = \lambda_3 = -1$:] \todo{déterminer la dimension du sous-espace propre associé.} 
%   \end{description}}
\end{enumerate}
