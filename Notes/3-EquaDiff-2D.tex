%-------------------------------------------------------------------------------
%-------------------------------------------------------------------------------
\section{ODE en dimension 2} \label{sec:EquaDiff-nonLineaire2D}
%-------------------------------------------------------------------------------

On s'intéresse ici au cas d'une système d'ODE $\dot y = F(y)$ avec $F : \Rbb^2 \mapsto \Rbb^2$. 

%-------------------------------------------------------------------------------
\subsection{Classification des équilibres} 
%-------------------------------------------------------------------------------

La classification des équilibres repose sur l'étude des valeurs propres $\lambda_1$ et $\lambda_2$ de la matrice jacobienne $J_{x^*}$ en tout point d'équilibre $x^*$ solution de $F(x^*) = 0$. On peut ainsi définir la classification suivante :
$$
\begin{tabular}{ll}
  $(\lambda_1, \lambda_2) \in \Rbb, (\lambda_1, \lambda_2) < 0 :$ & point stable ou 'puit' ; \\
  $(\lambda_1, \lambda_2) \in \Rbb, (\lambda_1, \lambda_2) > 0 :$ & point instable ou 'source' ; \\
  $(\lambda_1, \lambda_2) \in \Rbb, \lambda_1 > 0 , \lambda_2 < 0 :$ & point selle ; \\
  $(\lambda_1, \lambda_2) \in \Cbb, \Re(\lambda_1, \lambda_2) < 0 :$ & point de concentration ou 'puit en spirale' ; \\
  $(\lambda_1, \lambda_2) \in \Cbb, \Re(\lambda_1, \lambda_2) > 0 :$ & point de repulsion ou 'source en spirale'. \\
  $(\lambda_1, \lambda_2) \in \Cbb \setminus \Rbb:$ & voir ci-dessous.
\end{tabular}
$$

\begin{figure*}[ht]
  \begin{center}
    \includegraphics{L3-SU-MathBio-StabiliteEquilibresD2}
  \end{center}
  \caption{Stabilité des équilibres en dimension 2. $\tau = \tr(A)$, $\delta=|A|$, $\Delta = \tau^2 - 4 \delta$ (discriminant de $P_A(\lambda)$.}
\end{figure*}

Les théorèmes vus jusqu'à présent ne permettent pas d'étudier le dernier cas, qui correspond à des trajectoires limites cycliques.

%-------------------------------------------------------------------------------
\subsection{Existence d'orbites périodiques} 
%-------------------------------------------------------------------------------

\begin{definition*}[Orbite]
  Pour toute solution $(f, I)$ d'une ODE, on appelle \emph{orbite} l'ensemble $\{f(t), t \in I\}$. S'il existe $t_0: [t_0, \infty) \subset I$, l'ensemble $\{f(t), t \geq t_0\}$ est appelé \emph{semi-orbite positive}.
\end{definition*}

%-------------------------------------------------------------------------------
\begin{theorem*}[Poincarré-Bendixson]
  Soit $D$ un fermé borné de $\Rbb^2$ ne contenant pas de point d'équilibre d'une ODE $\dot y = F(y)$. S'il existe une semi-orbite positive $O$ entièrement comprise dans $D$ alors l'ensemble $C$ des points limite de $O$ est une orbite périodique. Si $O$ n'est pas l'ensemble $C$, $C$ et appelé cycle limite.
\end{theorem*}

\remark
Le théorème dit que si une trajectoire solution de l'ODE est piégée dans une région $D$ bornée de $\Rbb^2$, elle doit se rapprocher de plus en plus d'une courbe $C$ fermée. Il n'est souvent pas difficile d'identifier une telle région $D$, mais déterminer le cycle limite $C$ est souvent plus difficile.

%-------------------------------------------------------------------------------
\paragraph{Détermination d'une cycle limite.} 

\begin{definition*}[Hamiltionien]
  Un hamiltionien (ou une fonction hamiltonienne) d'une ODE est une fonction $H : \Rbb^n \mapsto \Rbb$ qui reste constante le long des trajectoires solution de l'ODE.
\end{definition*}

\remark
Cette démfinition revient à dire que les solutions $y(t)$ suivent des ligne de niveau de la fonction hamiltonienne.

%-------------------------------------------------------------------------------
\paragraph{Cas $n = 2$.} 
On considère une ODE avec $F : \Rbb^2 \mapsto \Rbb^2$ : 
$$
\left\{\begin{array}{rcl} \dot x & = & F_1(x, y) \\ \dot y & = & F_2(x, y) \end{array}\right.
$$
Soit $(x(t), y(t))$ une solution maximale sur $I$ pour la condition initiales $(x(t_0), y(t_0)) = (x_0, y_0)$. Un hamitonien est une fonction $H : \Rbb^2 \mapsto \Rbb^2$ qui vérifie
$$
\forall t \in I: \quad H(x(t), y(t)) = H(x_0, y_0).
$$
Les orbites sont alors les lignes de niveau du hamiltonien : 
$$
\{x, y: H(x, y, ) = c\}.
$$

Toute la difficulté réside alors dans la détermination de la fonction $H$. On a cependant déjà vu que, le long d'une ligne de niveau, la dérivée par rapport au temps de $H(x(t), y(t))$ est nulle, soit
\begin{align*}
  \dot x(t) \frac{\partial H}{\partial x} (x(t), y(t)) +
  \dot y(t) \frac{\partial H}{\partial y} (x(t), y(t)) & = 0 \\
  \Leftrightarrow \qquad 
  F_1(x, y) \frac{\partial H}{\partial x} (x, y) +
  F_2(x, y) \frac{\partial H}{\partial y} (x, y) & = 0
\end{align*}

%-------------------------------------------------------------------------------
\paragraph{Cas ``séparable''.} 
Dans le cas particulier où on peut trouver $f$ et $g$ telles que
$$
\frac{F_1(x, y)}{F_2(x, y)} = \frac{g(y)}{f(x)}
\qquad \Leftrightarrow \qquad 
f(x) F_1(x, y) = g(y) F_2(x, y),
$$
la condition devient
$$
F_1(x, y) \frac{\partial H}{\partial x} (x, y) +
F_2(x, y) \frac{\partial H}{\partial y} (x, y) = 0
\qquad \Leftrightarrow \qquad 
\frac1{f(x)} \frac{\partial H}{\partial x} (x, y) +
\frac1{g(y)} \frac{\partial H}{\partial y} (x, y) = 0
$$
On peut alors chercher un hamiltonien de la forme
$$
H(x, y) = h_1(x) + h_2(y)
\qquad \Rightarrow \qquad 
\frac{\partial H}{\partial x} (x, y) = h'_1(x), \qquad
\frac{\partial H}{\partial y} (x, y) = h'_2(y).
$$
Les fonctions $h_1$ et $h_2$ doivent alors satisfaire
$$
h'_1(x) = c f(x), \qquad
h'_2(x) = - c g(x).
$$
On utilisera cette méthode pour déterminer les cycles limites du modèle de Lotka-Volterra.
