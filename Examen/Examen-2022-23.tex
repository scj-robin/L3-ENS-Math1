\documentclass[french, 12pt]{article}

%-------------------------------------------------------------------------------
\usepackage[a4paper,top=2cm,bottom=2cm,left=2cm,right=2cm,marginparwidth=1.75cm]{geometry}
\usepackage{amsmath,amsfonts,amssymb,amsthm}
\usepackage[french]{babel}
\usepackage[utf8]{inputenc}
\usepackage[T1]{fontenc}
\usepackage{enumerate}
\usepackage{natbib}
\usepackage{graphicx}
\usepackage{xspace}
\usepackage{color,xcolor}
\usepackage{tikz}
\usepackage{remreset}
\usepackage{url}
\usepackage{boites}
% \usepackage{extsizes} % Permet \documentclass[french, 14pt]{extreport}
% \usepackage[a4paper,top=1cm,bottom=2cm,left=1cm,right=1cm,marginparwidth=.75cm]{geometry}
% \usepackage{minitoc}

\graphicspath{{../Figures/}}
% Environnement
\newtheorem{theorem}{Théorème}
\newtheorem{definition}{Définition}
\newtheorem{lemma}{Lemme}
\newtheorem{proposition}{Proposition}
\newtheorem*{theorem*}{Théorème}
\newtheorem*{definition*}{Définition}
\newtheorem*{proposition*}{Proposition}
\newtheorem*{corollary*}{Corollaire}
\newtheorem*{assumption*}{Hypothèse}
\newtheorem*{algorithm*}{Algorithme}
\newtheorem*{lemma*}{Lemme}
\newtheorem*{remark*}{Remarque}
\newtheorem*{exercise*}{Exercice}
\newtheorem{exercise}{Exercice}
\newcommand{\remark}{\bigskip\noindent\textbf{\textsl{Remarque.}}\xspace}
\newcommand{\remarks}{\bigskip\noindent\textbf{\textsl{Remarques.}}\xspace}
\newcommand{\parSR}[1]{\paragraph*{\textsl{#1}}\xspace}
\renewcommand{\proof}{\bigskip\noindent\underline{\textsl{Démonstration}.}\xspace}
\newcommand{\eproof}{$\blacksquare$}

% Effets, couleurs
\newcommand{\emphase}[1]{\textcolor{red}{#1}}
\newcommand{\demoProp}[1]{\noindent{\textbf{\textsl{Démonstration de la proposition \ref{#1} :}}}}
\newcommand{\itemdot}{\textbullet}

% Moments
\DeclareMathOperator{\Esp}{\mathbb{E}}
\DeclareMathOperator{\diag}{diag}
\DeclareMathOperator{\Cov}{\mathbb{C}ov}
\DeclareMathOperator{\tr}{tr}
\DeclareMathOperator{\Var}{\mathbb{V}}
\let\Pr\relax\DeclareMathOperator{\Pr}{\mathbb{P}}
\renewcommand{\d}{\text{d}}

% R, N, ...
\newcommand{\cst}{\text{cst}}
\newcommand{\Cbb}{\mathbb{C}}
\newcommand{\Ibb}{\mathbb{I}}
\newcommand{\Nbb}{\mathbb{N}}
\newcommand{\Rbb}{\mathbb{R}}
\newcommand{\Zbb}{\mathbb{Z}}

% Indicateurs

% Lois et ensembles
\newcommand{\Acal}{\mathcal{A}}
\newcommand{\Bcal}{\mathcal{B}}
\newcommand{\Ccal}{\mathcal{C}}
\newcommand{\Ecal}{\mathcal{E}}
\newcommand{\Gcal}{\mathcal{G}}
\newcommand{\Ical}{\mathcal{I}}
\newcommand{\Lcal}{\mathcal{L}}
\newcommand{\Mcal}{\mathcal{M}}
\newcommand{\Ncal}{\mathcal{N}}
\newcommand{\Pcal}{\mathcal{P}}
\newcommand{\Rcal}{\mathcal{R}}
\newcommand{\Scal}{\mathcal{S}}
\newcommand{\Ucal}{\mathcal{U}}
\newcommand{\Xcal}{\mathcal{X}}
\newcommand{\Ycal}{\mathcal{Y}}

% Comments
\newcommand{\SR}[2]{\textcolor{gray}{#1}\textcolor{red}{#2}}
\newcommand{\todo}[1]{\textcolor{red}{\`A faire~: {\sl #1}}}
\newcommand{\dessin}[1]{
\begin{center}\framebox{\begin{minipage}{\textwidth}
  \textcolor{purple}{#1}
\end{minipage}}\end{center}
\bigskip
}
\newcommand{\progres}[1]{
\begin{center}\framebox{\begin{minipage}{\textwidth}
  \textcolor{blue}{{\sl #1}}
\end{minipage}}\end{center}
\bigskip
}
\newcommand{\solution}[1]{
\begin{center}\framebox{\begin{minipage}{\textwidth}
  \noindent{\sl Solution :}
  #1
\end{minipage}}\end{center}
\bigskip
}
% \newcommand{\exemple}[1]{
% \begin{center}\framebox{\begin{minipage}{\textwidth}
%   \parSR{Exemple.}
%   #1
% \end{minipage}}\end{center}
% \bigskip
% }
\newcommand{\exemple}[1]{
\begin{breakbox}
  \parSR{Exemple.}
  #1
\end{breakbox}
\bigskip
}

\newcommand{\SRcorrect}[2]{\textcolor{gray}{#1}\textcolor{blue}{#2}}
\newcommand{\SRcomment}[1]{\textcolor{blue}{[{\sl SR: #1}]}}



% Section numbering
\usepackage{chngcntr}
\renewcommand{\thepart}{\Roman{part}}
% \counterwithout{section}{part}
\setcounter{secnumdepth}{3}
\setcounter{tocdepth}{1}

% Proposition numbering
% \numberwithin{proposition}{section}
% \numberwithin{exercise}{section}
% \numberwithin{equation}{section}

% %-------------------------------------------------------------------------------
% %-------------------------------------------------------------------------------
% \title{\Huge{Ce qu'un biologiste doit savoir en mathématiques}}
% \author{SR d'après \cite{Lam20} : Annexes}
% \date{\today}


\title{}

%-------------------------------------------------------------------------------
%-------------------------------------------------------------------------------
\begin{document}
%-------------------------------------------------------------------------------
%-------------------------------------------------------------------------------

\begin{centering}
  \footnotesize{\sc École Normale Supérieure de Paris} 
  
  \bigskip
  \footnotesize{\sc Licence de Biologie L3}
  
  \bigskip
  \footnotesize{\sc Année 2022–23}
  
  \bigskip
  {\bf Mathématiques I : ce qu’un biologiste ne doit pas ignorer} 
  
  \bigskip
  {\sc Examen}
  
\end{centering}

\bigskip
L'épreuve dure 2 heures. 
Les notes de cours individuelles sont autorisées.
L’usage de tout appareil électronique est interdit à l’exception d’une calculatrice.

\tableofcontents

%-------------------------------------------------------------------------------
%-------------------------------------------------------------------------------
\section{Algèbre linéaire}
%-------------------------------------------------------------------------------

%-------------------------------------------------------------------------------
\subsubsection{\'Evolution de fréquences alléliques}
%-------------------------------------------------------------------------------

\paragraph{Dynamique de fréquences alléliques.}
Dans une population diploïde panmictique, on s’intéresse à un gène existant sous la forme de $m$ allèles. On désigne par $p_i(t)$ la proportion du pool gamétique portant l’allèle $i$  à la génération $t$ et $p(t) = (p_i(t))_{1 \leq i \leq m}$ le vecteur des fréquences alléliques à la génération $t$, qui vérifie $\sum_{1 \leq i \leq m}p_i(t) = 1$. 

On définit la matrice, supposée symétrique, $A = [a_{ij}]_{1 \leq i, j \leq p}$ où les $a_{ij}$ sont tous positifs ou nuls, mais non tous nuls. La dynamique des fréquences alléliques est donnée par
\begin{align} \label{eq:dynFreqModele}
  \forall \; i = 1 \dots m: & & 
  p_i(t+1) & = V(t)^{-1} p_i(t) \sum_{j=1}^m a_{ij} p_j(t) \\
  \text{avec} & & 
  V(t) & = \sum_{i=1}^m \sum_{j=1}^m a_{ij} p_i(t) p_j(j) = p(t)^\top \; A \; p(t), \nonumber
\end{align}
de sorte que $\sum_{1 \leq i \leq m} p_i(t+1) = 1$.

\bigskip
\begin{enumerate}
  \item Interpréter l'équation \eqref{eq:dynFreqModele}. Pourquoi $A$ est-elle supposée symétrique ? Que représente $V(t)$ ?
  \solution{$a_{ij}$ représente l'avantage relatif du génotype $(A_i, A_j)$, qui est symétrique par nature. \\
  $V(p) = p^\top A p$ est l'avantage moyen d'un descendant, du fait de la reproduction panmictique. 
  
  \remark 
  Ce modèle est paramétré à une constante près, c'est-à-dire qu'on aboutit à la même dynamique en remplaçant $A$ par $B = k A$ pour n'importe quel $k > 0$.
  }
\end{enumerate}

\paragraph{Caractérisation d'un équilibre.}
On suppose à partir de maintenant qu'il existe un équilibre $p^* = (p^*_i)_{1 \leq i \leq m}$ pour le système \eqref{eq:dynFreqModele}, c'est-à-dire vérifiant : 
$$
p(t) = p^* \qquad \Rightarrow \qquad p(t+1) = p^*,
$$
et qu'il est ``non trivial'', c'est-à-dire dans lequel tous les allèles sont présents : 
$$
\forall \; i = 1 \dots m: \qquad p^*_i > 0
$$
et on note $V^* = V(p^*) > 0$.

\begin{enumerate}
  \setcounter{enumi}{1}
  \item Montrer que, pour tout $i$ : $\sum_{1 \leq j \leq m} a_{ij} p^*_j = V^*$.
  \solution{L'équilibre $p^*$ est un point stationnaire de la dynamique \eqref{eq:dynFreqModele}, donc, en posant $p(t+1) = p(t) = p^*$, il vient
  $$
  \forall \; i = 1 \dots m: \quad V^* p^*_i = p^*_i \sum_{j=1}^m a_{ij} p^*_j,
  $$
  soit $V^* = \sum_{j=1}^m a_{ij} p^*_j, \forall i$, puisque les $p_i^*$ sont tous non nuls.. }
\end{enumerate}

\paragraph{Condition d'optimalité d'un équilibre.}
On cherche maintenant à déterminer sous quelles conditions l'équilibre (non trivial) $p^*$ correspond à un optimum pour $V$.

\begin{enumerate}
  \setcounter{enumi}{2}
  \item En écrivant tout vecteur de fréquences alléliques $p$ sous la forme $p = p^* + x$, montrer que $V^*$ est maximal ssi
  \begin{equation*} % \label{eq:dynFreqCondition}
    \forall x \in \Rbb^m \text{ tel que } \sum_{i=1}^m x_i = 0: \qquad x^\top A x \leq 0.
  \end{equation*}
  \solution{Si $p = p^* + x$, on a 
  $$
  V(p) 
  = p^\top A p
  = {p^*}^\top A p^* + 2 x^\top A p^* + x^\top A x
  = V^* + 2 x^\top A p^* + x^\top A x.
  $$
  De plus, $p$ et $p^*$ étant des vecteurs de fréquences ($1_m^\top p = 1_m^\top p^* = 1$), si $p = p^* +x$, alors $1_m^\top x = \sum_i x_i = 0$. Or on a vu que $\sum_j a_{ij} p^*_j$ est indépendant de $i$ (et égal à $V^*$), donc $A p^* = V^* 1_m$, donc
  $$
  x^\top A p^* = x^\top V^* 1_m = V^* x^\top 1_m = 0
  $$
  donc
  $$
  V(p) 
  = V^* + x^\top A x,
  $$
  $V^*$ est donc maximal, ssi $\forall p, V(p) \leq V^*$, c'est-à-dire ssi $x^\top A x \leq 0$.
  }
  %
  \item Montrer qu'on peut écrire $A$ sous la forme $A = R \Lambda R^\top$ où
  $$
  \Lambda = \text{diag}(\lambda_1, \dots, \lambda_k, 
    \lambda_{k+1}, \dots, \lambda_{k+\ell}, 
    0, \dots, 0)
  $$
  avec $k \geq 0$, $\ell \geq 0$, $k+\ell \leq m$, $\lambda_1, \dots \lambda_k > 0$ et $\lambda_{k+1}, \dots \lambda_{k+\ell} < 0$.
  \solution{
  $A$ étant symétrique, elle est diagonalisable et ses vecteurs propres sont orthogonaux. On peut donc l'écrire sous la forme $A = R \Lambda R^\top$ en mettant en premier les valeurs propres strictement positives, puis strictement négatives, puis, éventuellement, nulles. 
  }
  %
  \item Montrer l'existence de $p^*$ assure que $k \geq 1$.
  \solution{
  Si $k = 0$, alors tous les $\lambda_i$ sont négatifs (ou nuls), c'est-à-dire que $A$ est négative ($A \preceq 0$), donc
  $$
  \forall x \in \Rbb^m: \qquad 
  x^\top A x 
  = x^\top R \Lambda R^\top x 
  = y^\top \Lambda y  
  = \sum_i \lambda_i y_i^2 \leq 0 
  $$
  en posant $y = R^\top x$, or $V^* = {p^*}^\top A p^* > 0$ (car $A \geq 0$ et $p^*_i > 0$, $\forall i$), donc on a nécessairement $k \geq 1$.
  }
  \item En déduire que $V^*$ est maximal ssi $k=1$ et $\ell \geq 1$. 
  \solution{
  On a vu que, pour que $V^*$ soit maximal, il faut que $x^\top A x \leq 0$ pour tout $x$ vérifiant $\sum_i x_i = 0$, qui définit une sous-espace de dimension $m-1$. $k$ ne peut donc pas être supérieur ou égal à 2 (car alors il existerait alors au moins deux dimensions dans lesquelles $x^\top A x > 0$). \\
  La condition $\ell \geq 1$ assure seulement qu'il existe des vecteurs de fréquences $p = p^* + x$ donnant une viabilité $V(p) = V^* + x^\top A x < V^*$. 
  }
\end{enumerate}


%-------------------------------------------------------------------------------
%-------------------------------------------------------------------------------
\section{Fonction de plusieurs variables}
%-------------------------------------------------------------------------------

%-------------------------------------------------------------------------------
\subsubsection{Application linéaire tangente à une forme quadratique} 
%-------------------------------------------------------------------------------

On considère une matrice $A \in \Mcal_n$ symétrique, un vecteur $v \in \Rbb^n$ et la fonction 
$$
\begin{array}{rlll}
  f : & \Rbb^n & \mapsto & \Rbb \\
  & x & \to & f(x) = x^\top A x + v^\top x.
\end{array}
$$

\begin{enumerate}
  \item Montrer qu'il existe un vecteur $g(x) \in \Rbb^n$, qu'on précisera, tel que l'application linéaire tangente à $f$ en $x$ s'écrit
  $$
  \begin{array}{rlll}
    D_xf : & \Rbb^n & \mapsto & \Rbb^n \\
    & h & \to & D_xf(h) = g(x)^\top h.
  \end{array}
  $$
  \solution{On écrit
  \begin{align*}
    f(x+h) 
    & = (x+h)^\top A (x+h) + v^\top (x+h)
    = f(x) + x^\top A h + h^\top A x + h^\top A h + v^\top h \\
    & = f(x) + (2 x^\top A + v^\top) h + h^\top A h
  \end{align*}
  puisque $x^\top A h = h^\top A x$. On remarque alors que $h^\top A h = o(\|h\|)$ pour conclure que, puisque $A$ est symétrique, l'application linéaire tangent $D_x f$ s'écrit bien
  $$
  D_xf(h) = g(x)^\top h
  \qquad \text{avec} \quad
  g(x) = 2 A x + v.
  $$}
  \item En supposant que $A$ est inversible, déterminer le point stationnaire $x^*$ où $g(x)$ s'annule.
  \solution{En supposant $A$ inversible, on a
  $$
  g(x^*) = 0
  \qquad \Leftrightarrow \qquad
  2 A x^* + v = 0
  \qquad \Leftrightarrow \qquad
  x^* = - \frac12 A^{-1} v.
  $$}
  \item Donner une condition sur$A$ pour que $A$ soit un minimum local. (On pourra calculer la matrice hessienne de $A$.)
  \solution{La matrice hessienne de l'application $f$ en tout point $x$ vaut $H_x = 2 A$ (il suffit de déterminer l'application linéaire tangente à $g(x)$).
  $x^*$ est donc un minimum ssi $A$ est strictement définie négative}
  \item Discuter l'utilité de l'hypothèse selon laquelle $A$ est symétrique.
  \solution{On peut décomposer $A$ en ses parties symétrique $S$ et anti-symétrique $T$ : 
  $$
  S = \frac12(A + A^\top), \qquad 
  T = \frac12(A - A^\top), \qquad 
  \Rightarrow \quad
  A = S + T
  $$
  et remarquer que
  $$
  f(x) 
  = x^\top A x + v^\top x
  = x^\top S x + \frac12 \underset{=0}{\underbrace{(x^\top A x - x^\top A^\top x)}} + v^\top x,  
  $$
  c'est-à-dire que seulle la partie symétrique de $A$ contribue à la fonction Ode $f$.}
\end{enumerate}




%-------------------------------------------------------------------------------
%-------------------------------------------------------------------------------
\section{Systèmes dynamiques}
%-------------------------------------------------------------------------------

%-------------------------------------------------------------------------------
\subsubsection{L3 Bio SU : TD2, exercice 1 \todo{}} 
%-------------------------------------------------------------------------------

\todo{Voir L3 Bio SU : TD2, exercice 1}


%-------------------------------------------------------------------------------
\subsubsection{L3 Bio SU : TD1, exercice 2 \todo{}} 
%-------------------------------------------------------------------------------

\todo{Voir L3 Bio SU : TD1, exercice 2}


%-------------------------------------------------------------------------------
\subsubsection{Système dynamique en $y^5$ \todo{}}
%-------------------------------------------------------------------------------

On souhaite déterminer les points d'équilibre (et leur nature) du système
$$
\dot y = F(y) = \mu y + 2 y^3 - y^5.
$$
On a 
$$
F(y) = y(\mu + y^2 - y^4)
$$
qui s'annule pour $y = 0$ et pour les solutions de $(\mu + 2 y^2 - y^4)$. En posant, $z = y^2$, $\mu + 2 z - z^2 = 0$ admet des solutions si $\Delta =  4(1 + \mu) \geq 0$, soit $\mu \geq -1$. Ces solutions sont alors $z^* = -1 \pm \sqrt{1+\mu}$. La seule solution possiblement positive est $z^* = -1 + \sqrt{1+\mu}$ et elle l'est ssi $\mu > 1$. \\
Le système admet donc un unique point fixe $x^*=0$ si $\mu < 1$ et un second point fixe $x^* = -1+\sqrt{1+\mu}$ si $\mu > 1$. \\
On a de plus
$$
F'(x) = \mu + 3 y^2 - 5y^4
$$
dont le signe est celui de $\mu$ pour $x=0$ et \todo{nature de $x^* = -1+\sqrt{1+\mu}$ si $\mu > 1$.}


%-------------------------------------------------------------------------------
\todo{Voir L3 Bio SU : TD1, exercice 2}

%-------------------------------------------------------------------------------
%-------------------------------------------------------------------------------
\section{Probabilités}
%-------------------------------------------------------------------------------

%-------------------------------------------------------------------------------
\subsubsection{Processus de Galton–Watson quadratique \todo{}} %-------------------------------------------------------------------------------
  On considère un processus de Bienaymé-Galton-Watson partant d'une population de taille 1 et dans laquelle le nombre de descendants par individu suit la loi suivante :
  $$=
  \Pr\{X = k\} = \left\{
    \begin{array}{ll}
      1 - a - b & \text{si $k = 0$} \\
      a & \text{si $k = 1$} \\
      b & \text{si $k = 2$}
    \end{array}\right..
  $$
  \begin{enumerate}
    \item Montrer que la fonction génératrice de $X$ est
    $$
    f(s) = 1 - (+b) + as + bs^2.
    $$
    \solution{\todo{}}
    \item Montrer que les solutions de l'équation $s = f(s)$ sont de la forme
    $$
    s = \frac{(1-a) \pm |m-1|}{2(1 -(a+b))}
    $$
    où $m$ est l'espérance du nombre de descendants d'un individu.
    \solution{\todo{}}
  \end{enumerate}



%-------------------------------------------------------------------------------
%-------------------------------------------------------------------------------
\end{document}
%-------------------------------------------------------------------------------
%-------------------------------------------------------------------------------


