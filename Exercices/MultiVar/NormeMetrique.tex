%-------------------------------------------------------------------------------
\subsubsection{Norme fondée sur une matrice définie positive}
%-------------------------------------------------------------------------------

  Soit $A$ une matrice de $\Mcal_n$ symétrique définie positive (strictement): $A \succ 0$. Montrer que $\|\cdot\|_A$ définie par $\|x\|_A = (x^\top A x)^{1/2}$ est une norme pour $\Rbb^n$.
\solution{On vérifie les 3 conditions définissant une norme.
  \begin{enumerate}
   \item Par définition des matrices définies positives, on a bien pour tout $x \in \Rbb^n$, $x^\top A x \geq 0$  et $\{x^\top A x = 0\} \Leftrightarrow \{x = 0\}$.
   \item On a également $\|\lambda x\|_A = ((\lambda x)^\top A (\lambda x))^{1/2} = (\lambda^2 x^\top A x)^{1/2} = |\lambda| \|x\|_A$ ;
   \item Puisque $A$ est symétrique, on a
   $$
   \|x + y\|^2_A 
   = \|x\|^2_A + \| y\|^2_A + 2 x^\top A y
   $$
   et on peut la décomposer sous la forme $A = P \Lambda P^\top$ où $P$ et $P^\top$ sont orthonormales. Posons $u = \Lambda^{1/2} P^\top x$ et $v = \Lambda^{1/2} P^\top y$ qui vérifient
   \begin{align*}
   \|u\|_2 & = \sqrt{x^\top P \Lambda^{1/2} \Lambda^{1/2} P^\top x} = \|x\|_A, & 
   \|v\|_2 & = \|y\|_A, \\
   x^\top A y & = u^\top v.
   \end{align*}
   On peut alors utiliser l'inégalité tringulaire pour la norme $\|\cdot\|_2$ : 
   \begin{align*}
    \|x + y\|_A 
    = \|u + v\|_2 
    \leq \|u\|_2 + \|v\|_2
    = \|x\|_A + \|y\|_A.    
   \end{align*}
   \SR{Pour mémoire, l'inégalité triangulaire pour la norme $\|\cdot\|_2$ se démontre par Cauchy-Schwarz :
   \begin{align*}
    \|u + v\|^2_2 
    & = \|u\|^2_2 + \|v\|^2_2 + 2 u^\top v \leq \|u\|^2_2 + \|v\|^2_2 + 2 |u^\top v| \\
    & \leq \|u\|^2_2 + \|v\|^2_2 + 2 \|u\|_2 \|v\|_2 = (\|u\|_2 + \|v\|_2)^2.    
   \end{align*}}{}
  \end{enumerate}
}

