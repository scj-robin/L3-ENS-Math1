%-------------------------------------------------------------------------------
\subsubsection{Processus de Galton–Watson quadratique \todo{}} %-------------------------------------------------------------------------------
  On considère un processus de Bienaymé-Galton-Watson partant d'une population de taille 1 et dans laquelle le nombre de descendants par individu suit la loi suivante :
  $$=
  \Pr\{X = k\} = \left\{
    \begin{array}{ll}
      1 - a - b & \text{si $k = 0$} \\
      a & \text{si $k = 1$} \\
      b & \text{si $k = 2$}
    \end{array}\right..
  $$
  \begin{enumerate}
    \item Montrer que la fonction génératrice de $X$ est
    $$
    f(s) = 1 - a - b + as + bs^2.
    $$
    \solution{\todo{}}
    \item En déduire le nombre de descendants moyen par individu $m$
    $$
    m = a + 2b.
    $$
    \solution{On a $f'_X(s) = a + 2bs$, soit $m = f'_X(1) = a + 2b$.}
    \item Montrer que les solutions de l'équation $q = f(q)$ sont de la forme
    $$
    q = \frac{(1-a) \pm (m-1)}{2(1 - a - b)}.
    $$
    \solution{On veur résoudre $q = f(q)$, c'est-à-dire
    $(1 - a -b) + (a-1)q + bq^2 = 0$
    dont le discriminant est
    $$
    \Delta = (a-1)^2 - 4 b (1 - a -b) = ... = (a + 2b - 1)^2 = (m-1)^2
    $$
    d'où la solution.}
    \todo{En fait les solutions sont $q = 1$ et $q = b/(1 - a - b)$}.
  \end{enumerate}
