%-------------------------------------------------------------------------------
\subsubsection{Application linéaire tangente à une forme quadratique} 
%-------------------------------------------------------------------------------

On considère une matrice $A \in \Mcal_n$ symétrique, un vecteur $v \in \Rbb^n$ et la fonction 
$$
\begin{array}{rlll}
  f : & \Rbb^n & \mapsto & \Rbb \\
  & x & \to & f(x) = x^\top A x + v^\top x.
\end{array}
$$

\begin{enumerate}
  \item Montrer qu'il existe un vecteur $g(x) \in \Rbb^n$, qu'on précisera, tel que l'application linéaire tangente à $f$ en $x$ s'écrit
  $$
  \begin{array}{rlll}
    D_xf : & \Rbb^n & \mapsto & \Rbb^n \\
    & h & \to & D_xf(h) = g(x)^\top h.
  \end{array}
  $$
  \solution{On écrit
  \begin{align*}
    f(x+h) 
    & = (x+h)^\top A (x+h) + v^\top (x+h)
    = f(x) + x^\top A h + h^\top A x + h^\top A h + v^\top h \\
    & = f(x) + (2 x^\top A + v^\top) h + h^\top A h
  \end{align*}
  puisque $x^\top A h = h^\top A x$. On remarque alors que $h^\top A h = o(\|h\|)$ pour conclure que, puisque $A$ est symétrique, l'application linéaire tangent $D_x f$ s'écrit bien
  $$
  D_xf(h) = g(x)^\top h
  \qquad \text{avec} \quad
  g(x) = 2 A x + v.
  $$}
  \item En supposant que $A$ est inversible, déterminer le point stationnaire $x^*$ où $g(x)$ s'annule.
  \solution{En supposant $A$ inversible, on a
  $$
  g(x^*) = 0
  \qquad \Leftrightarrow \qquad
  2 A x^* + v = 0
  \qquad \Leftrightarrow \qquad
  x^* = - \frac12 A^{-1} v.
  $$}
  \item Donner une condition sur$A$ pour que $x^*$ soit un minimum local. (On pourra calculer la matrice hessienne de $A$.)
  \solution{La matrice hessienne de l'application $f$ en tout point $x$ vaut $H_x = 2 A$ (il suffit de déterminer l'application linéaire tangente à $g(x)$).
  $x^*$ est donc un minimum ssi $A$ est strictement définie négative}
  \item Discuter l'utilité de l'hypothèse selon laquelle $A$ est symétrique.
  \solution{On peut décomposer $A$ en ses parties symétrique $S$ et anti-symétrique $T$ : 
  $$
  S = \frac12(A + A^\top), \qquad 
  T = \frac12(A - A^\top), \qquad 
  \Rightarrow \quad
  A = S + T
  $$
  et remarquer que
  $$
  f(x) 
  = x^\top A x + v^\top x
  = x^\top S x + \frac12 \underset{=0}{\underbrace{(x^\top A x - x^\top A^\top x)}} + v^\top x,  
  $$
  c'est-à-dire que seule la partie symétrique de $A$ contribue à la fonction $f$.}
\end{enumerate}


