%-------------------------------------------------------------------------------
\subsubsection{Processus de Galton–Watson géométrique} %-------------------------------------------------------------------------------
  On considère un processus de Bienaymé-Galton-Watson partant d'une population de taille 1 et dans laquelle le nombre de descendants par individu suit une loi géométrique $\Gcal(a)$ (pour mémoire, $X \sim \Gcal(a)$ ssi $\Pr\{X = k\} = (1-a) a^k$).
  \begin{enumerate}
    \item Calculer le nombre moyen $m$ de descendants par individus en fonction de $a$.
    \solution{On a
    $$
    m = \Esp(X) 
    = \sum_{k \geq 0} (1-a) \; a^k \; k 
    = a(1-a) \sum_{k \geq 0} k \; a^{k-1}
    $$
    où on reconnaît la dérivée de la fonction $f(a) = \sum_{k \geq 0} a^{k} = (1-a)^{-1}$, soit $f'(a) = (1-a)^{-2}$. On a donc
    $$
    m
    = a(1-a) (1-a)^{-2}
    = a / (1-a)
    $$
    (qui est une fonction monotone croissante de $a$).}
    \item Pour quelles valeurs de $a$ le processus est-il critique, sous-critique ou sur-critique ?
    \solution{On a $m \leq 1 \; \Leftrightarrow \; a \leq 1-a \; \Leftrightarrow \; a \leq 1/2$.  Le processus est donc critique pour $a = 1/2$, sous critique pour $a < 1/2$ et sur-critique pour $a > 1/2$.}
    \item Montrer que, pour $m > 1$, la probabilité d'extinction $q$ de cette population vaut $1/m$.
    \solution{On sait que la probabilité d'extinction $q$ partant d'une population de taille 1 est le plus petit point fixe de l'équation $s = f_X(s)$ où $f_X$ est la fonction génératrice des probabilités de la loi géométrique : 
    $$
    f_X(s) = (1 - a) / (1 - as).
    $$
    On cherche donc à résoudre $1 - a = (1 - as) s \; \Leftrightarrow \; as^2 - s + (1-a) = 0$ dont on sait que $s = 1$ est solution. On peut donc factoriser
    $$
    as^2 - s + (1-a) = a (s-1) \left(s  - \frac{1-a}a\right)
    $$
    où $(1-a)/a < 1$ dès que $a > 1/2$. La probabilité d'extinction vaut donc
    $$
    q = \frac{1-a}a = \frac1m.
    $$}
  \end{enumerate}
