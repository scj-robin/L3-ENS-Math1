%-------------------------------------------------------------------------------
\subsection{Normes}
%-------------------------------------------------------------------------------

%-------------------------------------------------------------------------------
\begin{exercise}[Equivalence des normes]
  Donner des constantes $c_1$ et $c_2$ permettant de comparer les trois normes $\|\cdot\|_1$, $\|\cdot\|_2$ et $\|\cdot\|_\infty$ dans $\Rbb^n$.
\end{exercise}
\solution{
}

%-------------------------------------------------------------------------------
\begin{exercise}[Norme]
  Soit $A$ une matrice de $\Mcal_n$ symétrique définie positive (strictement). Montrer que $\|\cdots\|_A$ définie par $\|x\|_A = (x^\top A x)^{1/2}$ est une norme.
\end{exercise}
\solution{On vérifie les 3 conditions définissant une norme.
  \begin{itemize}
   \item Par définition des matrices définies positives, on a bien pour tout $x \in \Rbb^n$, $x^\top A x \geq 0$  et $\{x^\top A x = 0\} \Leftrightarrow \{x = 0\}$.
   \item On a également $\|\lambda x\|_A = ((\lambda x)^\top A (\lambda x))^{1/2} = (\lambda^2 x^\top A x)^{1/2} = |\lambda| \|x\|_A$ ;
   \item Puisque $A$ est symétrique, on a
   $$
   \|x + y\|^2_A 
   = \|x\|^2_A + \| y\|^2_A + 2 x^\top A y
   $$
   et on peut la décomposer sous la forme $A = P \Lambda P^\top$ où $P$ et $P^\top$ sont orthonormales. Posons $u = \Lambda^{1/2} P^\top x$ et $v = \Lambda^{1/2} P^\top y$ qui vérifient
   \begin{align*}
   \|u\|_2 & = \sqrt{x^\top P \Lambda^{1/2} \Lambda^{1/2} P^\top x} = \|x\|_A, & 
   \|v\|_2 & = \|y\|_A,, \\
   x^\top A y & = u^\top v.
   \end{align*}
   Par Cauchy-Schwarz, on a alors
   \begin{align*}
    \|x + y\|^2_A 
    & = \|u\|^2_2 + \|v\|^2_2 + 2 u^\top v \leq \|u\|^2_2 + \|v\|^2_2 + 2 |u^\top v| \\
    & \leq \|u\|^2_2 + \|v\|^2_2 + 2 \|u\|_2 \|v\|_2 = (\|u\|_2 + \|v\|_2)^2 \\
    & = (\|x\|_A + \|y\|_A)^2.    
   \end{align*}
  \end{itemize}
}

%-------------------------------------------------------------------------------
\subsection{Application linéaire tangente}
%-------------------------------------------------------------------------------

%-------------------------------------------------------------------------------
\paragraph{Application linéaire tangente}
Soit la fonction
$$
\begin{array}{rrcl}
  f :  & \Mcal_n(\Rbb) & \mapsto & \Mcal_n(\Rbb) \\
  & X & \rightarrow & X^3.
\end{array}
$$
Démontrer qu'elle est partout différentiable et déterminer son application linéaire tangente $D_X f$ en toute matrice $X$ de $\Mcal_n(\Rbb)$.
\solution{
  On a
  $$
  f(X+H) = X^3 + X^2H + XHX + HX^2 + H^2X + HXH + XH^2 + H^3.
  $$
  Comme pour la fonction $f(X) = X^2$, on montre que 
  $$
  H^2X = o(\|H\|), \qquad HXH  = o(\|H\|), \qquad XH^2  = o(\|H\|), \qquad H^3 = o(\|H\|).
  $$
  On a donc 
  $$
  f(X+H) - f(X) = X^2H + XHX + HX^2 + o(\|H\|)
  $$
  où $X^2H + XHX + HX^2$ est linéaire en $H$. L'application linéaire tangente est donc
  $$
  \begin{array}{rrcl}
    L_X : & \Mcal_n(\Rbb) & \mapsto & \Mcal_n(\Rbb) \\
    & H & \rightarrow & X^2H + XHX + HX^2.
  \end{array}
  $$
}

%-------------------------------------------------------------------------------
\subsection{Matrice jacobienne}
%-------------------------------------------------------------------------------

%-------------------------------------------------------------------------------
\paragraph{Jacobienne.}
On considère une fonction $F$ de $[0, 1] \times [0, 1]$ dans $[0, 1] \times [0, 1]$ de classe $C^1$ qu'on écrit
$$
F(x, y) = (f_0(x, y), f_1(x, y)).
$$
\begin{enumerate}
  \item \'Ecrire la matrice jacobienne $J(x, y)$ de la fonction $F$ en tout $(x, y) \in [0, 1] \times [0, 1]$.
  \solution{\todo{}}
  \item \'Ecrire l'application linéaire tangente de $F \circ F$ en tout point $(x, y)$.
  \solution{\todo{}}
  \item On suppose que $F(1, 1) = (1, 1)$. Donner la matrice jacobienne $J_n(1, 1)$ de la fonction $F$ itérée $n$ fois avec elle-même, en $(x, y) = (1, 1)$.
  \solution{\todo{}}
\end{enumerate}

%-------------------------------------------------------------------------------
\subsection{Extrema}
%-------------------------------------------------------------------------------

Voir \url{www.bibmath.net/ressources/index.php?action=affiche&quoi=bde/analyse/calculdiff/extrema&type=fexo}, exercice 7
