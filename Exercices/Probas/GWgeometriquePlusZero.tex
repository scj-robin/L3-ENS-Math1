%-------------------------------------------------------------------------------
\subsubsection{Processus Galton–Watson géométrique avec excès de zéro}
%-------------------------------------------------------------------------------

Les organismes d’une espèce asexuée se reproduisent suivant un processus de Galton–Watson, où chaque individu laisse à la génération suivante un nombre aléatoire $X$ d’individus. \\
On note $p_k = \Pr\{X = k\}$ et on suppose que 
$$
p_0 = b
\qquad \text{et} \qquad 
p_k = (1 - b) (1 - a) a^{k-1} \text{ pour } k \geq 1.
$$
\begin{enumerate}
  \item Montrer que la fonction génératrice de $X$ vaut
  $$
  f(s) := \Esp(s^X) = b + (1-b)(1-a) \frac{s}{1 - as}.
  $$
  \solution{Par définition, on a
  \begin{align*}
    f(s) 
    & =\Esp(s^X) = \sum_{k \geq 0} p_k s^k 
    = b + (1-b) \sum_{\textcolor{red}{k \geq 1}} (1 - a) a^{k-1} s^k \\
    & = b + (1 - b) s \sum_{k \geq 1} (1 - a) (as)^{k-1}
    = b + (1 - b) s \sum_{k \geq 0} (1 - a) (as)^k
  \end{align*}
  où on reconnaît la fonction génératrice d'une variable aléatoire géométrique : 
  $$
  \sum_{k \geq 0} (1 - a) (as)^k = \frac{1 - a}{1 - as},
  $$
  qui donne le résultat.
  }
  \item Déterminer la valeur de la probabilité, notée $q$, qu'une population de cette espèce issue d’un seul individu fondateur s'éteigne.
  \solution{On sait que $q$ est le plus petit point fixe de la fonction $f$ dans l'intervalle $(0, 1]$. Il faut donc résoudre $f(s) = s$
  \begin{align*}
    \Leftrightarrow \qquad 
    b(1 - as) + (1-b)(1-a) s & = a - as^2 \\
    \Leftrightarrow \qquad
    as^2 - (a+b) s + b & = 0
  \end{align*}
  dont le discriminant vaut $\Delta = (a+b)^2 - 4ab = (a-b)^2$ et les solutions sont donc
  $$
  s = \frac{(a+b) \pm (a-b)}{2}
  \qquad \Leftrightarrow \qquad
  s \in \left\{\frac{b}a, 1\right\}.
  $$
  La probabilité d'extinction est donc $q = b/a$ si $b < a$ et 1 sinon.
  }
\end{enumerate}
