\documentclass[11pt]{article}  
%mettre \`a 12 pt pour la version d\'efinitive

\include{maquereaux}











\begin{document}

\begin{center}
\textsc{\'Ecole Normale Sup\'erieure de Paris} 

\vspace{0.2cm}

\textsc{Licence de Biologie L3} 

\vspace{0.2cm}

\textsc{Ann\'ee 2020--21}

\vspace{0.8cm}

\large\textbf{Math\'ematiques I : ce qu'un biologiste ne doit pas ignorer}

\vspace{0.8cm}

\normalsize

\textsc{Programme indicatif}



\end{center} 


\vspace{1cm}

\begin{tabular}{lcl}
\textbf{Responsable} &{\bf :}& Amaury Lambert\\
 & &Laboratoire de Probabilit\'es, Statistique \& Mod\'elisation -- Sorbonne Universit\'e  \\
 & \&&  Centre Interdisciplinaire de Recherche en Biologie -- Coll\`ege de France\\
 \textbf{Contact }&{\bf :}& 01 44 27 85 69 ou \texttt{amaury.lambert@upmc.fr}\\
 & \&& 01 44 27 13 91 ou \texttt{amaury.lambert@college-de-france.fr}\\
\textbf{Enseignants }&{\bf :}& Amaury Lambert (maths),\\
	&\&&Pierre Vincens, Beno\^it Perez-Lamarque et Corentin Clerc (info)\\
\textbf{Horaires }&{\bf :}& vendredi de 9h \`a midi et mercredi de 14h \`a 17h.
 \end{tabular}
 


\vspace{1.7cm}
 




\begin{enumerate}





\item \textbf{vendredi 16 octobre (AL).} Alg\`ebre lin\'eaire (I). Quelques exemples d'usage de matrices en biologie. Calcul de d\'eterminants par les cofacteurs.

\bigskip


\item \textbf{vendredi 23 octobre (AL).}  Alg\`ebre lin\'eaire (II). Polyn\^ome caract\'eristique et extraction de valeurs propres. Rappels des th\'eor\`emes g\'en\'eraux sur la diagonalisation, cas des matrices sym\'etriques r\'eelles et des matrices positives au sens de l'alg\`ebre bilin\'eaire.

\bigskip


\item  \textbf{vendredi 30 octobre.} Rel\^ache.

\bigskip

 
\item \textbf{vendredi 6 novembre (AL).} Alg\`ebre lin\'eaire (III). Th\'eorie de Perron--Frobenius sur les matrices \`a coefficients positifs. R\'egularit\'e, valeur propre dominante, th\'eor\`eme de Perron--Frobenius. Application \`a la dynamique des populations structur\'ees.\\
Fonctions de plusieurs variables (I). Diff\'erentiabilit\'e, d\'efinition d'application lin\'eaire tangente.

\bigskip



\item \textbf{vendredi 13 novembre (AL).}  Fonctions de plusieurs variables (II). Matrice jacobienne, jacobien et changement de variables en dimension sup\'erieure ou \'egale \`a 2. 
\bigskip


\item \textbf{mercredi 18 et vendredi 20 novembre (PV$+$BPL$+$CC).} EN DEMI-GROUPE. Programmation pour la mod\'elisation (I). 

\bigskip


\item \textbf{mercredi 25 et vendredi 27 novembre (PV$+$BPL$+$CC).} EN DEMI-GROUPE. Programmation pour la mod\'elisation (II). 

\bigskip


\item \textbf{mercredi 2 et vendredi 4 d\'ecembre (PV$+$BPL$+$CC).} EN DEMI-GROUPE. Programmation pour la mod\'elisation (III). 

\bigskip


\item \textbf{mercredi 9 d\'ecembre (AL).} ATTENTION CR\'ENEAU INHABITUEL. Syst\`emes dynamiques (I). \'equations diff\'erentielles et syst\`emes dynamiques en temps continu. Existence, unicit\'e des solutions. Recherche des \'equilibres et lin\'earisation. Classification des \'equilibres en dimensions 1 et 2.


\bigskip



\item \textbf{vendredi 11 d\'ecembre (AL).} Syst\`emes dynamiques (II). Exercices sur la dynamique proie-pr\'edateur, \'equations de Lotka--Volterra. Cycles et cycles-limites. Th\'eor\`eme de Poincar\'e--Bendixson. Bifurcations (selle-n\oe ud, transcritique, en fourche, de Hopf). Le mod\`ele SIR (sain et susceptible -- infect\'e -- sain et immunis\'e) de Kermack--McKendrick.


\bigskip


\item \textbf{mercredi 16 d\'ecembre (AL).} ATTENTION CR\'ENEAU INHABITUEL. Probabilit\'es (I). Cha\^ines de Markov : classes de communication, irr\'eductibilit\'e, \'etats transients, r\'ecurrents, absorbants. P\'eriode d'une cha\^ine de Markov irr\'eductible.


\bigskip


\item \textbf{vendredi 18 d\'ecembre (AL).} Probabilit\'es (II). Loi du nombre de visites d'un \'etat transient, th\'eor\`eme ergodique pour les cha\^ines r\'ecurrentes \`a espace d'\'etats fini. Fonctions g\'en\'eratrices. Processus de Bienaym\'e--Galton--Watson, probabilit\'e d'extinction, th\'eor\`eme limite surcritique. 

\bigskip


\item \textbf{vendredi 8  janvier (PV$+$BPL$+$CC).} MATIN et APR\`ES-MIDI EN DEMI-GROUPE alternant avec BIG. Programmation pour la mod\'elisation (IV). 

\bigskip


 \item \textbf{vendredi 15 janvier (AL).} MATIN et APR\`ES-MIDI EN DEMI-GROUPE alternant avec BIG. Probabilit\'es (III). Marches al\'eatoires, loi des grands nombres, r\'ecurrence des marches d'esp\'erance nulle. Processus de Poisson et de Poisson ponctuel. %\'etude des marches al\'eatoires simples (sauts de $-1$, 0 ou $+1$). Fonctions g\'en\'eratrices des temps d'atteinte. 

\bigskip



 
\item \textbf{semaine du 18 au 22 janvier (AL).} Validation.

 
\end{enumerate} 
 
\end{document}