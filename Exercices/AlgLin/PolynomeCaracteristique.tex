%-------------------------------------------------------------------------------
\subsubsection{Polynômes caractéristiques}
%-------------------------------------------------------------------------------

  \begin{enumerate}
% [Ex 1.3.3]
  \item On considère la matrice 
  $$
  A = \left[\begin{array}{cc} a & b \\ c & d\end{array}\right]
  $$
  Déterminer son polynôme caractéristique et donner les conditions sur $p = \tr(A)$ et $q = \det(A)$ pour que $A$ admette deux valeurs propres réelles.
  \solution{On a 
  $$
  |A - \lambda I_2| 
  = \left|\begin{array}{cc} a - \lambda & b \\ c & d - \lambda \end{array}\right|
  = (a - \lambda)( d - \lambda) - bc
  = \lambda^2 - (a+d) \lambda + (ad - bc)
  = \lambda^2 - p \lambda + q
  $$
  qui admet deux solutions réelles (éventuellement égales) ssi
  $$
  p^2 - 4 q \geq 0
  \qquad \Leftrightarrow \qquad
  q \leq p^2 / 4.
  $$}
  \item Montrer le polynôme caractéristique de la matrice 
  \begin{align*}
    B & = \left[\begin{array}{rrr}
      -1 & 3 & 1 \\ 0 & 2 & 1 \\ \alpha & \beta & 2
      \end{array}\right].  
  \end{align*}
  est
  $$
  P_B(\lambda) = -(1+\lambda)(\lambda^2 - 4 \lambda + 4 - \alpha - \beta)
  $$
  \solution{On développe par rapport à la première colonne.}
  \item En déduire une condition sur $\alpha$ et $\beta$ pour que $B$ soit diagonalisable.
  \solution{La premièrre valeur propre est donc $\lambda_1 = -1$. Le discriminant du polynôme 
  $$
  Q(\lambda) = \lambda^2 - 4 \lambda + 4 - (\alpha+\beta)
  $$
  vaut $\Delta = 16 - 4(4 - (\alpha+\beta)) = \alpha + \beta$. Les deux valeurs propres restantes $\lambda_2$ et $\lambda_3$ sont donc réelles et distinctes ssi $\alpha + \beta > 0$ et valent alors
  $$
  \lambda_2 = \frac{4 + \sqrt{\alpha + \beta}}{2}, 
  \qquad \lambda_3 = \frac{4 - \sqrt{\alpha + \beta}}{2}.
  $$
  Elles sont de plus distinctes de $\lambda_1 = -1$ ssi 
  $$
  \lambda_3 \neq -1 \qquad \leftrightarrow \qquad
  \alpha + \beta \neq 9.
  $$
  Dans ce cas ($\alpha + \beta > 0$ et $\alpha + \beta \neq 9$), les 3 valeurs propres sont réelles et distinctes et $B$ est donc diagonalisable. \\
  Reste à étudier les cas 
  \begin{description}
   \item[$\alpha + \beta = 0 \Leftrightarrow \lambda_2 = \lambda_3$:] \todo{déterminer la dimension du sous-espace propre associé.} 
   \item[$\alpha + \beta = 9 \Leftrightarrow \lambda_1 = \lambda_3$:] \todo{déterminer la dimension du sous-espace propre associé.} 
  \end{description}}
\end{enumerate}
