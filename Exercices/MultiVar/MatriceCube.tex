%-------------------------------------------------------------------------------
\subsubsection{Cube d'une matrice}
%-------------------------------------------------------------------------------

Soit la fonction
$$
\begin{array}{rrcl}
  f :  & \Mcal_n(\Rbb) & \mapsto & \Mcal_n(\Rbb) \\
  & X & \rightarrow & X^3.
\end{array}
$$
Démontrer qu'elle est partout différentiable et déterminer son application linéaire tangente $D_X f$ en toute matrice $X$ de $\Mcal_n(\Rbb)$.
\solution{
  On a
  $$
  f(X+H) = X^3 + X^2H + XHX + HX^2 + H^2X + HXH + XH^2 + H^3.
  $$
  Comme pour la fonction $f(X) = X^2$, on montre que 
  $$
  H^2X = o(\|H\|), \qquad HXH  = o(\|H\|), \qquad XH^2  = o(\|H\|), \qquad H^3 = o(\|H\|).
  $$
  On a donc 
  $$
  f(X+H) - f(X) = X^2H + XHX + HX^2 + o(\|H\|)
  $$
  où $X^2H + XHX + HX^2$ est linéaire en $H$. L'application linéaire tangente est donc
  $$
  \begin{array}{rrcl}
    L_X : & \Mcal_n(\Rbb) & \mapsto & \Mcal_n(\Rbb) \\
    & H & \rightarrow & X^2H + XHX + HX^2.
  \end{array}
  $$
}
