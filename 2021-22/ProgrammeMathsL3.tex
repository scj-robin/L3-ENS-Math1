\documentclass[a4paper, 10pt]{article}

%% Language and font encodings
\usepackage[english]{babel}
\usepackage[utf8x]{inputenc}
\usepackage[T1]{fontenc}
\usepackage{url}

%% Sets page size and margins
\usepackage[a4paper,top=2cm,bottom=2cm,left=2cm,right=2cm,marginparwidth=1.75cm]{geometry}

%-------------------------------------------------------------------------------
%-------------------------------------------------------------------------------
\begin{document}
%-------------------------------------------------------------------------------
%-------------------------------------------------------------------------------

%-------------------------------------------------------------------------------
%-------------------------------------------------------------------------------
\begin{center}
\textsc{\'Ecole Normale Supérieure de Paris} 

\bigskip
\textsc{Licence de Biologie L3} 

\bigskip
\textsc{Année 2021--22}

\bigskip
\large\textbf{Mathématiques I : ce qu'un biologiste ne doit pas ignorer}

\bigskip
\textsc{Programme indicatif}
\end{center} 


%-------------------------------------------------------------------------------
%-------------------------------------------------------------------------------
\bigskip
\begin{description}
\item[Responsable :] Amaury Lambert \\
  Institut de Biologie de l'ENS \& Centre Interdisciplinaire de Recherche en Biologie -- Collège de France
%   \item[Contact :] \url{amaury.lambert@ens.fr} ou \url{amaury.lambert@college-de-france.fr} 
  \item[Enseignants :] Stéphane Robin (maths) \& Pierre Vincens et Corentin Clerc (info)
 \item[Horaires :] vendredi de 9h à midi et mardi de 9h à 12h.
 \end{description}
 

%-------------------------------------------------------------------------------
%-------------------------------------------------------------------------------
\bigskip
\hline
\paragraph{{\sc Emploi du temps:}}
\begin{description}
\item[vendredi 8 octobre (SR) :] Algèbre linéaire (I). Quelques exemples d'usage de matrices en biologie. Calcul de déterminants par les cofacteurs.

\bigskip
\item[vendredi 15 octobre (SR) :]  Algèbre linéaire (II). Polyn\^ome caractéristique et extraction de valeurs propres. Rappels des théorèmes généraux sur la diagonalisation, cas des matrices symétriques réelles et des matrices positives au sens de l'algèbre bilinéaire.

\item[vendredi 22 octobre (SR) :] Algèbre linéaire (III). Théorie de Perron--Frobenius sur les matrices à coefficients positifs. Régularité, valeur propre dominante, théorème de Perron--Frobenius. Application à la dynamique des populations structurées.\\
Fonctions de plusieurs variables (I). Différentiabilité, définition d'application linéaire tangente.

\bigskip
\item[vendredi 29 octobre (SR) :]  Fonctions de plusieurs variables (II). Matrice jacobienne, jacobien et changement de variables en dimension supérieure ou égale à 2. 

\bigskip
\item[vendredi 5 novembre (SR) :] Systèmes dynamiques (I). équations différentielles et systèmes dynamiques en temps continu. Existence, unicité des solutions. Recherche des équilibres et linéarisation. Classification des équilibres en dimensions 1 et 2.

% \bigskip 
% \item[vendredi 12 novembre :] relâche

\bigskip
\item[vendredi 19 novembre (SR) :] Systèmes dynamiques (II). Exercices sur la dynamique proie-prédateur, équations de Lotka--Volterra. Cycles et cycles-limites. Théorème de Poincaré--Bendixson. Bifurcations (selle-n\oe ud, transcritique, en fourche, de Hopf). Le modèle SIR (sain et susceptible -- infecté -- sain et immunisé) de Kermack--McKendrick.

\bigskip
\item[mardi 23 novembre (SR) :] ({\sl Attention: créneau inhabituel}). Probabilités (I). Cha\^ines de Markov : classes de communication, irréductibilité, états transients, récurrents, absorbants. Période d'une cha\^ine de Markov irréductible.

\bigskip
\item[vendredi 26 novembre (SR) :] Probabilités (II). Loi du nombre de visites d'un état transient, théorème ergodique pour les cha\^ines récurrentes à espace d'états fini. Fonctions génératrices. Processus de Bienaymé--Galton--Watson, probabilité d'extinction, théorème limite surcritique. 

\bigskip
\item[mardi 30 novembre (SR) :] ({\sl Attention: créneau inhabituel}) Probabilités (III). Marches aléatoires, loi des grands nombres, récurrence des marches d'espérance nulle. %étude des marches aléatoires simples (sauts de $-1$, 0 ou $+1$). Fonctions génératrices des temps d'atteinte. 
 
\bigskip
\item[vendredi 3 décembre (SR) :] Probabilités (IV). Processus de Poisson et de Poisson ponctuel.

%-------------------------------------------------------------------------------
\bigskip
\item[vendredi 10 décembre (PV+CC) :] Programmation pour la modélisation (I). 

\bigskip
\item[vendredi 17 décembre (PV+CC) :] Programmation pour la modélisation (II). 

\bigskip
\item[vendredi 7 janvier (PV+CC) :] Programmation pour la modélisation (III). 

\bigskip
\item[vendredi 14 janvier (PV+CC) :] Programmation pour la modélisation (IV). 

%-------------------------------------------------------------------------------
\bigskip
\item[semaine du 20 au 24 janvier (SR) :] Validation.

\end{description} 
 
%-------------------------------------------------------------------------------
%-------------------------------------------------------------------------------
\end{document}
%-------------------------------------------------------------------------------
%-------------------------------------------------------------------------------
