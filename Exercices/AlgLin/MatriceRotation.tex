%-------------------------------------------------------------------------------
\subsubsection{Matrice $3\times3$}
%-------------------------------------------------------------------------------

Pour $\theta \in [0, 2\pi]$, on considère la matrice $A \in \Mcal_3$ :
$$
A = \left[\begin{array}{ccc}
            \cos\theta & -\sin\theta & 0 \\
            \sin\theta & \cos\theta & 0 \\
            0 & 0 & 1
          \end{array}\right].
$$

\begin{enumerate}
  \item Déterminer le polynôme caractéristique de $A$, noté $P_A(\lambda)$, et ses racines.
    \solution{
      En développant par rapport à la dernière ligne, on obtient
      \begin{align*}
        P_A(\lambda) 
          & = \left|\begin{array}{ccc}
                    \cos\theta - \lambda & -\sin\theta & 0 \\
                    \sin\theta & \cos\theta - \lambda & 0 \\
                    0 & 0 & 1 - \lambda
                  \end{array}\right|
          = (1 - \lambda) \left|\begin{array}{cc}
                    \cos\theta - \lambda & -\sin\theta  \\
                    \sin\theta & \cos\theta - \lambda 
                  \end{array}\right| \\
        & = (1 - \lambda) [(\cos\theta - \lambda)^2 + \sin^2\theta]
          = (1 - \lambda) [\lambda^2 - 2 \lambda \cos\theta + 1].
      \end{align*}
      Le discriminant du terme entre crochets vaut $\Delta = 4\cos^2\theta - 4 = 4(\cos^2\theta - 1)  \leq 0$. Les racines de $P_A(\lambda)$ sont donc
      $$
      \lambda = 1 
      \qquad \text{et} \qquad
      \lambda = \cos\theta \pm i \sqrt{1 - \cos^2\theta}. 
      $$
    }
  %
  \item A quelle condition $A$ est-elle diagonalisable?
    \solution{
      $A$ diagonalisable ssi toute ses valeurs propres sont réelles, c'est-à-dire ssi
      $$
      \Delta = 0 \qquad \Leftrightarrow \qquad
      \cos^2\theta = 1 \qquad \Leftrightarrow \qquad
      \cos\theta = \pm 1 \qquad \Leftrightarrow \qquad
      \theta \in \{0, \pi\}.
      $$
    }
  %
  \item Dans le cas général, quelle action (géométrique) $A$ opère-t-elle sur un vecteur de $\Rbb^3$ ?
    \solution{
      Pour $v = [x \; y \; z]^\top$, on a
      $$
      A v 
      = \left[\begin{array}{ccc}
                \cos\theta & -\sin\theta & 0 \\
                \sin\theta & \cos\theta & 0 \\
                0 & 0 & 1
              \end{array}\right] 
        \left[\begin{array}{c} x \\ y \\ z \end{array}\right]
      = \left[\begin{array}{c} x \cos\theta - y \sin\theta \\ x \sin\theta + y \cos\theta \\ z \end{array}\right]
      $$
      et, notamment, 
      $$
      \|A v\|_2 
      = \sqrt{(x \cos\theta - y \sin\theta)^2 + (x \sin\theta + y \cos\theta) + z^2}
      = \sqrt{x^2 + y^2 + z^2}
      = \|v\|_2 
      $$
      et
      $$
      \cos(<(x, y), (x \cos\theta - y \sin\theta, x \sin\theta + y \cos\theta)>)
      = \frac{(x^2 + y^2)\cos\theta}{(x^2 + y^2)}
      = \cos\theta
      $$
      L'application linéaire associée à $A$ fait pivoter les coordonnées $(x, y)$ d'un angle $\theta$ en laissant $z$ inchangée : il s'agit d'une rotation (d'angle $\theta$) autour de l'axe vertical.
     }
  %
\end{enumerate}



