%-------------------------------------------------------------------------------
\subsubsection{BGW géométrique avec excès de zéro}
%-------------------------------------------------------------------------------

\begin{exercise}[Probabilités] ~ \label{exam:Proba}
  Les organismes d’une espèce asexuée se reproduisent suivant un processus de Bienaymé–Galton–Watson, où chaque individu laisse à la génération suivante un nombre aléatoire $X$ d’individus. \\
  On note $p_k = \Pr\{X = k\}$ et on suppose que 
  $$
  p_0 = b
  \qquad \text{et} \qquad 
  p_k = (1 - b) (1 - a) a^{k-1} \text{ pour } k \geq 1.
  $$
  \begin{enumerate}
    \item Montrer que la fonction génératrice de $X$ vaut
    $$
    f(s) := \Esp(s^X) = b + (1-b)(1-a) \frac{s}{1 - as}.
    $$
    \solution{Par définition, on a
    \begin{align*}
      f(s) 
      & =\Esp(s^X) = \sum_{k \geq 0} p_k s^k 
      = b + (1-b) \sum_{\textcolor{red}{k \geq 1}} (1 - a) a^{k-1} s^k \\
      & = b + (1 - b) s \sum_{k \geq 1} (1 - a) (as)^{k-1}
      = b + (1 - b) s \sum_{k \geq 0} (1 - a) (as)^k
    \end{align*}
    où on reconnaît la fonction génératrice d'une variable aléatoire géométrique : 
    $$
    \sum_{k \geq 0} (1 - a) (as)^k = \frac{1 - a}{1 - as},
    $$
    qui donne le résultat.
    }
    \item Déterminer la valeur de la probabilité, notée $q$, qu'une population de cette espèce issue d’un seul individu fondateur s'éteigne.
    \solution{On sait que $q$ est le plus petit point fixe de la fonction $f$ dans l'intervalle $(0, 1]$. Il faut donc résoudre $f(s) = s$
    \begin{align*}
      \Leftrightarrow \qquad 
      b(1 - as) + (1-b)(1-a) s & = a - as^2 \\
      \Leftrightarrow \qquad
      as^2 - (a+b) s + b & = 0
    \end{align*}
    dont le discriminant vaut $\Delta = (a+b)^2 - 4ab = (a-b)^2$ et les solutions sont donc
    $$
    s = \frac{(a+b) \pm (a-b)}{2}
    \qquad \Leftrightarrow \qquad
    s \in \left\{\frac{b}a, 1\right\}.
    $$
    La probabilité d'extinction est donc $q = b/a$ si $b < a$ et 1 sinon.
    }
  \end{enumerate}
  On introduit un nombre aléatoire $N$ d'individus de cette espèce dans une île. On note $g$ la fonction génératrice de $N$. On suppose que les descendances des $N$ individus fondateurs évoluent indépendamment les unes des autres. On s'intéresse à la probabilité $p$ de colonisation, c'est-à-dire à la probabilité pour que cette population ne s'éteigne pas.
  \begin{enumerate}
    \setcounter{enumi}{2}
    \item A quelle condition $p$ est-elle non nulle ?
    \solution{La colonisation réussit si la descendance d'au moins un individu fondateur ne s'éteint pas, ce qui se produit avec probabilité $1 - q^N$. $p$ est donc nulle pour tout $N$ si $q = 1$ et non nulle ssi $q < 1$, c'est à dire si $b < a$.}
    \item Exprimer $p$ en fonction de $q$. 
    \solution{$1 - q^N$ est la probabilité de colonisation conditionnelle à l'effectif initial $N$. La probabilité de colonisation vaut donc
    $$
    p = \Esp(1 - q^N) = 1 - \Esp(q^N) = 1 - g(q).
    $$}
    \item Déterminer $p$ si $N = 1$ et si $N$ est distribué comme $X$. Commenter.
    \solution{
    \begin{description}
     \item[$N = 1$ :] on a immédiatement $p = 1 - q$. 
     \item[$N \overset{\Delta}{=} X$:] alors $g = f$ et donc
      $$
      p = 1 - g(q) = 1 - f(q) = 1 - q
      $$
      puisque $q$ est un point fixe de $f$ ($f(q) = q$). 
    \end{description}
    L'égalité des résultats n'est pas surprenante puisque le second cas correspond au premier décalé d'une génération.}
  \end{enumerate}
\end{exercise}

